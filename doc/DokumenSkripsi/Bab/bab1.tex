%versi 2 (8-10-2016) 
\chapter{Pendahuluan}
\label{chap:intro}
   
\section{Latar Belakang}
\label{sec:label}
\par Foundation merupakan kerangka kerja atau \textit{framework} yang responsive untuk mempercepat pembangunan \textit{front-end} atau antarmuka situs dan aplikasi. \textit{Framework} ini dapat bekerja pada perangkat apa pun(\textit{smartphone} dan desktop) dengan menggunakan komponen yang sudah diuji. Foundation dijalankan pertama kali dalam sebuah proyek pada tahun 2008. Kemudian tahun 2011, untuk mendorong pertumbuhan desain website yang responsive maka \textit{framework} ini bersifat \textit{open-source} dan dapat digunakan oleh publik.
\cite{zurbfoundation:17}.
\par Sejak Bootstrap diluncurkan pada Agustus 2011, \textit{framework} ini mulai populer. Bootstrap telah berkembang sepenuhnya menjadi proyek yang digerakkan oleh CSS untuk menggunakan sejumlah plugin JavaScript dan ikon yang sejalan dengan \textit{forms} dan \textit{buttons}. Pada dasarnya, ini memungkinkan untuk mendesain web yang responsif. Bootstrap memiliki fitur grid 12-kolom dan \textit{container} selebar 940px yang kuat. Salah satu yang menarik adalah \textit{developer} dapat menyesuaikan pembangunan sesuai dengan kebutuhan, seperti memilih fitur CSS dan JavaScript yang ingin disertakan dalam situs. \cite{bootstrap:19}
\par BlueTape merupakan aplikasi berbasis \textit{web} yang berfungsi mengolah beberapa kebutuhan administrasi fakultas secara \textit{paperless} yang digunakan dalam lingkungan FTIS UNPAR.  Aplikasi ini mempunyai fitur untuk manajemen transkrip nilai, perubahan kuliah dan jadwal dosen. \textit{Framework} yang digunakan dalam aplikasi BlueTape ada dua yaitu CodeIgniter(bagian ~\ref{sec:codeigniter}) dan Foundation(bagian ~\ref{sec:foundation}).
\par Dua faktor dilakukannya proses migrasi pada BlueTape menggunakan Bootstrap 4. Pertama Foundation 6 dikhususkan untuk situs dan email, sedangkan pada Bootstrap 4 lebih berfokus pada adanya tema yang beragam sehingga \textit{developer} memiliki banyak pilihan. Kedua Bootstrap 4 memiliki popularitas yang lebih tinggi dari Foundation 4 dilihat dari jumlah pencarian kata kunci pada Google \footnote{https://blog.templatetoaster.com/bootstrap-vs-foundation/}. Pada skripsi ini akan dirubah keseluruhan  antarmuka untuk setiap modul yang ada di dalam aplikasi BlueTape menggunakan \textit{framework} Bootstrap 4. Saat ini, setiap view menggunakan template yang menampilkan nama \textit{module}, menu navigasi, dan \textit{flash message}.


\section{Rumusan Masalah}
\label{sec:rumusan}
Rumusan masalah yang akan dibahas dalam penelitian ini:
\begin{enumerate}
	\item Bagaimana mengubah \textit{template} manajemen cetak transkrip, manajemen perubahan kuliah dan manajemen jadwal dosen dari framework Foundation 6 ke Bootstrap 4.
	\item Bagaimana \textit{plugin} yang ada di BlueTape saat ini dapat berjalan pada \textit{Bootstrap 4}.
\end{enumerate}

\section{Tujuan}
\label{sec:tujuan}
Tujuan yang ingin dicapai dalam penelitian ini :

	\begin{enumerate}	
	\item Mengubah \textit{template} cetak transkrip nilai, \textit{template} manajemen cetak transkrip, \textit{template} perubahan kuliah, \textit{module} manajemen perubahan kuliah, \textit{template} entri jadwal dosen dan \textit{template} lihat jadwal dosen dengan \textit{framework Bootstrap 4}.
	\item Memastikan \textit{plugin} yang ada di BlueTape saat ini dapat berjalan dengan baik pada \textit{Bootstrap 4}.
	\end{enumerate}

\section{Batasan Masalah}
\label{sec:batasan}
Dalam penelitian ini ditetapkan batasan-batasan masalah sebagai berikut.
\begin{enumerate}	
	\item Kegiatan migrasi tidak mengubah struktur database dan file yang berisi fungsi-fungsi \textit{create, read, update, delete} (CRUD).
	\item Kegiatan migrasi tidak menambah fitur baru, hanya mengubah penggunaan framework \textit{ Foundation 6} sesuai dengan fitur yang sudah ada menggunakan \textit{Bootstrap 4}.
	\item Proses pengembangan masih berjalan ketika pengerjaan skripsi dilakukan sehingga untuk mempermudah maka penulis menggunakan kode yang diunggah pada tanggal 16 Agustus 2019 untuk proses migrasi.	
	\end{enumerate}		


\section{Metodologi}
\label{sec:metlit}
Metode penelitian yang digunakan dalam skripsi ini adalah :
\begin{enumerate}
	\item Studi literatur memahami mengenai:
	\begin{enumerate}
		\item \textit{framework} CodeIgniter
		\item \textit{framework} Bootstrap 4
		\item \textit{framework} Foundation 6 dan \textit{plugin} - \textit{plugin} nya.
	\end{enumerate}
	\item  Membangun antarmuka sesuai tampilan website BlueTape. Proses pembuatan antarmuka dibagi menjadi 3 tahap:
	\begin{enumerate}
		\item Analisis elemen dan kelas yang digunakan dalam antarmuka dengan Foundation 6.
		\item Penjabaran elemen beserta kelas-kelas pada Foundation 6 dan Bootstrap 4.
		\item Implementasi kode dengan kelas-kelas pada Bootstrap 4 berdasarkan penjabaran .
	\end{enumerate}
	\item Menyusun dokumen skripsi. Detil pengerjaan dijelaskan pada bagian ~\ref{sec:sispem}.
\end{enumerate}


\section{Sistematika Pembahasan}
\label{sec:sispem}
Untuk penulisan skripsi ini dibagikan dalam 5 bab sebagai berikut :
\doublespacing
\begin{singlespace}
\noindent Bab Pendahuluan \\
Bab 1 menjelaskan mengenai latar belakang, rumusan masalah, tujuan, batasan masalah, metodologi penelitian dan sistematika penulisan.
\end{singlespace}

\begin{singlespace}
\noindent Bab Landasan Teori \\
Bab 2 berisi dasar-dasar teori pembuatan antarmuka BlueTape. Dasar-dasar teori yang digunakan diantaranya adalah pemrograman PHP, \textit{framework CodeIgniter}, \textit{framework Foundation 6}, \textit{framework Bootstrap 4}.
\end{singlespace}

\begin{singlespace}
\noindent Bab Analisis \\
Bab 3 berisi analisis komponen yang digunakan dalam website dan perbedaan antara \textit{framework} Foundation 6 dan Bootstrap 4.
\end{singlespace}

\begin{singlespace}
\noindent Bab Implementasi \\
Bab 4 berisi mengenai tampilan sebelum dan sesudah implementasi komponen dengan menggunakan Bootstrap 4.
\end{singlespace}

\begin{singlespace}
\noindent Bab Kesimpulan dan Saran \\
Bab 5 berisi mengenai kesimpulan dan saran dari pengerjaan skripsi.
\end{singlespace}
