%versi 2 (8-10-2016) 
\chapter{Pendahuluan}
\label{chap:intro}
   
\section{Latar Belakang}
\label{sec:label}

%Bagian ini akan diisi dengan apa yang melatarbelakangi pembuatan template skripsi ini.
%Termasuk juga masalah-masalah yang akan dihadapi untuk membuatnya, termasuk kurangnya kemampuan penguasaan \LaTeX{} sehingga template ini dibuat dengan mengandalkan berbagai contoh yang tersebar di dunia maya, yang digabung-gabung menjadi satu jua.
%Bagian lain juga akan dilengkapi, untuk sementara diisi dengan lorem ipsum versi bahasa inggris.

BlueTape merupakan aplikasi berbasis \textit{web} yang berfungsi mengolah beberapa kebutuhan administrasi fakultas secara \textit{paperless} yang digunakan dalam lingkungan FTIS UNPAR.  Aplikasi ini mempunyai fitur untuk manajemen transkrip nilai, perubahan kuliah dan jadwal dosen. \textit{Framework} yang digunakan dalam aplikasi BlueTape ada dua yaitu \textit{Codeigniter} dan \textit{Zurb Foundation}.  

Meskipun \textit{open-source}, saat ini \textit{Zurb Foundation} tidak sepopuler \textit{framework} \textit{Bootstrap}. 
\textit{Bootstrap} adalah \textit{ Javascript framework} yang didesain untuk membantu membangun komponen \textit{user interface} yang terdiri dari \textit{CSS, JavaScript/jQuery}, dan \textit{glyphicons}. Pembangunan \textit{website} yang lebih cepat dan besarnya komunitas yang ada berdampak pada banyaknya pengembang \textit{web} yang memanfaatkan \textit{framework Bootstrap}. Sehingga jumlah proyek yang dihasilkan oleh\textit{ framework Bootstrap} lebih banyak dibanding \textit{Zurb Foundation}. Dengan mengunakan framework \textit{CSS} maka akan mempercepat pengembangan situs web, karena framework CSS dilengkapi beberapa set kode yang dapat digunakan berulang kali (\textit{reuse}) dan memiliki kompatibilitas antar-browser. 

Pada skripsi ini akan dirubah keseluruhan  antarmuka untuk setiap modul yang ada di dalam aplikasi BlueTape menggunakan \textit{framework} Bootstrap 4. Saat ini, setiap view menggunakan template yang menampilkan nama \textit{module}, menu navigasi, dan \textit{flash message} (bila diperlukan).


\section{Rumusan Masalah}
\label{sec:rumusan}
Rumusan masalah yang akan dibahas dalam penelitian ini:
\begin{enumerate}
	\item Bagaimana merubah \textit{template} manajemen cetak transkrip, manajemen perubahan kuliah dan manajemen jadwal dosen dari framework \textbf{Zurb Foundation} ke \textbf{Bootstrap 4}
	\item Bagaimana mengimplentasikan \textit{plugin} yang tersedia di dalam \textit{Bootstrap 4}.
\end{enumerate}

\section{Tujuan}
\label{sec:tujuan}
Tujuan yang ingin dicapai dalam penelitian ini :

	\begin{enumerate}	
	\item Merubah \textit{template} cetak transkrip nilai, \textit{template} manajemen cetak transkrip, \textit{template} perubahan kuliah, \textit{module} manajemen perubahan kuliah, \textit{modul} entri jadwal dosen dan \textit{module} lihat jadwal dosen dengan \textit{framework Bootstrap 4}.
	\item Mengimplentasikan \textit{plugin} yang tersedia dalam \textit{library} Bootstrap 4.
	\end{enumerate}

\section{Batasan Masalah}
\label{sec:batasan}
%Untuk mempermudah pembuatan template ini, tentu ada hal-hal yang harus dibatasi, misalnya saja bahwa template ini bukan berupa style \LaTeX{} pada umumnya (dengan alasannya karena belum mampu jika diminta membuat seperti itu)
%
%\dtext{8}
Dalam penelitian ini ditetapkan batasan-batasan masalah sebagai berikut.
\begin{enumerate}	
	\item Aplikasi ini tidak merubah struktur database dan file yang berisi fungsi-fungsi CRUD.
	\item Aplikasi ini tidak menambah tampilan baru, hanya merubah penggunaan framework \textit{Zurb Foundation} sesuai dengan tampilan yang sudah ada menggunakan \textit{Bootstrap 4}
	\end{enumerate}		


\section{Metodologi}
\label{sec:metlit}
%Tentunya akan diisi dengan metodologi yang serius sehingga templatenya terkesan lebih serius.
%
%\dtext{9}
Metode penelitian yang digunakan dalam skripsi ini adalah :
\begin{enumerate}
	\item Studi literatur memahami mengenai :
	\begin{enumerate}
		\item \textit{framework} CodeIgniter
		\item \textit{framework} Bootstrap 4
		\item \textit{framework} Zurb Foundation dan plugin - plugin nya.
	\end{enumerate}
	\item  Membangun antarmuka sesuai tampilan website BlueTape. Proses pembuatan antarmuka dibagi menjadi 3 tahap :
	\begin{enumerate}
		\item Analisis tampilan antarmuka website BlueTape
		\item Perancangan tampilan antarmuka
		\item Implementasi		
	\end{enumerate}
\end{enumerate}


\section{Sistematika Pembahasan}
\label{sec:sispem}
%Rencananya Bab 2 akan berisi petunjuk penggunaan template dan dasar-dasar \LaTeX.
%Mungkin bab 3,4,5 dapt diisi oleh ketiga jurusan, misalnya peraturan dasar skripsi atau pedoman penulisan, tentu jika berkenan.
%Bab 6 akan diisi dengan kesimpulan, bahwa membuat template ini ternyata sungguh menghabiskan banyak waktu.
Untuk penulisan skripsi ini akan dibagikan dalam .. bab sebagai berikut :

\noindent Bab Pendahuluan

\noindent Bab 1 menjelaskan mengenai latar belakang, rumusan masalah, tujuan, batasan masalah, metodologi penelitian dan sistematika penulisan.


\noindent Bab Landasan Teori

\noindent Bab 2 berisi dasar-dasar teori pembuatan antarmuka BlueTape. Dasar-dasar teori yang digunakan diantaranya adalah pemrograman PHP, \textit{framework Codeigniter}, \textit{framework Zurb Foundation}, \textit{framework Bootstrap 4}.


\noindent Bab Analisis 

\noindent Bab 3 berisi analisis antarmuka yang sudah ada dan analisis antarmuka usulan.


\noindent Bab Perancangan antarmuka

\noindent Bab 4 program dan perancangan kelas-kelas program.


\noindent Bab Implementasi 

\noindent Bab 5 membahas mengenai pembuatan template utama aplikasi BlueTape yaitu , pembuatan menu aplikasi dan hasil eksekusi tampilan aplikasi. 


\noindent Bab Kesimpulan dan saran.

\noindent Bab 6 berisi kesimpulan setelah mengerjakan skripsi ini dan saran yang diberikan.
