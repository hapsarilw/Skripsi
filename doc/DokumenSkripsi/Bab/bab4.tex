\chapter{Implemetasi Website BlueTape dengan Bootstrap 4}
Bab 4 menjelaskan implementasi website BlueTape dengan Framework Bootstrap 4. Pertama akan dijelaskan file apa saja yang dirubah. Kemudian akan disajikan gambar visual mengenai tampilan terkini website beserta kelas yang digunakan dan terakhir detail perubahan kode yang dijelaskan dalam kode \texttt{diff}.
\section{Daftar File yang Diubah dan Diganti pada Folder}
Bagian ~\ref{lst:daftarfile} menjelaskan isi file website BlueTape dan status file yang diganti dan diubah.
\begin{lstlisting}[basicstyle=\ttfamily, frame=single, caption=Perubahan isi folder BlueTape,
columns=fullflexible, keepspaces=true, breaklines=true, label={lst:daftarfile}]
.
|-- nbproject
|-- vendor
|-- www
|   |-- application
|   |   |-- cache
|   |   |-- config
|   |   |   |-- auth.php // Kode diubah
|   |   |   |-- database.php // Kode diubah
|   |   |   |-- ..dst
|   |   |-- controller
|   |   |   |-- Auth.php
|   |   |   |-- EntriJadwalDosen.php
|   |   |   |-- index.html
|   |   |   |-- LihatJadwalDosen.php
|   |   |   |-- Migrate.php
|   |   |   |-- PerubahanKuliahManage.php // Kode diubah
|   |   |   |-- PerubahanKuliahRequest.php // Kode diubah
|   |   |   |-- TranskripManage.php // Kode diubah
|   |   |   |-- TranskripRequest.php // Kode diubah
|   |   |-- core
|   |   |-- helper
|   |   |-- hooks
|   |   |-- languange
|   |   |-- libraries
|   |   |-- logs
|   |   |-- migration
|   |   |-- models
|   |   |--third_party
|   |   |--views
|   |   |   |-- auth
|   |   |   |   |-- index.html
|   |   |   |   |-- login.php // Kode diubah
|   |   |   |-- EntriJadwalDosen
|   |   |   |   |-- main.php // Kode diubah
|   |   |   |-- errors
|   |   |   |-- LihatJadwalDosen
|   |   |   |   |-- main.php // Kode diubah
|   |   |   |-- PerubahanKuliahManage
|   |   |   |   |-- email.php
|   |   |   |   |-- index.html
|   |   |   |   |-- main.php // Kode diubah
|   |   |   |   |-- printview.php
|   |   |   |-- PerubahanKuliahRequest
|   |   |   |   |-- email.php
|   |   |   |   |-- index.html
|   |   |   |   |-- main.php // Kode diubah
|   |   |   |-- templates
|   |   |   |   |-- flasmessage.php // Kode diubah
|   |   |   |   |-- head_loggedin.php // Kode diubah
|   |   |   |   |-- script_foundation.php // Kode diubah
|   |   |   |   |-- topbar_loggedin.php // Kode diubah
|   |   |   |-- TranskripManage
|   |   |   |   |-- email.php
|   |   |   |   |-- index.html
|   |   |   |   |-- main.php // Kode diubah
|   |   |   |-- TranskripRequest
|   |   |   |   |-- email.php
|   |   |   |   |-- index.html
|   |   |   |   |-- main.php // Kode diubah
|   |   |   |-- index.html
|   |   |--.htacces
|   |   |--index.html
|   |-- public
|   |   |-- fonts
|   |   |   |-- OFL.txt
|   |   |   |-- TitilliumWeb-Black.ttf
|   |   |   |-- TitilliumWeb-Bold.ttf
|   |   |   |-- TitilliumWeb-BoldItalic.ttf
|   |   |   |-- TitilliumWeb-ExtraLight.ttf
|   |   |   |-- TitilliumWeb-ExtraLightItalic.ttf
|   |   |   |-- TitilliumWeb-Italic.ttf
|   |   |   |-- TitilliumWeb-Light.ttf
|   |   |   |-- TitilliumWeb-LightItalic.ttf
|   |   |   |-- TitilliumWeb-Regular.ttf
|   |   |   |-- TitilliumWeb-SemiBold.ttf
|   |   |   |-- TitilliumWeb-SemiBoldItalic.ttf
|   |   |-- img
|   |   |   |-- logo.png // File tetap
|   |   |-- lib 
|   |   |   |-- css
|   |   |   |   |-- bootstrap.css // File baru ditambahkan
|   |   |   |   |-- bootstrap-grid.css // File baru ditambahkan
|   |   |   |   |-- bootstrap-reboot.css // File baru ditambahkan
|   |   |   |   |-- app.css 
|   |   |   |   |-- foundation.css 
|   |   |   |   |-- foundation-datetimepicker.min.css 
|   |   |   |   |-- foundation-flex.css 
|   |   |   |   |-- foundation-icon.css 
|   |   |   |   |-- foundation-icon.eot 
|   |   |   |   |-- foundation-icon.svg 
|   |   |   |   |-- foundation-icon.ttf 
|   |   |   |   |-- foundation-icon.woff 
|   |   |   |-- fontawesome
|   |   |   |   |-- fontawesome.css // File baru ditambahkan
|   |   |   |-- jquery
|   |   |   |   |--jquery-3.4.1.min.js // File baru ditambahkan
|   |   |   |-- js
|   |   |   |   |-- bootstrap.js // File baru ditambahkan
|   |   |   |   |-- bootstrap.bundle.js // File baru ditambahkan
|   |   |   |   |-- vendor // Folder dihapus
|   |   |   |   |-- app.js // Folder dihapus
|   |   |   |   |-- foundation.js // Folder dihapus
|   |   |   |-- xdan-datetimepicker 
|   |   |   |   |-- jquery.datetimepicker.full.min.js
|   |   |   |   |-- jquery.datetimepicker.min.js
|   |   |   |   |-- jquery.datetimepicker.min.js
|   |   |-- webfonts // Folder ditambahkan, berfungsi untuk simpan ikon.
|   |-- system
|   |-- .htaccess
|   |-- .composer.json
|   |-- .contributig.md
|   |-- .index.php
|   |-- .licence.txt
|   |-- readme.rst
|   |-- web.config
.
\end{lstlisting}

\section{Pengaturan Akses bagi \textit{admin}}
Sebelumm menjalankan website, \textit{admin} akan menyunting email admin dan database agar lebih mudah unutk diakses. Hasil perubahan kode ditampilkan dalam berbentuk diff, dimana bagian berwarna merah merupakan kode lampau(Foundation 6) dan bagian berwarna hijau merupakan kode terkini  (Bootstrap 4).

\subsection{Perubahan Akses URL}
Saat admin pertama kali menjalankan aplikasi CodeIgniter maka diperlukan konfigurasi \texttt{base\_url}. Sehingga URL baru terbentuk dan dapat mengakses \textit{resource} yang ada pada direktori \texttt{root}. Penggunaan \texttt{base\_url} dalam website dijelaskan pada kode ~\ref{lst:config}\\

\begin{lstlisting}[language=diff, caption=Perubahan file /config/config.php,  basicstyle=\ttfamily, frame=single,
columns=fullflexible, keepspaces=true, breaklines=true, label={lst:config}]
--- a/www/application/config/config.php
+++ b/www/application/config/config.php

- $config['base_url'] = 'https://bluetape.azurewebsites.net';
+ $config['base_url'] = 'http://127.0.0.1/';
\end{lstlisting}

\subsection{Perubahan Akses Email Admin}

Kode ~\ref{lst:modules} mengarahkan admin ke halaman login setelah aplikasi berjalan, karena website menggunakan \textit{Google API Console} untuk proses autentikasi maka perlu ditambahkan email admin baru pada file \path{www/application/config/modules.php}. Email ini telah didaftarkan sebelumnya di API tersebut, sehingga admin dapat menggunakan seluruh fitur pada website. \\

\begin{lstlisting}[language=diff, caption=Perubahan file /config/modules.php,  basicstyle=\ttfamily, frame=single,
columns=fullflexible, keepspaces=true, breaklines=true, label={lst:modules}]
--- a/www/application/config/modules.php
+++ b/www/application/config/modules.php

@@ -20,12 +20,12 @@
$config['roles'] = array(
-  'root' => array('pascal@unpar.ac.id', 'shao.wei@unpar.ac.id'),
+  'root' => array('pascal@unpar.ac.id', 'shao.wei@unpar.ac.id', 'amihapsahapsa@gmail.com'),

\end{lstlisting}

\section{Perubahan Kode pada Tampilan}

Bagian ini akan menjelaskan hasil implementasi dalam bentuk \textit{diff}. Diff membedakan kode berdasarkan warna dimana bagian berwarna merah merupakan kode lampau(implementasi dengan Foundation 6) dan bagian berwarna hijau merupakan kode terkini  (implementasi dengan Bootstrap 4).
   
\subsection{Penggunaan \textit{library} Bootstrap 4 pada Website}

Library Bootstrap 4 yang sudah dimasukan pada CodeIgniter akan diintegrasikan pada file www/application/config/modules.php dalam bentuk file \texttt{.js} dan \texttt{head\_loggedin.php} dalam bentuk file \texttt{.css}. File ini nantinya akan dipakai keseluruh halaman website kecuali halaman login.\\

Pertama kode ~\ref{lst:scriptfoundation} memanggil file js dan jquery untuk Bootstrap serta plugin \texttt{xdan-datetimepicker} agar dapat di\textit{load}. \\

\begin{lstlisting}[language=diff, caption=Penambahan \path{\views\templates\script_foundation.php},  basicstyle=\ttfamily, frame=single,
columns=fullflexible, keepspaces=true, breaklines=true, label={lst:scriptfoundation}]
--- a\www\application\views\templates\script_foundation.php		
+++ b\www\application\views\templates\script_foundation.php	

@@ -1,7 +1,5 @@
<?php
defined('BASEPATH') OR exit('No direct script access allowed');
- ?><script src="/public/js/vendor/jquery.min.js"></script>
- <script src="/public/js/vendor/what-input.min.js"></script>
- <script src="/public/js/foundation.min.js"></script>
- <script src="/public/js/app.js"></script>
+ ?><script src="/public/lib/jquery/jquery-3.4.1.min.js"></script>
+ <script src="/public/lib/js/bootstrap.js"></script>
<script src="/public/lib/xdan-datetimepicker/jquery.datetimepicker.full.min.js"></script>
\end{lstlisting}

Kedua file css akan dipanggil, ada tiga macam file yang dipanggil: 
\begin{itemize}
	\item \textit{library} Bootstrap 4.
	\item Plugin \texttt{xdan-datetimepicker}.
	\item \textit{library} Font Awesome.
\end{itemize}
File yang berisi kode ~\ref{lst:headloggedin} akan digunakan pada seluruh halaman website kecuali halaman login.
 
\begin{lstlisting}[language=diff, caption=Perubahan file \path{\views\templates\head_loggedin.php},  basicstyle=\ttfamily, frame=single,
columns=fullflexible, keepspaces=true, breaklines=true, label={lst:headloggedin}]
--- a\www\application\views\templates\head_loggedin.php
+++ b\www\application\views\templates\head_loggedin.php

@@ -2,11 +2,10 @@
defined('BASEPATH') OR exit('No direct script access allowed');
?><head>
<meta charset="utf-8" />
<meta http-equiv="x-ua-compatible" content="ie=edge">
<meta name="viewport" content="width=device-width, initial-scale=1.0" />
<title><?= $this->config->item('module-names')[$currentModule] ?></title>
-    <link rel="stylesheet" href="/public/css/foundation.css" />
-    <link rel="stylesheet" href="/public/css/foundation-icons.css" />
-    <link rel="stylesheet" href="/public/css/app.css" />
-    <link rel="stylesheet" href="/public/lib/xdan-datetimepicker/jquery.datetimepicker.min.css" />
+    <link rel="stylesheet" href="/public/lib/css/bootstrap.css" />
+    <link rel="stylesheet" href="/public/lib/fontawesome/fontawesome.css">
+    <link rel="stylesheet" href="public/lib/xdan-datetimepicker/jquery.datetimepicker.min.css">
\end{lstlisting}

\subsection{Halaman Login}
Pemanggilan file css dilakukan kembali pada file \texttt{login.php} untuk digunakan pada halaman login website BlueTape. Selain itu file HTML berada di file ini, penjelasan tertera pada kode ~\ref{lst:login}
\begin{lstlisting}[language=diff, caption=Perubahan file \path{\views\auth\login.php},  basicstyle=\ttfamily, frame=single,
columns=fullflexible, keepspaces=true, breaklines=true, label={lst:login}]
--- a\www\application\views\auth\login.php	
+++ b\www\application\views\auth\login.php	
@@ -4,26 +4,29 @@

-        <link rel="stylesheet" href="/public/css/foundation.css" />
-        <link rel="stylesheet" href="/public/css/app.css" />
+        <link rel="stylesheet" href="/public/lib/css/bootstrap.css" />
+        <link rel="stylesheet" href="/public/lib/css/bootstrap-grid.css" />
+        <link rel="stylesheet" href="/public/lib/css/bootstrap-reboot.css" />
+        <link rel="stylesheet" href="/public/lib/fontawesome/fontawesome.css" />


###    Implementasi Grid
-        	<div class="row">
-        		<div class="large-6 large-centered columns centered">
+        <div class="container">
+        	<div class="row justify-content-center">
+        		<div class="col-lg-6">

#    Implementasi Posisi Text
- 		<a href="<?= $authURL; ?>" class="button expand">Login dengan Google</a><br/><br/>
- 		<a class="text-center" href="https://github.com/ftisunpar/BlueTape/wiki/UserGuide" target="_blank">Petunjuk Penggunaan</a>
+ 		<a href="<?= $authURL; ?>" class="btn btn-primary btn-lg">Login dengan Google</a><br/><br/>
+ 		<a href="https://github.com/ftisunpar/BlueTape/wiki/UserGuide" target="_blank">Petunjuk Penggunaan</a>
\end{lstlisting}

\noindent Penjelasan kode diatas tertera pada tabel ~\ref{tabelKodeManajemenPerubahanKuliah}:
\begin{table}[H]
	\centering
	\begin{tabularx}{\textwidth}{lX}
		\toprule
		Implementasi     & Penjelasan \\
		\midrule
		\textbf{Pemanggilan file} & File css dari Foundation akan diganti dnegan file css dari Bootstrap 4.\\
		\textbf{Grid} & Dalam Foundation 6 untuk membuat satu baris konten hanya menggunakan kelas \texttt{.row}, namun di Bootstrap 4 perlu ditambahakan kelas \texttt{.container}.\\
		 & Untuk size sama - sama menggunakan lebar 6 grid yang posisi nya diletakkan ditengah website.\\
		\textbf{Text} & Teks akan diletakkan ditengah, hanya di Foundation 6 saja perlu menginisiasi kelas \texttt{.text-center} pada Bootstrap 4 tidak perlu.\\
		\bottomrule
	\end{tabularx}%
	\caption{Penjelasan kode yang dipakai pada halaman manajemen perubahan kuliah.}
	\label{tabelKodeManajemenPerubahanKuliah}
\end{table}

Implementasi notifikasi user terdapat pada kode ~\ref{lst:flashmessage}.

\begin{lstlisting}[language=diff, caption=Perubahan file \path{\views\templates\flashmessage.php},  basicstyle=\ttfamily, frame=single,
columns=fullflexible, keepspaces=true, breaklines=true, label={lst:flashmessage}]
--- a\www\application\views\templates\flashmessage.php
+++ b\www\application\views\templates\flashmessage.php

@@ -1,10 +1,9 @@
<?php
defined('BASEPATH') OR exit('No direct script access allowed');
- ?><div class="row">
+ ?>
<?php if (isset($_SESSION['error'])): ?>
-        <div class="callout alert"><?= $_SESSION['error'] ?></div>
+        <div class="alert alert-danger" role="alert"><?= $_SESSION['error'] ?></div>
<?php endif; ?>
<?php if (isset($_SESSION['info'])): ?>
-        <div class="callout primary"><?= $_SESSION['info'] ?></div>
+        <div class="alert alert-primary" role="alert"><?= $_SESSION['info'] ?></div>
<?php endif; ?>
- </div>

\end{lstlisting}

\noindent Penjelasan kode diatas dijabarkan dalam tabel ~\ref{tabelKodeLogin}:
\begin{table}[H]
	\centering
	\begin{tabularx}{\textwidth}{lX}
		\toprule
		Implementasi     & Penjelasan \\
		\midrule
		\textbf{Alert} & Labelling menggunakan \texttt{.callout} pada Foundation 6. Boostrap 4 mrnggunakan kelas \texttt{.alert} dan atribut \texttt{role}.\\
		\bottomrule
	\end{tabularx}%
	\caption{Penjelasan kode konversi halaman login.}
	\label{tabelKodeLogin}
\end{table}

\subsection{Menu Navigasi}
Kode ~\ref{lst:topbarloggedin} menjelaskan menu navigasi secara umum terdiri dari kelas untuk judul website, letak menu dan menu yang sedang aktif.
\begin{lstlisting}[language=diff, caption=Perubahan file \path{\views\templates\topbar_loggedin.php} ,  basicstyle=\ttfamily, frame=single,
columns=fullflexible, keepspaces=true, breaklines=true, label={lst:topbarloggedin}]
--- a\www\application\views\templates\topbar_loggedin.php
+++ b\www\application\views\templates\topbar_loggedin.php

@@ -1,24 +1,23 @@
#    Implementasi Komponen Navbar
- <div class="title-bar" data-responsive-toggle="navigation-menu" data-hide-for="medium">
- 	<button class="menu-icon" type="button" data-toggle></button>
- <div class="title-bar-title"><img src="/public/img/logo.png" class="textsized" alt="B"/></div>
- </div>
-
- <div class="top-bar" id="navigation-menu">
-  <div class="top-bar-left">
-    <ul class="menu" data-responsive-menu="dropdown">
-      <li class="menu-text"><img src="/public/img/logo.png" class="textsized" alt="B"/></li>
+ 		
+ 			
+ <nav class="navbar navbar-expand-lg navbar-dark bg-dark">
+   <a class="navbar-brand" href="#"><img src="/public/img/logo.png" width="50"/></a>
+ 	<button class="navbar-toggler" type="button" data-toggle="collapse" data-target="#navbarSupportedContent" aria-controls="navbarSupportedContent" aria-expanded="false" aria-label="Toggle navigation">
+ 		<span class="navbar-toggler-icon"></span>
+ 	</button>
+ 		
+ <div class="collapse navbar-collapse" id="navbarSupportedContent">
+ 	<ul class="navbar-nav mr-auto">

#    Implementasi kelas .active
- <li<?= $module === $currentModule ? ' class="menu-active"' : '' ?>><a href="/<?= $module ?>"><?= $this->config->item('module-names')[$module] ?></a></li>
+ <li<?= $module == $currentModule ? ' class="nav-item active"' : ' class="nav-item "' ?>><a class="nav-link" href="/<?= $module ?>"><?= $this->config->item('module-names')[$module] ?></a></li>

#    Implementasi Komponen Navbar
- </div>
- <div class="top-bar-right">
- 	<ul class="menu">
- 		<li><a href="/auth/logout">Logout</a></li>
+ 	<ul class="navbar-nav ml-auto">
+ 		<li class="nav-item">
+ 		    <a class="nav-link" href="/auth/logout">Logout</a>
+ 		</li>

\end{lstlisting}

\noindent Catatan untuk kode ~\ref{lst:topbarloggedin} dijabarkan dalam tabel ~\ref{tabel:KodeLogin}:
\begin{table}[H]
	\centering
	\begin{tabularx}{\textwidth}{lX}
		\toprule
		Implementasi     & Penjelasan \\
		\midrule
		\textbf{Navbar} & Pada foundation 6 ikon dropdown belum berfungsi, sehingga pada Bootstrap 4 menu dibuat \textit{responsive} sehingga terdapat ikon yang menyimpan daftar menu.\\
		\bottomrule
	\end{tabularx}%
	\caption{Penjelasan kode konversi navigasi bar.}
	\label{tabel:KodeLogin}
\end{table}

\subsection{Halaman Permintaan Cetak Transkrip }
Perubahan halaman permintaan cetak transkrip dilampirkan dalam Kode ~\ref{lst:mainTranskripRequest}.
\begin{lstlisting}[language=diff, caption=Perubahan file \path{\views\TranskripRequest\main.php} ,  basicstyle=\ttfamily, frame=single,
columns=fullflexible, keepspaces=true, breaklines=true, label={lst:mainTranskripRequest}]
--- a\www\application\views\TranskripRequest\main.php
+++ b\www\application\views\TranskripRequest\main.php
 
@@ -4,88 +4,114 @@
# Implementasi grid, container dan card
- <div class="row">
- 	<div class="medium-12 column">
- 	<div class="callout">
- 	<h5>Permohonan Baru</h5>
+ 	<div class="container">
+            <div class="row ">
+ 	<div class="col">
+ 		<div class="card">
+ 			<div class="card-header">
+ 				Permohonan Baru
+ 			</div>
+ 			<div class="card-body">

#Implementasi form
- <div class="large-4 column">
- 	<label>Yang memohon:
- 		<input type="email" name="requestByEmail" value="<?= $requestByEmail ?>" readonly="readonly"/>
- 	</label>
+ <div class="col-lg-4">
+ 	<label class="col-form-label">Yang memohon:</label>
+ 	<input class="form-control" type="email" name="requestByEmail" value="<?= $requestByEmail ?>" readonly/>
- <div class="large-4 column">
- 	<label>NPM:
- 		<input type="text" value="<?= $requestByNPM ?>" readonly="readonly"/>
- 	</label>
+ <div class="col-lg-4">
+ 	<label class="col-form-label">NPM:</label>
+ 	<input class="form-control" type="text" value="<?= $requestByNPM ?>" readonly/>
- <div class="large-4 column">
- 	<label>Nama:
- 		<input type="text" name="requestByName" value="<?= $requestByName ?>" readonly="readonly"/>
- 	</label>
+ <div class="col-lg-4">
+ 	<label class="col-form-label">Nama:</label>
+ 	<input class="form-control" type="text" name="requestByName" value="<?= $requestByName ?>" readonly/>
 
<div class="row">
- <div class="large-4 column">
- 	<label>Tipe Transkrip:
- 		<select name="requestType">
+ <div class="col-lg-4">
+ 	<label class="col-form-label">Tipe Transkrip:</label>
+ 	<select class="form-control" name="requestType">
- 	</label>
- <div class="large-8 column">
- 	<label>Keperluan:
- 		<input type="text" name="requestUsage" required/>
- 	</label>
+ <div class="col-lg-8">
+ 	<label class="col-form-label">Keperluan:</label>
+ 	<input class="form-control" type="text" name="requestUsage" required/>
- <input type="submit" class="button" value="Kirim Permohonan">
+ <br>
+ <div class="row">
+ 	<div class="col-lg-12">
+ 		<input class="btn btn-primary" type="submit" class="button" value="Kirim Permohonan">
+ 	</div>
+ </div>

#Implementasi tabel histori permohonan
- <div class="callout">
- <h5>Histori Permohonan</h5>
- <table class="stack">
+ </div>
+ <br>
+ <div class="card">
+ 	<div class="card-header">
+ 		Histori Permohonan
+ 	</div>
+ 	<div class="card-body">
+ 		<table class="table table-striped">
- <th>ID</th>
- <th>Status</th>
- <th>Tanggal Permohonan</th>
- <th>Tipe Transkrip</th>
- <th>Tanggal Jawab/Cetak</th>
- <th>Keterangan</th>
- <th>Aksi</th>
+ <th scope="col">ID</th>
+ <th scope="col">Status</th>
+ <th scope="col">Tanggal Permohonan</th>
+ <th scope="col">Tipe Transkrip</th>
+ <th scope="col">Tanggal Jawab/Cetak</th>
+ <th scope="col">Keterangan</th>
+ <th scope="col">Aksi</th>
- <td>#<?= $request->id ?></td>
- <td><span class="<?= $request->labelClass ?> label"><?= $request->status ?></span></td>
+ <th>#<?= $request->id ?></th>
+ <td><span class="badge badge-<?= $request->labelClass ?>"><?= $request->status ?></span></td>

#Implementasi link ikon font awesome dan komponen modal
- <div class="reveal" id="detail<?= $request->id ?>" data-reveal>
+ <a data-toggle="modal" data-target="#lihatModal<?= $request->id ?>" id="detail<?= $request->id ?>">
+ 	<i class="fas fa-eye"></i>
+ </a>

#Implementasi komponen modal berisi tabel detail permohonan
- <h5>Detail Permohonan #<?= $request->id ?></h5>
- <table class="stack">
+ <div class="modal fade" id="lihatModal<?= $request->id ?>" tabindex="-1" role="dialog" aria-labelledby="exampleModalCenterTitle" aria-hidden="true">
+ 	<div class="modal-dialog modal-dialog-centered" role="document">
+ 	    <div class="modal-content">
+ 	        <div class="modal-header">
+ 	            <h5 class="modal-title" id="exampleModalLongTitle">Detail Permohonan #<?= $request->id ?></h5>
+ 	            <button type="button" class="close" data-dismiss="modal" aria-label="Close">
+ 	                <span aria-hidden="true">&times;</span>
+ 	            </button>
+ 	        </div>
+ 	        <div class="modal-body">
+ 	            <table class="table ">

@@ -102,13 +128,13 @@
- <th>Jawaban</th>
+                    <th scope="col">Jawaban</th>

@@ -119,17 +145,17 @@
- <button class="close-button" data-close aria-label="Tutup" type="button">
- <span aria-hidden="true">&times;</span>
- </button>
+         </div>
+     </div>

# Link modal pada foundation 6 diubah
- <a data-open="detail<?= $request->id ?>"><i class="fi-eye"></i></a>
+ </div> 
\end{lstlisting}
\noindent Catatan untuk kode ~\ref{lst:mainTranskripRequest} dijabarkan dalam tabel ~\ref{table:kodePermintaanCetakTranskrip}:
\begin{table}[H]
	\centering
	\begin{tabularx}{\textwidth}{lX}
		\toprule
		Implementasi     & Penjelasan \\
		\midrule
		\textbf{Konten} & Dalam Bootstrap 4 pembagian konten menggunakan \texttt{.card} yang terdiri dari \texttt{.card-header} dan \texttt{card-body} sedangkan di Foundation 6 hanya menggunakan kelas \texttt{.callout}. Lalu untuk form, field pada Foundation 6 hanya menggunakan tag \texttt{<input>} sedangkan pada Bootstrap 4 menggunakan kelas \texttt{.form-control}.\\
		\textbf{Modal} & Untuk modal pada Foundation 6 hanya memiliki satu bagian modal yaitu kelas \texttt{.reveal} sedangkan pada Bootstrap 4 menggunakan \texttt{.modal} yang terdiri dari \texttt{.modal-fade, modal-content, modal-header}.
		\bottomrule
	\end{tabularx}%
	\caption{Penjelasan kode permintaan cetak transkrip.}
	\label{table:kodePermintaanCetakTranskrip}
\end{table}

\subsubsection{Controller Permintaan Cetak Transkrip}
Pemberian label pada kolom 'status' menggunakan logika sederhana yang terletak pada controller, hasil dilampirkan dalam kode ~\ref{lst:transkriprequest}.
%Kode
\begin{lstlisting}[language=diff, caption=Perubahan file \www\application\controllers\TranskripRequest.php,  basicstyle=\ttfamily, frame=single,
columns=fullflexible, keepspaces=true, breaklines=true, label={lst:transkriprequest}]
--- a\www\application\controllers\TranskripRequest.php	
+++ b\www\application\controllers\TranskripRequest.php	

@@ -29,13 +29,13 @@
$request->labelClass = 'secondary';
} else if ($request->answer === 'printed') {
$request->status = 'TERCETAK';
$request->labelClass = 'success';
} else if ($request->answer === 'rejected') {
$request->status = 'DITOLAK';
-                $request->labelClass = 'alert';
+                $request->labelClass = 'danger';
}
\end{lstlisting}

\subsection{Halaman Manajemen Cetak Transkrip} 
Pada kode ~\ref{lst:mainTranskripManage} menjelaskan perubahan kode pada halaman manajemen permintaan transkrip.

\subsubsection{Halaman Utama}
\begin{lstlisting}[language=diff, caption=Perubahan file \path{\views\TranskripManage\main.php},  basicstyle=\ttfamily, frame=single,
columns=fullflexible, keepspaces=true, breaklines=true, label={lst:mainTranskripManage}]
--- a\www\application\views\TranskripManage\main.php
+++ b\www\application\views\TranskripManage\main.php

@@ -4,47 +4,63 @@
#Implementasi container dan komponen card.
- <div class="row">
- 	<div class="callout">
- 		<h5>Permintaan Transkrip</h5>
+ <div class="container">
+ 	<div class="card">
+ 		<div class="card-header">
+ 		Permintaan Transkrip
+ 		</div>
+ 	<div class="card-body">

#Implementasi komponen input-group
- <span class="input-group-label">Cari NPM:</span>
+ <div class="input-group-prepend">
+ 	<span class="input-group-text">Cari NPM:</span>
+ </div>
- 	<input name="npm" class="input-group-field" type="text" placeholder="2013730013" maxlength="10" minlength="10"<?= $npmQuery === NULL ? '' : " value='$npmQuery'" ?>/>
- <div class="input-group-button">
- 	<input class="button" type="submit" value="Cari"/>
+ 	<input name="npm" class="form-control" type="text" placeholder="2013730013" maxlength="10" minlength="10"<?= $npmQuery === NULL ? '' : " value='$npmQuery'" ?>/>
+ <div class="input-group-append">
+ 	<input class="btn btn-primary" type="submit" value="Cari"/>

#Implementasi tabel
- <table class="stack">
+ <br>
+ <table class="table table-striped">

- <th>ID</th>
- <th>Status</th>
- <th>Tanggal Permohonan</th>
- <th>Tipe Transkrip</th>
- <th>NPM</th>
- <th>Aksi</th>
+ <th scope="col">ID</th>
+ <th scope="col">Status</th>
+ <th scope="col">Tanggal Permohonan</th>
+ <th scope="col">Tipe Transkrip</th>
+ <th scope="col">NPM</th>
+ <th scope="col">Aksi</th>

# Implementasi komponen label pada kolom status
- <td><span class="<?= $request->labelClass ?> label"><?= $request->status ?></span></td>
+ <td><span class="badge badge-<?= $request->labelClass ?>"><?= $request->status ?></span></td>

#Implementasi modal berisi tabel detail permohonan
- <div class="reveal" id="detail<?= $request->id ?>" data-reveal>
- <h5>Detail Permohonan #<?= $request->id ?></h5>
- <table class="stack">
+ <div class="modal fade" id="detail<?= $request->id ?>" tabindex="-1" role="dialog" aria-hidden="true">
+ 	<div class="modal-dialog modal-dialog-centered" role="document">
+ 		<div class="modal-content">
+ 			<div class="modal-header">
+ 				<h5 class="modal-title" id="exampleModalLongTitle">Detail Permohonan #<?= $request->id ?></h5>
+ 				<button type="button" class="close" data-dismiss="modal" aria-label="Close">
+ 				<span aria-hidden="true">&times;</span>
+ 				</button>
+ 			</div>
+ 			<div class="modal-body">
+ 				<table class="table table-striped">

@@ -78,31 +94,61 @@
#Implementasi ikon link menuju modal lihat
+ <a data-toggle="modal" data-target="#detail<?= $request->id ?>" id="detailIkon<?= $request->id ?>">
+ <i class="fas fa-eye"></i>
+ </a>

#Implementasi modal
- <button class="close-button" data-close aria-label="Tutup" type="button">
+ <div class="modal fade" id="tolak<?= $request->id ?>" tabindex="-1" role="dialog" aria-hidden="true">
+ 	<div class="modal-dialog modal-dialog-centered" role="document">
+ 		<div class="modal-content">
+ 			<div class="modal-header">
+ 				<h5 class="modal-title" id="exampleModalLongTitle">Tolak Permohonan #<?= $request->id ?></h5>
+ 				<button type="button" class="close" data-dismiss="modal" aria-label="Close">

#Implementasi form tolak permintaan
- <label>Email penjawab:
- 	<input type="text" value="<?= $answeredByEmail ?>" readonly="true"/>
- </label>
- <label>Alasan penolakan:
- 	<input name="answeredMessage" class="input-group-field" type="text" required/>
- </label>
+ <div class="form-group">
+ 	<label>Email penjawab:</label>
+ 	<input class="form-control" type="text" value="<?= $answeredByEmail ?>" readonly="true"/>
+ </div>
+ <div class="form-group">
+ 	<label>Alasan penolakan:</label>
+ 	<input class="form-control" name="answeredMessage" type="text" required/>
+ </div>                                          
- <input type="submit" class="alert button" value="Tolak"/>
+ <div class="form-group">
+ 	<input type="submit" class="btn btn-danger" value="Tolak"/>

#Implementasi ikon link modal tolak.
+ <a data-toggle="modal" data-target="#tolak<?= $request->id ?>" id="detailIkon<?= $request->id ?>">
+ 	<i class="fas fa-thumbs-down"></i>
+ </a>

#Implementasi modal cetak
- <button class="close-button" data-close aria-label="Tutup" type="button">
+ <div class="modal fade" id="cetak<?= $request->id ?>" tabindex="-1" role="dialog" aria-hidden="true">
+ 	<div class="modal-dialog modal-dialog-centered" role="document">
+ 		<div class="modal-content">
+ 			<div class="modal-header">
+ 				<h5 class="modal-title" id="exampleModalLongTitle">Cetak Permohonan #<?= $request->id ?></h5>
+ 				<button type="button" class="close" data-dismiss="modal" 

@@ -111,39 +157,58 @@
#Implementasi form cetak transkrip
- <label>Email penjawab:
- 	<input type="text" value="<?= $answeredByEmail ?>" readonly="true"/>
- </label>
- <label>Keterangan tambahan:
- 	<input name="answeredMessage" class="input-group-field" type="text" required/>
- </label>
+ <div class="form-group">
+ 	<label class="col-form-label">Email penjawab:</label>
+ 	<input class="form-control" type="text" value="<?= $answeredByEmail ?>" readonly="true"/>
+ </div>
+ <div class="form-group">
+ 	<label class="col-form-label">Keterangan tambahan:</label>
+ 	<input class="form-control" name="answeredMessage" type="text" required/>
+ </div>                                            
- <input type="submit" class="button" value="Sudah dicetak"/>
+ <div class="form-group">
+ 	<input class="btn btn-primary" type="submit" class="button" value="Sudah dicetak"/>

#Implementasi ikon link cetak transkrip
+ <a data-toggle="modal" data-target="#cetak<?= $request->id ?>" id="detailIkon<?= $request->id ?>">
+  	<i class="fas fa-print"></i>
+  </a>

#Implementasi modal hapus permohonan transkrip
- <button class="close-button" data-close aria-label="Tutup" type="button">
+ <div class="modal fade" id="hapus<?= $request->id ?>" tabindex="-1" role="dialog" aria-hidden="true">
+ 	<div class="modal-dialog modal-dialog-centered" role="document">
+ 		<div class="modal-content">
+ 			<div class="modal-header">
+ 				<h5 class="modal-title" id="exampleModalLongTitle">Hapus Permohonan #<?= $request->id ?></h5>
+ 				<button type="button" class="close" data-dismiss="modal" aria-label="Close">
- <input type="submit" class="alert button" value="Hapus"/>
+ <input class="btn btn-danger" type="submit" value="Hapus"/>

# Implementasi ikon link menuju modal hapus
- <a data-open="hapus<?= $request->id ?>"><i class="fi-trash"></i></a>
+ <a data-toggle="modal" data-target="#hapus<?= $request->id ?>" id="detailIkon<?= $request->id ?>">
+ 	<i class="fas fa-trash"></i>
+ </a>
\end{lstlisting}
\noindent Catatan untuk kode ~\ref{lst:topbarloggedin} dijabarkan dalam tabel ~\ref{tabel:KodeManajemenCetakTranskrip}:
\begin{table}[H]
	\centering
	\begin{tabularx}{\textwidth}{lX}
		\toprule
		Implementasi     & Penjelasan \\
		\midrule
		\textbf{Border untuk konten} & Dalam Bootstrap 4 pembagian konten menggunakan \texttt{.card} yang terdiri dari \texttt{.card-header} dan \texttt{card-body} sedangkan di Foundation 6 hanya menggunakan kelas \texttt{.callout}. Lalu untuk form, field pada Foundation 6 hanya menggunakan tag \texttt{<input>} sedangkan pada Bootstrap 4 menggunakan kelas \texttt{.form-control}.\\
		\textbf{Modal}  & Lalu terdapat \textit{input group field}, dimana pada Bootstrap 4 setiap tag \texttt{span} diikuti kelas \texttt{.input-group-prepend}\\
		\bottomrule
	\end{tabularx}%
	\caption{Penjelasan kode konversi manajemen cetak transkrip.}
	\label{tabel:KodeManajemenCetakTranskrip}
\end{table}

\subsubsection{Controller Manajemen Cetak Transkrip}
%Kode
Pemberian label pada kolom 'status' menggunakan logika sederhana yang terletak pada controller, hasil dilampirkan dalam kode ~\ref{lst:transkripManage}.
\begin{lstlisting}[language=diff, caption=Controller Manajemen Cetak Transkrip,  basicstyle=\ttfamily, frame=single,
columns=fullflexible, keepspaces=true, breaklines=true, label={lst:transkripManage}]
--- a\www\application\controllers\TranskripManage.php
+++ b\www\application\controllers\TranskripManage.php
@@ -43,13 +43,13 @@
$request->labelClass = 'warning';
} else if ($request->answer === 'printed') {
$request->status = 'TERCETAK';
$request->labelClass = 'success';
} else if ($request->answer === 'rejected') {
$request->status = 'DITOLAK';
-                $request->labelClass = 'alert';
+                $request->labelClass = 'danger';
}
\end{lstlisting}

\subsection{Halaman Permintaan Perubahan Kuliah}
Dalam Bootstrap 4 pembagian konten menggunakan \texttt{.card} yang terdiri dari \texttt{.card-header} dan \texttt{card-body} sedangkan di Foundation 6 hanya menggunakan kelas \texttt{.callout}. 
Lalu untuk form, field pada Foundation 6 hanya menggunakan tag \texttt{<input>} sedangkan pada Bootstrap 4 menggunakan kelas \texttt{.form-control}.
Lalu terdapat \textit{input group field}, dimana pada Bootstrap 4 setiap tag \texttt{span} diikuti kelas \texttt{.input-group-prepend}.
Untuk modal pada Foundation 6 hanya memiliki satu bagian modal yaitu kelas \texttt{.reveal} sedangkan pada Bootstrap 4 menggunakan \texttt{.modal} yang terdiri dari \texttt{.modal-fade, modal-content, modal-header}.
Tabel pada Foundation 6 pada tag \texttt{<th>} tidak diikuti kelas, namun pada Bootstrap 4 perlu menggunakan atribut \texttt{scope=col}.
\begin{lstlisting}[language=diff, caption=Perubahan file \path{\views\PerubahanKuliahRequest\main.php} ,  basicstyle=\ttfamily, frame=single,
columns=fullflexible, keepspaces=true, breaklines=true, label={lst:mainPerubahanKuliahRequest}]
--- a\www\application\views\PerubahanKuliahRequest\main.php
+++ b\www\application\views\PerubahanKuliahRequest\main.php

@@ -5,136 +5,147 @@
#    Implementasi Komponen Card
- <div class="row">
-    <div class="large-12 column">
-        <div class="callout">
-            <h5>Permohonan Baru</h5>
-            <form method="POST" action="/PerubahanKuliahRequest/add">
+        <div class="container">
+            <div class="card">
+                <div class="card-header">
+                    Permohonan Baru
+                </div>
+                <div class="card-body">
+                    <form class="p-3" method="POST" action="/PerubahanKuliahRequest/add">

#    Implementasi Komponen Form
- <div class="row">
-    <div class="large-4 column">
-        <label>Pemohon:
-            <input type="email" name="requestByEmail" value="<?= $requestByEmail ?>" readonly="readonly"/>
-        </label>
+ <div class="form-group row">
+    <div class="col-sm-6">
+        <label class="col-form-label">Pemohon:</label>
+        <input class="form-control" type="email" name="requestByEmail" value="<?= $requestByEmail ?>" 

#    Implementasi Komponen Form
- <div class="large-8 column">
- <label>Nama:
-    <input type="text" name="requestByName" value="<?= $requestByName ?>" readonly="readonly"/>
- </label>
+ <div class="col-sm-6">
+    <label class="col-form-label">Nama:</label>
+    <input class="form-control" type="text" name="requestByName" value="<?= $requestByName ?>" readonly="readonly"/>

#    Implementasi Komponen Grid dan Form
- <div class="row">
-    <div class="large-2 column">
-        <label>Kode MK:
-            <input type="text" name="mataKuliahCode" required maxlength="9" pattern="[A-Z]{3}[0-9]{3}([0-9]{3})?" title="Kode MK dalam format XYZ123"/>
-        </label>
+  <div class="form-group row">
+     <div class="col-sm-2">
+        <label class="col-form-label">Kode MK:</label>
+        <input class="form-control" type="text" name="mataKuliahCode" required maxlength="9" pattern="[A-Z]{3}[0-9]{3}([0-9]{3})?" title="Kode MK dalam format XYZ123"/>

- <div class="large-5 column">
- <label>Nama Mata Kuliah:
-    <input type="text" name="mataKuliahName" required/>
- </label>
+ <div class="col-sm-5">
+    <label class="col-form-label">Nama Mata Kuliah:</label>
+    <input class="form-control" type="text" name="mataKuliahName" required/>

- <div class="large-1 column">
- <label>Kelas:
-    <input type="text" name="class" maxlength="1"/>
- </label>
+ <div class="col-sm-1">
+    <label class="col-form-label">Kelas:</label>
+    <input class="form-control" type="text" name="class" maxlength="1"/>

- <div class="large-4 column">
-   <label>Jenis Perubahan:
-    <select name="changeType">
+ <div class="col-sm-4">
+   <label class="col-form-label">Jenis Perubahan:</label>
+    <select name="changeType" class="form-control">

- <div class="row">
-    <div class="large-3 column">
-        <label>Dari Hari &amp; Jam:
-            <input class="disableable" type="text" name="fromDateTime" id="fromDateTime"/>
-        </label>
+ <div class="form-group row">
+    <div class="col-sm-3">
+        <label class="col-form-label">Dari Hari &amp; Jam:</label>
+        <input id="datetimepicker" class="form-control disableable" type="text" name="fromDateTime">

-    <div class="large-3 column">
-        <label>Dari Ruang:
-            <input class="disableable" type="text" name="fromRoom"/>
-        </label>
+   <div class="col-sm-3">
+        <label class="col-form-label">Dari Ruang:</label>
+        <input class="form-control disableable" type="text" name="fromRoom"/>

-    <div class="large-6 column">
-        <label>Keterangan Tambahan:
-            <input class="disableable" type="text" name="remarks"/>
-        </label>
+        <div class="col-sm-6">
+            <label class="col-form-label">Keterangan Tambahan:</label>
+            <input class="form-control disableable" type="text" name="remarks"/>

- <div class="row toFields">
-    <div class="large-3 column">
-        <label>Menjadi Hari &amp; Jam:
-            <input class="disableable toDateTime" type="text" name="toDateTime[]"/>
-        </label>
+    <div class="form-group row toFields">
+      <div class="col-sm-3">
+            <label class="col-form-label">Menjadi Hari &amp; Jam:</label>
+            <input id="datetimepicker" class="form-control disableable toDateTime" type="text" name="toDateTime[]"/>

-    <div class="large-3 column">
-        <label>Menjadi Ruang:
-            <input class="disableable toRoom" type="text" name="toRoom[]"/>
-        </label>
+    <div class="col-sm-3">
+        <label class="col-form-label">Menjadi Ruang:</label>
+        <input class="form-control disableable toRoom" type="text" name="toRoom[]"/>

-    <div class="large-6 column">
-        <br/>
-        <a href="#" class="eraseButton button secondary">Hapus</a>
+    <div class="col-sm-2">
+        <br/><br>
+        <a href="#" class="eraseButton btn btn-secondary">Hapus</a>

- <div class="row" id="sendDiv">
-    <div class="large-12 column">
-        <input type="submit" class="button" value="Kirim Permohonan">
-        <a href="#" id="addToButton" class="button secondary">Tambah Pertemuan Ekstra</a>
+    <div class="form-group row" id="sendDiv">
+      <div class="col-sm-12">
+        <input type="submit" class="btn btn-primary" value="Kirim Permohonan">
+        <a href="#" id="addToButton" class="btn btn-secondary">Tambah Pertemuan Ekstra</a>

#    Implementasi Komponen Card dan Tabel
- <div class="callout">
- <h5>Histori Permohonan</h5>
- <table class="stack">
+ </div>
+  <br>
+ <div class="card">
+    <div class="card-header">
+      Histori Permohonan
+ </div>
+ <div class="card-body">
+    <table class="table table-striped table-responsive">

#    Implementasi Kelas pada Elemen <th>
-        <th>ID</th>
-        <th>Status</th>
-        <th>Tanggal Permohonan</th>
-        <th>Kode MK</th>
-        <th>Perubahan</th>
-        <th>Tanggal Jawab</th>
-        <th>Keterangan</th>
-        <th>Aksi</th>
+        <th scope="col">ID</th>
+        <th scope="col">Status</th>
+        <th scope="col">Tanggal Permohonan</th>
+        <th scope="col">Kode MK</th>
+        <th scope="col">Perubahan</th>
+        <th scope="col">Tanggal Jawab</th>
+        <th scope="col">Keterangan</th>
+        <th scope="col">Aksi</th>

#    Implementasi Badges
-  <td><span class="<?= $request->labelClass ?> label"><?= $request->status ?></span></td>
+  <td><span class=" badge badge-<?= $request->labelClass ?>"><?= $request->status ?></span></td>

#    Implementasi Ikon Font Awesome
-  <a data-open="detail<?= $request->id ?>"><i class="fi-eye"></i></a>
+  <a data-toggle="modal" data-target="#detail<?= $request->id ?>" id="detailIkon<?= $request->id ?>">
+     <span style="font-size: 18px; color: Dodgerblue;">
+        <i class="fas fa-eye"></i>
+     </span>
+     </a>

#    Implementasi Komponen Modal
-    <div class="reveal" id="detail<?= $request->id ?>" data-reveal>
-        <h5>Detail Permohonan #<?= $request->id ?></h5>
-        <table class="stack">
+    <div class="modal fade" id="detail<?= $request->id ?>" tabindex="-1" role="dialog" aria-hidden="true">
+     <div class="modal-dialog modal-dialog-centered" role="document">
+        <div class="modal-content">
+           <div class="modal-header">
+              <h5 class="modal-title" id="exampleModalLongTitle">Detail Permohonan #<?= $request->id ?></h5>
+              <button type="button" class="close" data-dismiss="modal" aria-label="Close">
+                 <span aria-hidden="true">&times;</span>
+              </button>
+           </div>
+           <div class="modal-body">
+             <table class="table table-striped">

#Penggunaan jQuery
@@ -208,12 +219,13 @@
+jQuery('#datetimepicker').datetimepicker();
\end{lstlisting}

\subsubsection{Controller Permintaan Perubahan Kuliah}
%Kode
Pemberian label pada kolom 'status' menggunakan logika sederhana yang terletak pada controller, hasil dilampirkan dalam kode ~\ref{lst:perubahanKuliahRequest}.
\begin{lstlisting}[language=diff, caption=Controller Request Perubahan Kuliah,  basicstyle=\ttfamily, frame=single,
columns=fullflexible, keepspaces=true, breaklines=true, label={lst:perubahanKuliahRequest}]
--- a\www\application\controllers\PerubahanKuliahRequest.php
+++ b\www\application\controllers\PerubahanKuliahRequest.php

@@ -28,13 +28,13 @@
$request->labelClass = 'secondary';
} else if ($request->answer === 'confirmed') {
$request->status = 'TERKONFIRMASI';
$request->labelClass = 'success';
} else if ($request->answer === 'rejected') {
$request->status = 'DITOLAK';
-                $request->labelClass = 'alert';
+                $request->labelClass = 'danger';
}
\end{lstlisting}

\subsection{Halaman Manajemen Perubahan Kuliah}
Dalam Bootstrap 4 pembagian konten menggunakan \texttt{.card} yang terdiri dari \texttt{.card-header} dan \texttt{card-body} sedangkan di Foundation 6 hanya menggunakan kelas \texttt{.callout}. 
Lalu untuk form, field pada Foundation 6 hanya menggunakan tag \texttt{<input>} sedangkan pada Bootstrap 4 menggunakan kelas \texttt{.form-control}.
Lalu terdapat \textit{input group field}, dimana pada Bootstrap 4 setiap tag \texttt{span} diikuti kelas \texttt{.input-group-prepend}.
Untuk modal pada Foundation 6 hanya memiliki satu bagian modal yaitu kelas \texttt{.reveal} sedangkan pada Bootstrap 4 menggunakan \texttt{.modal} yang terdiri dari \texttt{.modal-fade, modal-content, modal-header}.
Tabel pada Foundation 6 pada tag \texttt{<th>} tidak diikuti kelas, namun pada Bootstrap 4 perlu menggunakan atribut \texttt{scope=col}.

\begin{lstlisting}[language=diff, caption=Perubahan file \path{\views\PerubahanKuliahManage\main.php},  basicstyle=\ttfamily, frame=single,
columns=fullflexible, keepspaces=true, breaklines=true, label={lst:mainPerubahanKuliahManage}]
--- a\www\application\views\PerubahanKuliahManage\main.php
+++ b\www\application\views\PerubahanKuliahManage\main.php

@@ -4,40 +4,44 @@
#    Implementasi Komponen Card
- <div class="row">
-    <div class="callout">
-        <h5>Permohonan Perubahan Kuliah</h5>
-        <table class="stack">
+        <div class="container">
+            <div class="card">
+                <div class="card-header">
+                    Permohonan Perubahan Kuliah
+                </div>
+                <br>
+                <div class="card-body">
+                    <table class="table table-striped">

#    Implementasi Kelas Column pada elemen <th>
- <th>ID</th>
- <th>Status</th>
- <th>Tanggal Permohonan</th>
- <th>Kode MK</th>
- <th>Perubahan</th>
- <th>Aksi</th>
+ <th scope="col">ID</th>
+ <th scope="col">Status</th>
+ <th scope="col">Tanggal Permohonan</th>
+ <th scope="col">Kode MK</th>
+ <th scope="col">Perubahan</th>
+ <th scope="col">Aksi</th>

#    Implementasi Komponen Badge
- <td><span class="<?= $request->labelClass ?> label"><?= $request->status ?></span></td>
+ <td><span class="badge badge-<?= $request->labelClass ?>"><?= 

#    Implementasi Komponen Form
-   <label>Email penjawab:
-    <input type="text" value="<?= $answeredByEmail ?>" readonly="true"/>
-   </label>
-   <label>Keterangan:
-    <input name="answeredMessage" class="input-group-field" type="text"/>
-   </label>
+ <div class="form-group">
+   <label>Email penjawab:</label>
+   <input class="form-control" type="text" value="<?= $answeredByEmail ?>" readonly="true"/>
+ </div>
+ <div class="form-group">
+   <label>Keterangan:</label>
+   <input class="form-control" name="answeredMessage" class="input-group-field" type="text"/>
+ </div>

#    Implementasi Komponen Button                                             
-    	<input type="submit" class="success button" value="Konfirmasi"/>
+    <div class="form-group">
+       <input type="submit" class="btn btn-success" value="Konfirmasi"/>
+    </div>

#    Implementasi Komponen Modal dan Library Ikon Awesome
- <a data-open="detail<?= $request->id ?>"><i class="fi-eye"></i></a>
- <a target="_blank" href="/PerubahanKuliahManage/printview/<?= $request->id ?>"><i class="fi-print"></i></a>
- <a data-open="konfirmasi<?= $request->id ?>"><i class="fi-like"></i></a>                                    
- <a data-open="tolak<?= $request->id ?>"><i class="fi-dislike"></i></a>
- <a data-open="hapus<?= $request->id ?>"><i class="fi-trash"></i></a>
+ <a data-toggle="modal" data-target="#detail<?= $request->id ?>" id="detailIkon<?= $request->id ?>"><i class="fas fa-eye blueiconcolor"></i></a>
+ <a target="_blank" href="/PerubahanKuliahManage/printview/<?= $request->id ?>"><i class="fas fa-print"></i></a>
+ <a data-toggle="modal" data-target="#konfirmasi<?= $request->id ?>"><i class="fas fa-thumbs-up"></i></a>
+ <a data-toggle="modal" data-target="#tolak<?= $request->id ?>"><i class="fas fa-thumbs-down"></i></a>
+ <a data-toggle="modal" data-target="#hapus<?= $request->id ?>"><i 

#    Implementasi Komponen  Modal
@@ -49,19 +53,24 @@
- <div class="reveal" id="detail<?= $request->id ?>" data-reveal>
- <h5>Detail Permohonan #<?= $request->id ?></h5>
- <table class="stack">
+ <div class="modal fade" id="detail<?= $request->id ?>" tabindex="-1" role="dialog" aria-hidden="true">
+    <div class="modal-dialog modal-dialog-centered" role="document">
+       <div class="modal-content">
+          <div class="modal-header">
+             <h5 class="modal-title" id="exampleModalLongTitle">Detail Permohonan #<?= $request->id ?></h5>
+             <button type="button" class="close" data-dismiss="modal" aria-label="Close">
+               <span aria-hidden="true">&times;</span>
+             </button>
+           </div>
+           <div class="modal-body">
+               <table class="table table-striped">

#    Implementasi Komponen Modal
@@ -121,68 +130,101 @@
-               <button class="close-button" data-close aria-label="Tutup" type="button">
+       </div>
+     </div>
+   </div>
+ </div>
+  <div class="modal fade" id="konfirmasi<?= $request->id ?>" tabindex="-1" role="dialog" aria-hidden="true">
+    <div class="modal-dialog modal-dialog-centered" role="document">
+        <div class="modal-content">
+            <div class="modal-header">
+               <h5 class="modal-title" id="exampleModalLongTitle">Konfirmasi Permohonan #<?= $request->id ?></h5>
+                <button type="button" class="close" data-dismiss="modal" aria-label="Close">
#    Implementasi Komponen Modal
-        <button class="close-button" data-close aria-label="Tutup" type="button">
+       </div>
+     </div>
+   </div>
+ </div>
+ <div class="modal fade" id="tolak<?= $request->id ?>" tabindex="-1" role="dialog" aria-hidden="true">
+   <div class="modal-dialog modal-dialog-centered" role="document">
+      <div class="modal-content">
+        <div class="modal-header">
+           <h5 class="modal-title" id="exampleModalLongTitle">Tolak Permohonan #<?= $request->id ?></h5>
+             <button type="button" class="close" data-dismiss="modal" 

#    Implementasi Komponen Form
-    <label>Email penjawab:
-        <input type="text" value="<?= $answeredByEmail ?>" readonly="true"/>
-    </label>
-    <label>Alasan penolakan:
-        <input name="answeredMessage" class="input-group-field" type="text" required/>
-    </label>
+ <div class="form-group">
+    <label>Email penjawab:</label>
+    <input class="form-control" type="text" value="<?= $answeredByEmail ?>" readonly="true"/>
+ </div>
+ <div class="form-group">
+    <label>Alasan penolakan:</label>
+    <input class="form-control" name="answeredMessage" class="input-group-field" type="text" required/>
+ </div>

#    Implementasi Komponen Button                                          
-     <input type="submit" class="alert button" value="Tolak"/>
+ <div class="form-group">
+     <input type="submit" class="btn btn-danger" value="Tolak"/>
+ </div>

#    Implementasi Komponen Modal
-    	    <button class="close-button" data-close aria-label="Tutup" type="button">
+ <div class="modal fade" id="hapus<?= $request->id ?>" tabindex="-1" role="dialog" aria-hidden="true">
+  <div class="modal-dialog modal-dialog-centered" role="document">
+    <div class="modal-content">
+       <div class="modal-header">
+          <h5 class="modal-title" id="exampleModalLongTitle">Hapus Permohonan #<?= $request->id ?></h5>
+           <button type="button" class="close" data-dismiss="modal" 
#    Implementasi Modal
- <div class="reveal" id="hapus<?= $request->id ?>" data-reveal>
-	<h5>Hapus Permohonan</h5>
+ <div class="modal-body">


#    Implementasi Komponen Button
- <input type="submit" class="alert button" value="Hapus"/>
+ <input type="submit" class="btn btn-danger" value="Hapus"/>
\end{lstlisting}

\subsubsection{Controller Manajemen Perubahan Kuliah}
%Kode
Pemberian label pada kolom 'status' menggunakan logika sederhana yang terletak pada controller, hasil dilampirkan dalam kode ~\ref{lst:perubahanKuliahManage}.
\begin{lstlisting}[language=diff, caption=Controller PerubahanKuliahManage,  basicstyle=\ttfamily, frame=single,
columns=fullflexible, keepspaces=true, breaklines=true, label={lst:perubahanKuliahManage}]
--- a\www\application\controllers\PerubahanKuliahManage.php	
+++ b\www\application\controllers\PerubahanKuliahManage.php	

@@ -38,13 +38,13 @@
$request->labelClass = 'warning';
} else if ($request->answer === 'confirmed') {
$request->status = 'TERKONFIRMASI';
$request->labelClass = 'success';
} else if ($request->answer === 'rejected') {
$request->status = 'DITOLAK';
-                $request->labelClass = 'alert';
+                $request->labelClass = 'danger';
}
\end{lstlisting}  

\subsection{Halaman Entri Jadwal Dosen}
Dalam Bootstrap 4 pembagian konten menggunakan \texttt{.card} yang terdiri dari \texttt{.card-header} dan \texttt{card-body} sedangkan di Foundation 6 hanya menggunakan kelas \texttt{.callout}. 
Lalu untuk form, field pada Foundation 6 hanya menggunakan tag \texttt{<input>} sedangkan pada Bootstrap 4 menggunakan kelas \texttt{.form-control}.
Lalu terdapat \textit{input group field}, dimana pada Bootstrap 4 setiap tag \texttt{span} diikuti kelas \texttt{.input-group-prepend}.
Untuk modal pada Foundation 6 hanya memiliki satu bagian modal yaitu kelas \texttt{.reveal} sedangkan pada Bootstrap 4 menggunakan \texttt{.modal} yang terdiri dari \texttt{.modal-fade, modal-content, modal-header}.
Tabel pada Foundation 6 pada tag \texttt{<th>} tidak diikuti kelas, namun pada Bootstrap 4 perlu menggunakan atribut \texttt{scope=col}.
\begin{lstlisting}[language=diff, caption=Kode untuk Halaman Entri Jadwal Dosen,  basicstyle=\ttfamily, frame=single,
columns=fullflexible, keepspaces=true, breaklines=true, label={lst:mainEntriJadwalDosen}]
--- a\www\application\views\EntriJadwalDosen\main.php
+++ b\www\application\views\EntriJadwalDosen\main.php

@@ -6,78 +6,86 @@
#    Implementasi Komponen Grid
+ <div class="container">
+    <div class="card">
+        <div class="card-header">
+            Tambah Jadwal
+        </div>
+        <div class="card-body">

- <div class="large-12 column callout">
-  <h5>Tambah Jadwal</h5>
-    <div class="large-4 columns">
+ <div class="col-lg-4">

#    Implementasi Komponen Form
- <select name="hari"> 
+ <select class="form-control" name="hari">

- <select name="jam_mulai"> 
+ <select class="form-control" name="jam_mulai">

- <div class=" large-4 columns">
+ <div class="col-lg-4">

- <select name="durasi"> 
+ <select class="form-control" name="durasi">

- <select name="jenis_jadwal"> 
+ <select class="form-control" name="jenis_jadwal">

- <div class="large-4 columns">
-    Label <input type="text" name="label_jadwal"><br>
-    <input type="submit" class="button" value="Tambah">
-    </form>
+ <div class="col-lg-4">
+    Label <input class="form-control" type="text" name="label_jadwal"><br><br>
+    <input class="btn btn-primary" type="submit" value="Tambah">
+    </form><br>

#    Implemetasi Komponen Card dan Tabel
-
- <div class="large-12 column callout">
-    <h5>Daftar Jadwal</h5>
- <div class="table-scroll" id="jadwal_table">
-	<table border=1 style="border-color:black ; border-collapse:separate">
+ <br>
+ <div class="card">
+ 	<div class="card-header">
+    	Daftar Jadwal
+ 	</div>
+ 	<div class="card-body">
+   	<div id="jadwal_table">
+        	<table class="table table-bordered table-striped">

#    Perubahan Lebar Cell pada Tabel
- <td style='width:10%'></td>
+ <th></th>

#    Perubahan Style di jQuery
for ($i = 0; $i < 5; $i++) {
- echo "<td style='width:18%'> $namaHari[$i] </td>"; //Membuat Header Tabel yang berisi daftar hari
+ echo "<th> $namaHari[$i] </th>"; //Membuat Header Tabel yang berisi daftar hari

#    Pemberian Border di jQuery
foreach ($dataJadwal as $dataHariIni) {
$colIdx = $dataHariIni->hari + 1;   // + 1 karena perbedaan selisih index tabel dan value hari di database 
$rowIdx = $dataHariIni->jam_mulai - 6;  // + 1 karena perbedaan selisih index tabel dan value jam_mulai di database 
@@ -86,12 +94,13 @@
+ $border = "border border-secondary align-middle";

@@ -101,23 +110,24 @@
#    Implementasi jQuery untuk Tabel
$($cellLocation).css('background-color', '<?php echo $color; ?>');
$($cellLocation).attr('rowspan', <?php echo $dataHariIni->durasi ?>);
+ $($cellLocation).addClass('<?php echo $border; ?>');

#    Implementasi jQuery untuk Modal
- $($menuName).foundation('open');
+ $($menuName).modal();

@@ -143,41 +153,47 @@
#    Implementasi Komponen Form
- <a href="/EntriJadwalDosen/deleteAll/export/" class="alert button" onClick="return konfirmasi();">Delete All</a>
- <a href="/EntriJadwalDosen/export/" class="button">Ekspor ke XLS</a>
+ <a href="/EntriJadwalDosen/deleteAll/export/" class="btn btn-danger" onClick="return konfirmasi();">Delete All</a>
+ <a href="/EntriJadwalDosen/export/" class="btn btn-primary">Ekspor ke XLS</a>

#    Implementasi Komponen Grid
- <div class="large-2 column ">
+ <div class="col-lg-2">


#    Implementasi Komponen Button@@
- <input type="submit" id="deletebtn<?php echo $dataHariIni->id ?>" name="deletebtn<?php echo $dataHariIni->id ?>" class="alert button" value="Delete">
- </form><div>
+ <input class="btn btn-danger" type="submit" id="deletebtn<?php echo $dataHariIni->id ?>" name="deletebtn<?php echo $dataHariIni->id ?>" value="Delete">
+ </form>

#    Implementasi Komponen Modal
- <div id="edit_menu<?php echo $dataHariIni->id ?>" class="reveal"  data-reveal >
- 		<button class="close-button" data-close aria-label="Close modal" type="button">
+ <div class="modal fade" id="edit_menu<?php echo $dataHariIni->id ?>" tabindex="-1" role="dialog" aria-hidden="true">
+  <div class="modal-dialog modal-dialog-centered" role="document">
+   <div class="modal-content">
+    <div class="modal-header">
+     <h5 class="modal-title" id="exampleModalLongTitle">Edit Jadwal #<?= $dataHariIni->id ?></h5>
+ 		<button type="button" class="close" data-dismiss="modal" aria-label="Close">

@@ -189,13 +205,13 @@
#    Implementasi Komponen Form
- <select name="hari"> 
+ <select class="form-control" name="hari">

- <select name="jam_mulai"> 
+ <select class="form-control" name="jam_mulai">

@@ -205,13 +221,13 @@
- <select name="durasi"> 
+ <select class="form-control" name="durasi">

@@ -221,35 +237,39 @@
- <select name="jenis_jadwal"> 
+ <select class="form-control" name="jenis_jadwal">

- </select>
- Label <input type="text" name="label_jadwal" value="<?php echo $dataHariIni->label; ?>"><br> 
-   <div class="row large-4 column">
-    <div class="large-2 column">
-        <input type="submit" name="submitId<?php echo $dataHariIni->id ?>" class="button" value="Save  ">
-        </form>
+ </select><br>
+ Label
+ <input class="form-control" type="text" name="label_jadwal" value="<?php echo $dataHariIni->label; ?>"><br>
+   <div class="row">
+    <div class="col-lg-2">
+       <input class="btn btn-primary" type="submit" name="submitId<?php echo $dataHariIni->id ?>" class="button" value="Save  ">
\end{lstlisting}


\subsection{Halaman	Lihat Jadwal Dosen}
\begin{lstlisting}[language=diff, caption=Kode untuk Halaman Lihat Jadwal Dosen,  basicstyle=\ttfamily, frame=single,
columns=fullflexible, keepspaces=true, breaklines=true, label={lst:mainEntriJadwalDosen}]
--- a\www\application\views\LihatJadwalDosen\main.php
+++ b\www\application\views\LihatJadwalDosen\main.php

@@ -6,30 +6,30 @@
#    Implementasi Komponen Grid dan Card
-
+ <div class="container">
+    <div class="card card-body">

#    Implementasi Komponen Tabs
- <div class="large-12 column">
- <ul class="tabs" data-tabs id="tab_jadwal">
+ <div class="col-lg-12">
+	<ul class="nav nav-tabs" data-tabs id="tab_jadwal">

- <li class="tabs-title is-active"><a href="#hal<?php echo $idx; ?>" aria-selected="true"><?php foreach ($currRow as $data) {
+ <li class="nav-item"><a class="nav-link active" href="#hal<?php echo $idx; ?>" 

- <li class="tabs-title"><a a href="#hal<?php echo $idx; ?>"><?php foreach ($currRow as $data) {
+ <li class="nav-item"><a class="nav-link" href="#hal<?php echo $idx; ?>"><?php foreach ($currRow as $data) {

@@ -41,39 +41,39 @@
#    Implementasi Styling Tabel Bergaris
- <div class="table-scroll" id="jadwal_table<?php echo $idx; ?>">
+ <div id="jadwal_table<?php echo $idx; ?>">

- <table id="tabel<?php echo $idx; ?>" border=1 style="border-color:black ; border-collapse:separate">
+ <table class="table table-bordered table-striped" id="tabel<?php echo $idx; ?>" >

#    Perubahan Kode di jQuery
- echo "<td style='width:18%'>" . $namaHari[$i] . "</td>";
+ echo "<th>" . $namaHari[$i] . "</th>";

$colIdx = $dataHariIni->hari + 1;   // + 1 karena perbedaan selisih index tabel dan value hari di database 
$rowIdx = $dataHariIni->jam_mulai - 6;  // + 1 karena perbedaan selisih index tabel dan value jam_mulai di database 
@@ -84,27 +84,29 @@
+ $border = "border border-secondary align-middle";

$($cellLocation).css('background-color', '<?php echo $color; ?>');
$($cellLocation).attr('rowspan', <?php echo $dataHariIni->durasi ?>);
+ $($cellLocation).addClass('<?php echo $border; ?>');

#    Implementasi Komponen Button
@@ -135,13 +135,13 @@
- <a href="/LihatJadwalDosen/export/" class="button">Ekspor ke XLS</a>
+ <a class="btn btn-primary" href="/LihatJadwalDosen/export/" class="button">Ekspor ke XLS</a>
\end{lstlisting}


 





