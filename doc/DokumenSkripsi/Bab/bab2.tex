%versi 2 (8-10-2016)
\chapter{Landasan Teori}

\section{BlueTape}
Bluetape merupakan aplikasi berbasis web, berguna sebagai aplikasi yang menunjang proses administrasi dalam lingkungan FTIS UNPAR. Web ini dapat diakses pada link: \url{http://www.bluetape.azurewebsites.net}. \cite{blueTape}
Setiap subbab menampilkan antarmuka halaman dan penjelesan fitur yang tersedia serta siapa yang dapat mengakses halaman tersebut (dosen, staf TU atau mahasiswa).

\subsection{Login}
\label{ss:login}
\noindent Gambar ~\ref{fig:tampilan_login} adalah halaman pertama yang diakses \textit{user}.
\begin{figure} [H]
	\centering  
	\includegraphics[width=\textwidth,height=\textheight,keepaspectratio]{tampilan_login.png} 
	\caption{Antarmuka halaman login.} 
	\label{fig:tampilan_login}
\end{figure}
Halaman utama aplikasi BlueTape akan mengarahkan \textit{user} untuk \textit{login} dengan menggunakan Google, user akan login dengan melihat beberapa kondisi ini:
\begin{itemize}
	\item Apabila \textit{user} belum pernah login menggunakan akun UNPAR(xxx@student.unpar.ac.id atau yyy@unpar.ac.id) maka  \textit{user} akan diminta untuk memasukan email UNPAR dan password
	\item Apabila user sudah pernah login menggunakan akun UNPAR, maka \textit{user} akan diminta untuk memilih akun beserta password.
	\item User akan terhubung otomatis dengan akun @gmail.com. Apabila BlueTape menolak autentikasi user maka: User akan diminta untuk buka halaman Gmail lalu klik avatar di kanan atas dan memilih akun UNPAR yang tepat pada tombol "Add Account"
\end{itemize}
User akan melihat beberapa menu sesuai dengan \textit{role} user, sebagai mahasiswa, staf TU, dll.\\
\noindent Pada bagian atas halaman login terdapat notifikasi bagi user saat proses login dan logout. 

\subsection{Dosen} 
Apabila dosen masuk ke website BlueTape maka dapat mengakses dua halaman yaitu halaman perubahan kuliah dan halaman entri jadwal dosen.

\subsubsection{Perubahan Kuliah}
Halaman permintaan perubahan kuliah terdiri dari dua konten: "Permohonan Baru" dan "Histori Permohonan" tertera pada gambar ~\ref{fig:tampilanPerubahanKuliah}. 
\begin{figure} [H]
	\centering  
	\includegraphics[width=\textwidth,height=\textheight,keepaspectratio]{tampilan_perubahan_kuliah.png} 
	\caption{Antarmuka halaman perubahan kuliah.} 
	\label{fig:tampilanPerubahanKuliah}
\end{figure}
Modul perubahan kuliah berguna untuk mengirimkan permintaan perubahan mata kuliah yang dikirim oleh dosen kepada staf Tata Usaha. Kolom - kolom yang terdapat dalam modul ini:
\begin{itemize}
	\item Kode MK (Mata Kuliah)
	\item Nama Mata Kuliah
	\item Kelas
	\item Jenis perubahan (diganti / tambahan / ditiadakan)
	\item Dari (hari/jam dan ruang) dan ke(hari/jam dan tempat)
	\item Keterangan
\end{itemize}
Setiap konten dipisahkan oleh border. Pada "Permohonan Baru" terdapat sebuah form yang berisi field-fiel dengan lebar yang berbeda - beda dan dikelompokan pada beberapa baris. Terdapat tiga tombol yaitu tombol biru "Kirim Permohonan" dan tombol abu-abu "Tambah pertemuan Ektra" .
Konten "Histori Permohonan" berisi tabel bergaris yang menampilkan data histori permohonan. \noindent
Dosen juga dapat membuat lebih dari 1 kelas pengganti, dengan mengklik tombol "Tambah Pertemuan Ekstra", nantinya baris baru akan terbentuk. 
\subsubsection{Entri Jadwal Dosen}
Entri Jadwal Dosen berisi dua konten: "Tambah Jadwal" dan "Daftar Jadwal". Konten "Tambah Jadwal" tertera pada gambar ~\ref{fig:tampilanEntriJadwalDosen}. \textit{Form} yang terdiri dari label, \textit{field} dari form tersebut dan tombol biru "Tambah". Konten "Daftar Jadwal" terdiri dari sebuah tabel dimana untuk \textit{cell} yang terisi mereferensikan ke modal edit jadwal dosen. 
\begin{figure} [H]
	\centering  
	\includegraphics[width=\textwidth,height=\textheight,keepaspectratio]{tampilan_entri_jadwal_dosen.png} 
	\caption{Antarmuka halaman entri jadwal dosen.} 
	\label{fig:tampilanEntriJadwalDosen}
\end{figure}
Selain itu terdapat dua tombol pada bawah tabel: tombol merah "Delete" dan tombol biru "Ekspor ke XLS".
\subsubsection{Tambah Jadwal}
Pada bagian entri jadwal, dosen informatika dapat mengisikan hari, jam mulai, durasi, label, dan sejenisnya. Berikut ini jenis yang dapat dipilih:
\begin{itemize}
	\item Konsultasi: \textit{Cell} memiliki background kuning.
	\item Terjadwal: \textit{Cell} memiliki background hijau.
	\item Kelas: \textit{Cell} memiliki background putih.
\end{itemize}

\subsubsection{Ubah/Hapus Jadwal}
Dosen dapat mengubah atau menghapus jadwal yang tertera pada tabel. Lalu \textit{pop-up} window akan terbuka dengan pilihan-pilihan yang sesuai dengan permintaan dosen.

\subsubsection{Hapus Semua}
Tombol "Delete All" memiliki \textit{button} berwarna merah dapat digunakan untuk menghapus secara cepat seluruh jadwal yang telah dosen buat sebelumnya.

\subsubsection{Ekspor ke XLS}
Tombol "Ekspor ke XLS" memiliki \textit{button} berwarna hijau berfungsi untuk membuat file excel (.xls) untuk jadwal dosen.

\subsection{Mahasiswa}
Apabila mahasiswa masuk ke website BlueTape maka dapat mengakses dua halaman yaitu halaman cetak transkrip dan halaman lihat jadwal dosen.
\subsubsection{Cetak Transkrip}
\noindent Gambar ~\ref{fig:tampilanCetakTranskrip} adalah halaman permintaan cetak transkrip yang dapat diakses oleh mahasiswa.
\begin{figure} [H]
	\centering  
	\includegraphics[width=\textwidth,height=\textheight,keepaspectratio]{tampilan_cetak_transkrip.png} 
	\caption{Antarmuka halaman permintaan cetak transkrip.} 
	\label{fig:tampilanCetakTranskrip}
\end{figure}
\par Halaman permintaan cetak transkrip terdiri dari dua konten yaitu: "Permohonan Baru" dan "Histori Permohonan". Setiap konten akan dipisahkan dengan border. Pada konten "Permohonan Baru" terdiri dari sebuah form yang berisi \textit{field} yang terdiri dari 2 \textit{rows} dan sebuah tombol "Kirim Permohonan". Lalu konten "Histori Permohonan" terdiri dari sebuah tabel dimana data akan dipanggil dari \textit{database}.  dan kolom 'Aksi' menngarahkan \textit{user} ke sebuah modal lihat permintaan transkrip yang direpresentasikan dengan ikon mata.\\ \\

\subsubsection{Lihat Jadwal Dosen}
\noindent Gambar ~\ref{fig:tampilanLihatJadwalDosen} adalah halaman permintaan cetak transkrip yang dapat diakses oleh mahasiswa.
\begin{figure} [H]
	\centering  
	\includegraphics[width=\textwidth,height=\textheight,keepaspectratio]{tampilan_lihat_jadwal_dosen.png} 
	\caption{Antarmuka halaman lihat jadwal dosen.} 
	\label{fig:tampilanLihatJadwalDosen}
\end{figure}
Mahasiswa dapat melihat jadwal mingguan berbentuk tabel dan diatas tabel memiliki tabs yang menyimpan seluruh nama dosen dan ketika tabs dipilih maka tabel jadwal dosen akan ditampilkan pada bagian bawah halaman. 
Lalu terdapat tombol "Ekspor ke XLS" berupa \textit{button} berwara hijau pada halaman lihat jadwal dosen. 

\subsection{Staf Tata Usaha}
Apabila staf tata usaha masuk ke website BlueTape maka dapat mengakses dua halaman yaitu halaman manajemen perubahan kuliah dan halaman manajemen cetak transkrip.
\subsubsection{Manajemen Perubahan Kuliah}
\noindent Gambar ~\ref{fig:tampilanManajemenPerubahanKuliah} adalah halaman permintaan cetak transkrip yang dapat diakses oleh staf tata usaha.
\begin{figure} [H]
	\centering  
	\includegraphics[width=\textwidth,height=\textheight,keepaspectratio]{tampilan_manajemen_perubahan_kuliah.png} 
	\caption{Antarmuka halaman manajemen perubahan kuliah.} 
	\label{fig:tampilanManajemenPerubahanKuliah}
\end{figure}
Staf Tata Usaha dapat melakukan manajemen permintaan perubahan kuliah. Sebuah tabel akan menampilkan daftar permohonan dengan menampilkan tanggal kapan permohonan dibuat.
Setiap daftar permohonan akan memiliki beberapa tombol :

\begin{itemize}
	\item \includegraphics[height=0.7\baselineskip]{tombol_eye.png} berfungsi untuk melihat detail permohonan sehingga dapat menentukan apakah permohonan disetujui atau tidak.
	\item \includegraphics[height=0.7\baselineskip]{tombol_print.png} berfungsi untuk membuka pop-up print-out pengumuman.
	\item \includegraphics[height=0.7\baselineskip]{tombol_good.png} berfungsi sebagai konfirmasi bahwa pengumuman telah dicetak dan disebarkan.
	\item \includegraphics[height=0.7\baselineskip]{tombol_bad.png} berfungsi untuk menyatakan bahwa permohonan ditolak. Staf Tata Usaha akan mengisi alasan mengapa permohonan ditolak sehingga tidak membingungkan pemohon.
	\item \includegraphics[height=0.7\baselineskip]{tombol_trash.png} berfungsi untuk menghapus permohonan \textbf{secara permanen}. Staf Tata Usaha dihimbau agar tidak menggunakan tombol ini kecuali dalam keadaan terpaksa.
\end{itemize}

\subsubsection{Manajemen Cetak Transkrip}
\noindent Gambar ~\ref{fig:tampilanManajemenCetakTranskrip} adalah halaman permintaan cetak transkrip yang dapat diakses oleh staf tata usaha.
\begin{figure} [H]
	\centering  
	\includegraphics[width=\textwidth,height=\textheight,keepaspectratio]{tampilan_manajemen_cetak_transkrip.png} 
	\caption{Antarmuka halaman manajemen cetak transkrip.} 
	\label{fig:tampilanManajemenCetakTranskrip}
\end{figure}
Staf Tata Usaha dapat melihat daftar perminttan transkrip dalam bentuk tabel. Keterangan mengenai transkrip dapat dilihat menggunakan tombol \includegraphics[height=0.6\baselineskip]{tombol_eye.png} (detail). Selain itu terdapat dua pilihan jawaban dalam setiap daftar permintaan yaitu \includegraphics[height=0.7\baselineskip]{tombol_bad.png} (tolak) dan 
\includegraphics[height=0.7\baselineskip]{tombol_print.png} (cetak). Masing-masing tombol memerlukan keterangan tambahan mengeai alasan mengapa transkrip dapat dicetak maupun ditolak.\noindent
Modul ini berguna untuk manajemen permohonan cetka transkrip. Terdapat sebuah tabel yang menapilkan daftar pemohonan dengan tanggal yang terurut. Staf Tata Usaha dapat mencari daftar permintaan berdasarkan NPM pemohon.
Beberapa tombol yang tersedia untuk setiap permohonan :

\begin{itemize}
	\item \includegraphics[height=0.7\baselineskip]{tombol_eye.png} berfungsi untuk melihat detail permohonan sehingga dapat menentukan apakah permohonan disetujui atau tidak.
	: \item \includegraphics[height=0.7\baselineskip]{tombol_print.png} berfungsi untuk membuka pop-up print-out pengumuman. Dalam pop-up akan disediakan sebuah link menuju halaman percetakan transkrip pada SIAkad.
	\item \includegraphics[height=0.7\baselineskip]{tombol_bad.png} berfungsi untuk menyatakan bahwa permohonan ditolak. Staf Tata Usaha akan mengisi alasan mengapa permohonan ditolak sehingga tidak membingungkan pemohon.
	\item \includegraphics[height=0.7\baselineskip]{tombol_trash.png} berfungsi untuk menghapus permohonan \textbf{secara permanen}. Staf Tata Usaha dihimbau agar tidak menggunakan tombol ini kecuali dalam keadaan terpaksa.
\end{itemize}  

\section{CodeIgniter}
\label{sec:codeigniter}
CodeIgniter adalah kerangka pengembangan aplikasi yang digunakan programmer untuk membangun web menggunakan PHP. Tujuannya adalah memungkinkan programmer mengembangkan proyek-proyek dengan lebih cepat daripada  menulis kode dari awal, tersedia banyak library untuk tugas-tugas yang biasa diperlukan dalam membangun aplikasi, serta antarmuka dan struktur logis yang sederhana untuk mengakses library tersebut. 
Pada bagian ini akan dibahas konsep dasar untuk menjalankan \textit{framework} CodeIgniter:
\begin{itemize}
	\item \textit{Application Flow Chart} : Penjelasan bagaimana data berjalan dalam sistem.
	\item URLs : Penjelasan penggunaan \textit{routing} dalam CodeIgniter.
	\item Konsep Model-View-Controller : Penjelasan pola desain arsitektur pengembangan aplikasi yang digunakan untuk mengatur file sehingga kode mudah dikelola. MVC membuat data, presentasi dan alur dalam aplikasi berjalan secara terpisah.
\end{itemize}
Pola Model, View, Controller (MVC) 

\subsection{Application Flow Chart}
\begin{figure} [H]
	\centering  
	\includegraphics[scale=1.0]{appflowchart.png}  
	\caption{Flow Chart Aplikasi CodeIgniter}
	\label{fig:flowChartCodeIgniter} 
\end{figure}
\noindent Gambar ~\ref{fig:flowChartCodeIgniter} mengilustrasikan bagaimana alur kerja pembuatan aplikasi berbasis \textit{framework} CodeIgniter:
\begin{enumerate}
	\item \texttt{Index.php} : bertindak sebagai \textit{front controller}, menginisiasi \textit{base resources} yang dibutuhkan untuk menjalankan CodeIgniter.
	\item \texttt{Router} : akan memeriksa permintaan HTTP untuk menetapkan hal apa yang harus dilakukan dengan permintaan tersebut.
	\item \texttt{Cache} : Apabila terdapat \textit{cache}, maka \textit{cache} tersebut akan dikirimkan langsung ke browser, dengan melewati sistem eksekusi normal.
	\item \texttt{Security} : Sebelum \textit{controller} dimuat, \textit{HTTP request} dan \textit{user} mana pun yang mengirimkan data diseleksi dahulu untuk keamanan.
	\item \texttt{Controller} : Terdiri dari \textit{model, core libraries, helpers}, dan \textit{resources} yang dibutuhkan untuk proses \textit{request} tertentu.
	\item \texttt{View} : Tampilan yang telah selesai dirubah kemudian dikirim ke \textit{web browser} untuk dilihat. Jika \textit{caching} diaktifkan, tampilan di \textit{cache} terlebih dahulu sehingga pada permintaan selanjutnya dapat dilayani.
	\cite{codeigniter}
\end{enumerate}

\subsection{CodeIgniter URLs}
URL routing (route) adalah salah satu metode untuk memetakan URL ke dalam sumber daya tertentu dengan memberikan nama lain dari sumber daya yang dimaksud.
Penerapan \textit{route} pada CodeIgniter menggunakan pendekatan berbasis segmen :
\begin{lstlisting}[style=customphp, language=PHP, basicstyle=\ttfamily, frame=single, columns=fullflexible, keepspaces=true, breaklines=true, showstringspaces=false, label={lst:urlCI}, caption=Kode URLs pada CodeIgniter] 
example.com/class/function/ID
\end{lstlisting}

\noindent Segmen URL menggunakan pendekatan Model-View-Controller. Penjelasaan mengenai MVC ada pada bagian ~\ref{s:modelCI}, ~\ref{s:viewCI} dan ~\ref{s:controllerCI}. Berikut ini penjelasan dari kode ~\ref{lst:urlCI}.
\begin{enumerate}
	\item Segmen pertama menyatakan kelas \textit{controller} yang harus dipanggil.
	\item Segmen kedua menyatakan fungsi kelas, atau metode, yang harus dipanggil.
	\item Segmen ketiga dan setiap segmen setelahnya menyatakan ID dan variabel apa pun yang akan diteruskan ke \textit{controller}.
\end{enumerate}

\subsection{Model}
\label{s:modelCI}
\textit{Model} merepresentasikan struktur data. Biasanya kelas \textit {model} akan berisi fungsi yang membantu untuk \textit{retrieve, insert}, dan \textit{update} informasi di database. Contohnya pada kode ~\ref{lst:modelCII} menggambarkan sebuah model yang berisi fungsi \textit{insert, update} dan \textit{retrieve}:

\begin{lstlisting}[style=customphp, language=PHP, basicstyle=\ttfamily, frame=single, columns=fullflexible, keepspaces=true, breaklines=true, showstringspaces=false, label={lst:modelCII}, caption=Struktur model pada codeIgniter.]  
class Model_name extends CI_Model {
	class Blog_model extends CI_Model {	
		public $title;
		public $content;
		public $date;
		
		public function get_last_ten_entries()
		{
			$query = $this->db->get('entries', 10);
			return $query->result();
		}
		
		public function insert_entry()
		{
			$this->title    = $_POST['title']; 
			$this->content  = $_POST['content'];
			$this->date     = time();
			
			$this->db->insert('entries', $this);
		}
		
		public function update_entry()
		{
			$this->title    = $_POST['title'];
			$this->content  = $_POST['content'];
			$this->date     = time();
			
			$this->db->update('entries', $this, array('id' => $_POST['id']));
		}
	
	}
}
\end{lstlisting}
Bagian model akan menjelaskan bagaimana anatomi suatu model , me-\textit{load} model tersebut  dan menghubungkan nya dengan database.

\subsubsection{Anatomi Model}
\label{ss:anatomiModelCI}
Kelas model akan disimpan dalam direktori \texttt{application/models/directory}. Kode ~\ref{lst:modelCI} adalah prototipe dasar dari sebuah model kelas :

\begin{lstlisting}[style=customphp, language=PHP, basicstyle=\ttfamily, frame=single, columns=fullflexible, keepspaces=true, breaklines=true, showstringspaces=false, label={lst:modelCI}, caption=Struktur model pada codeIgniter.]  
class Model_name extends CI_Model {

}
\end{lstlisting}

Dimana \texttt{Model\_name} merupakan nama dari kelas. Kelas harus menggunakan huruf kapital pada huruf pertama dan sisanya menggunakan huruf kecil. Apabila kode ~\ref{lst:modelCI} digunakan maka file akan terlihat seperti kode ~\ref{lst:userCI}
\begin{lstlisting}[style=customphp, language=PHP, basicstyle=\ttfamily, frame=single, columns=fullflexible, keepspaces=true, breaklines=true, showstringspaces=false, label={lst:userCI}, caption=Contoh model codeIgniter.]  
application/models/Blog_model.php
\end{lstlisting}

\subsubsection{Loading a Model}
\label{ss:loadModelCI}
Kemudian model akan dimuat dan dipanggil didalam metode \textit{controller}. Untuk memuat sebuah model dapat menggunakan metode pada kode ~\ref{lst:loadCI}:

\begin{lstlisting}[style=customphp, language=PHP, basicstyle=\ttfamily, frame=single, columns=fullflexible, keepspaces=true, breaklines=true, showstringspaces=false, label={lst:loadCI}, caption=Contoh pemanggilan model pada codeIgniter.] 
$this->load->model('model_name');
\end{lstlisting}

\subsubsection{Koneksi ke Database}
\label{ss:databaseModelCI}
Model yang sudah dimuat tidak terhubung secara langsung ke database. Secara manual dapat diatur konektivitas database melalui parameter ketiga yang tertera pada kode ~\ref{lst:databaseCI} :

\begin{lstlisting}[style=customphp, language=PHP, basicstyle=\ttfamily, frame=single, columns=fullflexible, keepspaces=true, breaklines=true, showstringspaces=false, label={lst:databaseCI}, caption=Struktur database pada CodeIgniter.]
$config['hostname'] = 'localhost';
$config['username'] = 'myusername';
$config['password'] = 'mypassword';
$config['database'] = 'mydatabase';
$config['dbdriver'] = 'mysqli';
$config['dbprefix'] = '';
$config['pconnect'] = FALSE;
$config['db_debug'] = TRUE;

$this->load->model('model_name', '', $config);
\end{lstlisting}

\subsection{View}
\label{s:viewCI}
\par \textit{View } adalah informasi yang disajikan kepada pengguna. Tampilan atau \textit{View} biasanya akan menjadi halaman web, tetapi dalam CodeIgniter tampilan juga bisa berupa fragmen halaman seperti header atau footer. \textit{Views} tidak pernah dipanggil secara langsung, harus dimuat dalam sebuah \textit{controller}. Dalam \textit{MVC framework}, \textit{controller} bertanggung jawab untuk mengambil \textit{view} tertentu. \\ \\
Berikut ini akan dijelaskan bagaimana membuat sebuah view, menge-\textit{load} view , menyimpan view tersebut di dalam subdirektori dan menambahkan data dinamis ke dalam view.

\subsubsection{Membuat sebuah View}
\label{ss:sebuahViewCI}
Untuk membuat sebuah view dalam CodeIgniter, pertama dengan membuat file php yang berisi kode HTML. Misalkan file \verb|blogview.php| yang terlihat pada kode ~\ref{lst:viewCI}:
\begin{lstlisting}[style=customhtml, language=HTML, basicstyle=\ttfamily, frame=single, columns=fullflexible, keepspaces=true, breaklines=true, showstringspaces=false, label={lst:viewCI}, caption=Contoh view pada CodeIgniter.] 
<html>
	<head>
		<title>My Blog</title>
	</head>
	<body>
		<h1>Welcome to my Blog!</h1>
	</body>
</html>
\end{lstlisting}

\noindent Kemudian file php tersebut disimpan dalam direktori \texttt{application/views/}.

\subsubsection{Loading sebuah View}
\label{ss:loadViewCI}
Untuk memuat sebuah \textit{view}, maka dapat menggunakan sintaks  ~\ref{lst:loadViewCI}.

\begin{lstlisting}[style=customphp, language=PHP, basicstyle=\ttfamily, frame=single, columns=fullflexible, keepspaces=true, breaklines=true, showstringspaces=false, label={lst:loadViewCI}, caption=Sintaks load view pada CodeIgniter.] 
$this->load->view('name');
\end{lstlisting}

Apabila terdapat file \texttt{Blog.php} yang berisi objek Blog  dan ingin me-\textit{load} \textit{view}-nya yang terdapat di \texttt{blogview.php} maka dapat dipanggil sesuai dengan kode ~\ref{lst:blogView} pada line ke-6:
\begin{lstlisting}[style=customphp, language=PHP, basicstyle=\ttfamily, frame=single, columns=fullflexible, keepspaces=true, breaklines=true, showstringspaces=false, label={lst:blogView}, caption=Contoh load view pada CodeIgniter.] 
<?php
	class Blog extends CI_Controller {
	
	public function index()
	{
		$this->load->view('blogview');
	}
}
\end{lstlisting}

\noindent Setelah di\textit{load} sebuah view yang dibuat akan menghasilkan URL baru seperti terlihat pada kode ~\ref{lst:urlViewCI}:
\begin{lstlisting}[style=customphp, language=PHP, basicstyle=\ttfamily, frame=single, columns=fullflexible, keepspaces=true, breaklines=true, showstringspaces=false, label={lst:urlViewCI}, caption=Contoh load view pada CodeIgniter.] 
example.com/index.php/blog/
\end{lstlisting}

\subsubsection{Memuat Beberapa View}
\label{ss:beberapaViewCI}
CodeIgniter mampu menangani beberapa panggilan dari dalam \textit{controller}. Apabila ada lebih dari satu panggilan yang terjadi, maka \textit{views} akan dilampirkan secara bersamaan.



\noindent Berikut ini kode ~\ref{lst:loadSomeViewCI} yang digunakan jika \textit{developer} ingin mempunyai sebuah halaman yang terdiri dari \texttt{header, menu, content} dan \texttt{footer}. 


\begin{lstlisting}[style=customphp, language=PHP, basicstyle=\ttfamily, frame=single, columns=fullflexible, keepspaces=true, breaklines=true, showstringspaces=false, label={lst:loadSomeViewCI}, caption=Contoh load beberapa view pada CodeIgniter.] 
<?php

class Page extends CI_Controller {

	public function index()
	{
		$data['page_title'] = 'Your title';
		$this->load->view('header');
		$this->load->view('menu');
		$this->load->view('content', $data);
		$this->load->view('footer');
	}

}
?>
\end{lstlisting}

\subsubsection{Menyimpan Views didalam Sub Direktori}
\label{ss:subdirektoriViewCI}
Untuk menyimpan didalam sub direktori maka dapat menyertakan nama direktori yang memuat \textit{view}, sintaks yang digunakan tertera pada kode ~\ref{lst:loadSubDirektoriCI}:
\begin{lstlisting}[style=customphp, language=PHP, basicstyle=\ttfamily, frame=single, columns=fullflexible, keepspaces=true, breaklines=true, showstringspaces=false, label={lst:loadSubDirektoriCI}, caption=Struktur penyimpanan views dalam sub direktori.] 
$this->load->view('directory_name/file_name');
\end{lstlisting}

\subsubsection{Menambahkan data dinamis ke View}
\label{ss:dinamisViewCI}
Data yang dikirim dari controller menuju view berbentuk array atau objek, sehingga dapat dilampirkan pada parameter kedua dalam view untuk metode loading.\\



\noindent Kode ~\ref{lst:dynamicDataCI} menjelaskan penggunaan dengan array:
\begin{lstlisting}[style=customphp, language=PHP, basicstyle=\ttfamily, frame=single, columns=fullflexible, keepspaces=true, breaklines=true, showstringspaces=false, label={lst:dynamicDataCI}, caption=Struktur data dinamis.] 
$data = array(
	'title' => 'My Title',
	'heading' => 'My Heading',
	'message' => 'My Message'
);

$this->load->view('blogview', $data);
\end{lstlisting}

\noindent Kemudian, kode ~\ref{lst:dynamicObjectDataCI} menjelaskan penggunaan dengan objek:
\begin{lstlisting}[style=customphp, language=PHP, basicstyle=\ttfamily, frame=single, columns=fullflexible, keepspaces=true, breaklines=true, showstringspaces=false, label={lst:dynamicObjectDataCI}, caption=Struktur data dinamis berbentuk objek.] 
$data = new Someclass();
$this->load->view('blogview', $data);
\end{lstlisting}

\noindent Sehingga apabila dimasukan ke controller, kode yang ditambahkan seperti terlihat pada kode ~\ref{lst:dynamicControllerDataCI}:
\begin{lstlisting}[style=customphp, language=PHP, basicstyle=\ttfamily, frame=single, columns=fullflexible, keepspaces=true, breaklines=true, showstringspaces=false, label={lst:dynamicControllerDataCI}, caption=Struktur data dinamis dalam controller.] 
class Blog extends CI_Controller {

	public function index()
	{
		$data['title'] = "My Real Title";
		$data['heading'] = "My Real Heading";
		
		$this->load->view('blogview', $data);
	}
}
\end{lstlisting}

\noindent Untuk mengakses data dari model Blog, maka dapat dipanggil dengan variabel nya seperti terlihat pada kode ~\ref{lst:htmlCI} :
\begin{lstlisting}[style=customhtml, language=HTML, basicstyle=\ttfamily, frame=single, columns=fullflexible, keepspaces=true, breaklines=true, showstringspaces=false, label={lst:htmlCI}, caption=Akses data dinamis dalam file HTML.] 
<html>
	<head>
		<title><?php echo $title;?></title>
	</head>
	<body>
		<h1><?php echo $heading;?></h1>
	</body>
</html>
\end{lstlisting}

\subsection{Controller}
\label{s:controllerCI}
\par \textit{Controller} bertindak sebagai penengah antara Model, View dan \textit{resources} lain yang dibutuhkan untuk proses \textit{HTTP requests} dan untuk menghasilkan sebuah halaman web. \textit{Controller} secara sederhana merupakan sebuah file yang dinamakan dengan aturan tertentu sehingga dapat dihubungkan dengan sebuah URl.

Apabila terdapat kelas \texttt{Blog} yang terlihat pada kode ~\ref{lst:contohControllerCI}.
\begin{lstlisting}[style=customphp, language=PHP, basicstyle=\ttfamily, frame=single, columns=fullflexible, keepspaces=true, breaklines=true, showstringspaces=false, label={lst:contohControllerCI}, caption=Contoh controller pada codeIgniter.] 
<?php
	class Blog extends CI_Controller {
	
		public function index()
		{
			echo 'Hello World'
		}
	}
?>
\end{lstlisting} 



Maka controller yang dipanggil akan menjadi tertera berikut:
\begin{lstlisting}[style=customphp, language=PHP, basicstyle=\ttfamily, frame=single, columns=fullflexible, keepspaces=true, breaklines=true, showstringspaces=false, label={lst:controllerCI}, caption=Controller pada codeIgniter.] 
example.com/index.php/blog/
\end{lstlisting}

Dalam kode ~\ref{lst:controllerCI}, \textit{CodeIgniter} berusaha menemukan \textit{controller} bernama \texttt{Blog.php} untuk dimuat. Ketika sebuah nama \textit{controller} sesuai dengan segmen pertama dari sebuah URl, maka URl akan memuatnya. 


\section{jQuery}

jQuery merupakan \textit{library} Javascript dimana setiap kode yang dibuat akan disimpan ke dalam \textit{methods} yang dapat dipanggil \textit{user} dengan satu baris kode. Bagian ini akan menjelaskan penggunaan dasar pada jQuery. \cite{jquery}

\subsection{Penggunaan Dasar jQuery}
\noindent Sintaks dalam jQuery dibuat untuk memilih elemen dalam HTML dan melakukan beberapa aksi dalam elemen. 

Sintaks dasar dalam jQuery tertera pada kode ~\ref{lst:dasarjQuery}:
\begin{lstlisting}[style=JavaScript, language=JavaScript, basicstyle=\ttfamily, frame=single, columns=fullflexible, keepspaces=true, breaklines=true, showstringspaces=false, label={lst:dasarjQuery}, caption=Pemanggilan jQuery.]
$(selector).action()
\end{lstlisting}
Penjelasan sintaks sebagai berikut:
\begin{itemize}
	\item \texttt{\$}: berfungsi untuk mendefinisikan library jQuery.
	\item \texttt{selector}: berfungsi untuk menjalankan \textit{query} atau menemukan elemen HTML.
	\item \texttt{action()}: berfungsi untuk menjalankan elemen.
\end{itemize}



Berikut ini contoh penggunaan sintaks dalam jQuery:
\begin{itemize}
	\item \texttt{\$(this).hide()}: menyembunyikan elemen saat ini.
	\item \texttt{\$("p").hide()}: menyembunyikan elemen \texttt{<p>}.
	\item \texttt{\$(".test").hide()}: menyembunyikan elemen \texttt{class="test"}.
	\item \texttt{\$("\#test").hide()}: menyembunyikan elemen \texttt{id="test"}.
\end{itemize}

\subsection{Metode \texttt{addClass()}}
jQuery mempunyai metode untuk memanipulasi \textit{style} dari sebuah elemen dengan menggunakan metode \texttt{addClass()}. Metode ini menambahkan atribut kelas ke elemen yang berbeda. 
\begin{lstlisting}[style=JavaScript, language=JavaScript,  basicstyle=\ttfamily, frame=single, columns=fullflexible, keepspaces=true, breaklines=true, showstringspaces=false, label=stylejQuery]
.important {
font-weight: bold;
font-size: xx-large;
}

.blue {
color: blue;
}
\end{lstlisting}
\begin{lstlisting}[style=JavaScript, language=JavaScript, basicstyle=\ttfamily, frame=single, columns=fullflexible, keepspaces=true, breaklines=true, showstringspaces=false, label={lst:addClass}, caption=Metode addClass().]
$("button").click(function(){
$("#div1").addClass("important blue");
});
\end{lstlisting}


\section{Foundation 6}
\label{sec:foundation}
Foundation adalah \textit{front-end framework} yang responsif yang membuatnya mudah untuk merancang situs web, aplikasi, dan email di perangkat apa pun.

\subsection{Struktur File}
Berikut ini isi dari \textit{package} Foundation 6 terlihat pada listing ~\ref{lst:fileFoundation}.
\begin{lstlisting}[basicstyle=\ttfamily, frame=single, columns=fullflexible, keepspaces=true, breaklines=true, showstringspaces=false, label={lst:fileFoundation}, caption=Struktur File Foundation.]
.
|-- css
|   |-- css 
|   |-- foundatcss 
|   |-- foundation-datetimepicker.css 
|   |-- foundation-fcss 
|   |-- foundation-icss 
|   |-- foundation-ieot 
|   |-- foundation-isvg 
|   |-- foundation-ittf 
|   |-- foundation-icoff 
|-- js
|   |-- vendor 
|   |-- app.js 
|   |-- foundation.js 
|-- img
.
\end{lstlisting}

Framework Foundation terdiri dari 3 folder utama:
\begin{itemize}
	\item Folder \texttt{css} terdiri dari semua \textit{CSS Style} yang digunakan dalam Foundation 6. Didalam folder terdapat versi yag diperkecil yaitu \verb|foundation.min.css| atau versi yang tidak dikompresi \verb|foundation.css|. Lalu seluruh modifikasi \textit{stylesheets} ditempatkan didalam folder ini agar lebih terstruktur.
	\item Folder \texttt{img} tempat meletakkan semua gambar untuk projek web.
	\item Folder \texttt{js} terdiri dari semua file Javascript.
\end{itemize} 
\cite{zurbfoundation:17}

\subsection{Sistem Grid}
Kode ~\ref{lst:gridFoundation} terlihat penggunaan grid pada Foundation dapat dilakukan dengan menambahkan sebuah elemen dengan sebuah kelas \texttt{.row} sehingga akan membuat blok horizontal yang berisi kolom vertikal. Kemudian kelas \texttt{.column} akan ditambahkan pada baris tersebut, lalu masing-masing kolom ditentunkan kelasnya dengan tiga pilihan yaitu 
\texttt{.small-\#}, \texttt{.medium-\#} dan \texttt{.large-\#}.

\begin{lstlisting}[style=customhtml, language=HTML,  basicstyle=\ttfamily, frame=single, columns=fullflexible, keepspaces=true, breaklines=true, showstringspaces=false, label={lst:gridFoundation}, caption=Struktur Grid Foundation 6.] 
<div class="row">
	<div class="columns small-2 large-4"><!-- ... --></div>
	<div class="columns small-4 large-4"><!-- ... --></div>
	<div class="columns small-6 large-4"><!-- ... --></div>
</div>
<div class="row">
	<div class="columns large-3"><!-- ... --></div>
	<div class="columns large-6"><!-- ... --></div>
	<div class="columns large-3"><!-- ... --></div>
</div>
<div class="row">
	<div class="columns small-6 large-2"><!-- ... --></div>
	<div class="columns small-6 large-8"><!-- ... --></div>
	<div class="columns small-12 large-2"><!-- ... --></div>
</div>
<div class="row">
	<div class="columns small-3"><!-- ... --></div>
	<div class="columns small-9"><!-- ... --></div>
</div>
<div class="row">
	<div class="columns large-4"><!-- ... --></div>
	<div class="columns large-8"><!-- ... --></div>
</div>
<div class="row">
	<div class="columns small-6 large-5"><!-- ... --></div>
	<div class="columns small-6 large-7"><!-- ... --></div>
</div>
<div class="row">
	<div class="columns large-6"><!-- ... --></div>
	<div class="columns large-6"><!-- ... --></div>
</div>
\end{lstlisting}

\begin{figure} [H]
	\centering  
	\includegraphics[scale=0.7]{gridbasic_zurb.png}  
	\caption{Grid pada Foundation 6}
	\label{fig:gridFoundation}	 
\end{figure}

\subsection{Navigation dan Media Attributes}
Komponen menu yang fleksibel pada Foundation membuat pembangunan navigasi secara umum lebih mudah karena semua pola memiliki markup yang sama.

\subsubsection{Basic Menu}
Menu terdiri dari sebuah \texttt{<ul>} yang diisi oleh beberapa tag \texttt{<li>}. Secara default, menu akan berorientasi horizontal.

Berikut ini contoh penggunaan kode navigasi pada menu yang terlihat pada kode ~\ref{lst:basicFoundation}):

\begin{lstlisting}[style=customhtml, language=HTML,  basicstyle=\ttfamily, frame=single, columns=fullflexible, keepspaces=true, breaklines=true, showstringspaces=false, label={lst:basicFoundation}, caption=Basic Menu Foundation 6.] 
<ul class="menu">
	<li><a href="#">One</a></li>
	<li><a href="#">Two</a></li>
	<li><a href="#">Three</a></li>
	<li><a href="#">Four</a></li>
</ul>
\end{lstlisting}

\begin{figure} [H]  
	\centering  
	\includegraphics[scale=0.7]{basicmenu_zurb.png}  
	\caption{\textit{Basic Navigation Menu} pada Foundation}
	\label{fig:navFoundation}
\end{figure}

\subsubsection{Item Alignment}
Secara default, setiap item dalam menu akan berlajur ke arah kiri. Menu dapat diubah lajurnya ke arah kanan dengan menggunakan kelas \texttt{.align-right} (terlihat pada kode ~\ref{lst:itemAlignmentFoundation}) didalam kelas \texttt{.menu}.
\begin{lstlisting}[style=customhtml, language=HTML,  basicstyle=\ttfamily, frame=single, columns=fullflexible, keepspaces=true, breaklines=true, showstringspaces=false, label={lst:itemAlignmentFoundation}, caption=Item Alignment Foundation 6.] 
<ul class="menu align-right">
	<li><a href="#">One</a></li>
	<li><a href="#">Two</a></li>
	<li><a href="#">Three</a></li>
	<li><a href="#">Four</a></li>
</ul>
\end{lstlisting}

\begin{figure}[H]
	\centering  
	\includegraphics[scale=0.7]{basicmenu_right_zurb.png}  
	\caption{Menu \textit{align to right} dalam Foundation}
	\label{fig:alignRightFoundation}
\end{figure}


\subsubsection{Active State}
Kelas \texttt{.is-active} dapat ditambahkan ke dalam tag \texttt{<li>} untuk membuat menu terpilih yang aktif terlihat saat di klik. \\
Berikut penggunaan kelas yang dicantumkan pada kode ~\ref{lst:activeStateFoundation}:

\begin{lstlisting}[style=customhtml, language=HTML,  basicstyle=\ttfamily, frame=single, columns=fullflexible, keepspaces=true, breaklines=true, showstringspaces=false, label={lst:activeStateFoundation}, caption=Active State Foundation 6.] 
<ul class="menu">
	<li class="is-active"><a>Home</a></li>
	<li><a>About</a></li>
	<li><a>Nachos</a></li>
</ul>
\end{lstlisting}
Hasil dari kode diatas terlihat pada gambar ~\ref{fig:activeStateFoundation}.
\begin{figure} [H]
	\centering  
	\includegraphics[scale=0.7]{activestatemenu_zurb.png}  
	\caption{Menu \textit{active state} dalam Foundation}
	\label{fig:activeStateFoundation}
\end{figure}


\subsubsection{Text}
Karena \textit{padding} untuk setiap item menu menggunakan tag \texttt{<a>}, maka apabila sebuah item yang hanya berisi teks, maka teks tersebut tidak selaras dengan item menu lainnya. Sehingga untuk menyiasatinya, dapat menggunakan kelas \texttt{.menu-text} yang dituliskan dalam tag \textit{<li>} dengan menyertakan sebuah teks tanpa link.\\
Berikut penggunaan kelas yang dicantumkan pada kode ~\ref{lst:textFoundation}:

\begin{lstlisting}[style=customhtml, language=HTML,  basicstyle=\ttfamily, frame=single, columns=fullflexible, keepspaces=true, breaklines=true, showstringspaces=false, label={lst:textFoundation}, caption=Text Foundation 6.]
<ul class="menu">
	<li class="menu-text">Site Title</li>
	<li><a href="#">One</a></li>
	<li><a href="#">Two</a></li>
	<li><a href="#">Three</a></li>
</ul>
\end{lstlisting}
Hasil dari kode diatas terlihat pada gambar ~\ref{fig:menuTextFoundation}:
\begin{figure} [H]
	\centering  
	\includegraphics[scale=0.7]{menutext_zurb.png}  
	\caption{\textit{Menu text} dalam Foundation}
	\label{fig:menuTextFoundation}
\end{figure}

\subsection{Komponen}
Selain elemen HTML biasa, Foundation berisi elemen antarmuka lain yang umum digunakan. Ini termasuk tombol dengan fitur canggih (misalnya, pengelompokan tombol atau tombol dengan opsi drop-down, daftar navigasi, tab horizontal dan vertikal, navigasi, label, dan pemformatan untuk pesan seperti peringatan.

\subsubsection{Button}
\textit{Basic button} dapat digunakan untuk banyak tujuan, sehingga penting untuk \textit{developer} menggunakan tag yang tepat. Berikut ini penjelasan penggunaan \textit{Basic button} dalam Foundation
\begin{itemize}
	\item Penggunaan tag \texttt{<a>} digunakan apabila tombol memiliki \textit{link} ke halaman lain, atau \textit{link} menuju ke halaman itu sendiri. Penggunaan links tidak membutuhkan JavaScript.
	\item Penggunaan tag \texttt{<button>} jika tombol melakukan tindakan yang mengubah sesuatu pada halaman seperti proses \textit{delete} atau \textit{save}. Elemen \texttt{<button>} akan membutuhkan JavaScript agar proses tersebut berfungsi. 
\end{itemize}
Berikut penggunaan kelas yang dicantumkan pada kode ~\ref{lst:buttonFoundation}:
\begin{lstlisting}[style=customhtml, language=HTML,  basicstyle=\ttfamily, frame=single, columns=fullflexible, keepspaces=true, breaklines=true, showstringspaces=false, label={lst:buttonFoundation}, caption=Button pada foundation 6.] 
<!-- Anchors (links) -->
<a href="about.html" class="button">Learn More</a>
<a href="#features" class="button">View All Features</a>

<!-- Buttons (actions) -->
<button type="button" class="success button">Save</button>
<button type="button" class="alert button">Delete</button>
\end{lstlisting}
Hasil dari kode diatas terlihat pada gambar ~\ref{fig:buttonFoundation}:
\begin{figure} [H]
	\centering  
	\includegraphics[scale=0.7]{basicbutton_zurb.png}  
	\caption{Basic Button pada Foundation}
	\label{fig:buttonFoundation}
\end{figure}

\noindent Warna pada button dapat diterapkan untuk memperlihatkan fungsi yang sesuai dengan aksi yang digunakan. Berikut penggunaan kelas yang dicantumkan pada kode ~\ref{lst:basicButtonFoundation}:
\begin{lstlisting}[style=customhtml, language=HTML,  basicstyle=\ttfamily, frame=single, columns=fullflexible, keepspaces=true, breaklines=true, showstringspaces=false, label={lst:basicButtonFoundation}, caption=Basic Button pada foundation 6.] 
<a class="button primary" href="#">Primary</a>
<a class="button secondary" href="#">Secondary</a>
<a class="button success" href="#">Success</a>
<a class="button alert" href="#">Alert</a>
<a class="button warning" href="#">Warning</a>
\end{lstlisting}

\noindent Hasil dari kode diatas terlihat pada gambar ~\ref{fig:colorButtonFoundation}:
\begin{figure} [H]
	\centering  
	\includegraphics[scale=0.7]{coloringbutton_zurb.png}  
	\caption{Coloring Button pada Foundation}
	\label{fig:colorButtonFoundation}
\end{figure}

\subsubsection{Tabel}

Tabel dalam foundation akan menjadikan proses penampilan data bersifat responsif dan memiliki tata letak yang bisa disesuaikan oleh kebutuhan \textit{developer}.

Berikut penggunaan kelas yang dicantumkan pada kode ~\ref{lst:tabelFoundation}:
\begin{lstlisting}[style=customhtml, language=HTML,  basicstyle=\ttfamily, frame=single, columns=fullflexible, keepspaces=true, breaklines=true, showstringspaces=false, label={lst:tabelFoundation}, caption=Tabel pada foundation 6.]  
<table>
	<thead>
		<tr>
			<th width="200">Table Header</th>
			<th>Table Header</th>
			<th width="150">Table Header</th>
			<th width="150">Table Header</th>
		</tr>
	</thead>
	<tbody>
		<tr>
			<td>Content Goes Here</td>
			<td>This is longer content Donec id elit non mi porta gravida at eget metus.</td>
			<td>Content Goes Here</td>
			<td>Content Goes Here</td>
		</tr>
		<tr>
			<td>Content Goes Here</td>
			<td>This is longer Content Goes Here Donec id elit non mi porta gravida at eget metus.</td>
			<td>Content Goes Here</td>
			<td>Content Goes Here</td>
		</tr>
		<tr>
			<td>Content Goes Here</td>
			<td>This is longer Content Goes Here Donec id elit non mi porta gravida at eget metus.</td>
			<td>Content Goes Here</td>
			<td>Content Goes Here</td>
		</tr>
	</tbody>
</table>
\end{lstlisting}
\noindent Hasil dari kode diatas terlihat pada gambar ~\ref{fig:tableFoundation}:
\begin{figure} [H]
	\centering  
	\includegraphics[scale=0.7]{basictable_zurb.png}  
	\caption{Basic Table pada Foundation}
	\label{fig:tableFoundation}
\end{figure}

\subsubsection{Stacked Table}
\texttt{Stacked Table} diaplikasikan menggunakan kelas \texttt{.stack} untuk membuat tumpukan tabel pada layar kecil.


\noindent Berikut penggunaan kelas yang dicantumkan pada kode ~\ref{lst:stackFoundation}:
\begin{lstlisting}[style=customhtml, language=HTML,  basicstyle=\ttfamily, frame=single, columns=fullflexible, keepspaces=true, breaklines=true, showstringspaces=false, label={lst:stackFoundation}, caption=Tabel stack pada foundation 6.] 
<table class="stack"></table>
\end{lstlisting}
\noindent Hasil dari penggunaan kelas diatas terlihat pada gambar ~\ref{fig:stackTableFoundation}:
\begin{figure} [H]
	\centering  
	\includegraphics[scale=0.7]{stacktable_zurb.png}  
	\caption{Table Stack pada Foundation}
	\label{fig:stackTableFoundation}
\end{figure}

\subsubsection{Scrolling Table}
Apabila suatu tabel memiliki banyak data sehingga data yang ditampilkan memenuhi layar secara horizontal, maka dapat menggunakan kelas \texttt{.table-scroll}.

\noindent Berikut penggunaan kelas yang dicantumkan pada kode ~\ref{lst:scrollTableFoundation}:
\begin{lstlisting}[style=customhtml, language=HTML,  basicstyle=\ttfamily, frame=single, columns=fullflexible, keepspaces=true, breaklines=true, showstringspaces=false, label={lst:scrollTableFoundation}, caption=Scroll tabel pada foundation 6.] 
<table class="stack"></table>
\end{lstlisting}
\noindent Hasil dari penggunaan kelas diatas terlihat pada gambar ~\ref{fig:scrollTableFoundation}:
\begin{figure} [H]
	\centering  
	\includegraphics[scale=0.7]{tablescrolling_zurb.png}  
	\caption{Table Scrolling pada Foundation}
	\label{fig:scrollTableFoundation}
\end{figure}


\subsubsection{Forms}
\texttt{Forms} pada Foundation dibuat dengan kombinasi standar dari elemen \texttt{<form>}, serta \textit{grid rows} dan \textit{columns} atau \textit{cells}. 

\subsubsection{Text Inputs}
Kode berikut ini akan membuat sebuah \textit{text field} yang bisa diterapkan untuk \textit{field} : \texttt{text, date, datetime, datetime-local, email, month, number, password, search, tel, time, url,} dan \texttt{week}.\\

\noindent Berikut penggunaan kelas yang dicantumkan pada kode ~\ref{lst:textinputsFoundation}:

\begin{lstlisting}[style=customhtml, language=HTML,  basicstyle=\ttfamily, frame=single, columns=fullflexible, keepspaces=true, breaklines=true, showstringspaces=false, label={lst:textinputsFoundation}, caption=Text inputs pada foundation 6.] 
<form>
	<div class="grid-container">
		<div class="grid-x grid-padding-x">
			<div class="medium-6 cell">
				<label>Input Label
					<input type="text" placeholder=".medium-6.cell">
				</label>
			</div>
			<div class="medium-6 cell">
				<label>Input Label
					<input type="text" placeholder=".medium-6.cell">
				</label>
			</div>
		</div>
	</div>
</form>
\end{lstlisting}
\noindent Hasil dari kode diatas terlihat pada gambar ~\ref{fig:textinputZurb}:
\begin{figure} [H]
	\centering  
	\includegraphics[scale=0.7]{input_zurb.png}  
	\caption{Text Input pada Foundation}
	\label{fig:textinputZurb} 
\end{figure}

\subsubsection{Select Menus}
Penggunaan \texttt{select menus} digunakan apabila \textit{programmer} menginginkan banyak pilihan data dalam satu menu.\\
\noindent Berikut penggunaan kelas yang dicantumkan pada kode ~\ref{lst:selectMenusFoundation}:
\begin{lstlisting}[style=customhtml, language=HTML,  basicstyle=\ttfamily, frame=single, columns=fullflexible, keepspaces=true, breaklines=true, showstringspaces=false, label={lst:selectMenusFoundation}, caption=Select menus pada foundation 6.] 
<label>Select Menu
	<select>
		<option value="husker">Husker</option>
		<option value="starbuck">Starbuck</option>
		<option value="hotdog">Hot Dog</option>
		<option value="apollo">Apollo</option>
	</select>
</label>
\end{lstlisting}
\noindent Hasil dari kode diatas terlihat pada gambar ~\ref{fig:selectmenuFoundation}:
\begin{figure} [H]
	\centering  
	\includegraphics[scale=0.4]{selectmenu_zurb.png}  
	\caption{Select Menu pada Foundation}
	\label{fig:selectmenuFoundation}
\end{figure}

\subsubsection{Inline Labels and Buttons}
Untuk melampirkan teks tambahan atau kontrol ke kiri atau kanan bidang input, elemen akan berada didalam \texttt{.input-grup}, kelas-kelas ini akan ditambahkan ke elemen:
\begin{itemize}
	\item \texttt{.input-grup-field} pada bidang teks.
	\item \texttt{.input-grup-label} pada label teks.
	\item \texttt{.input-grup-button} pada tombol. 
\end{itemize}

\noindent Berikut penggunaan kelas yang dicantumkan pada kode ~\ref{lst:inlineFoundation}:
\begin{lstlisting}[style=customhtml, language=HTML,  basicstyle=\ttfamily, frame=single, columns=fullflexible, keepspaces=true, breaklines=true, showstringspaces=false, label={lst:inlineFoundation}, caption=Select menus pada foundation 6.] 
<div class="input-group">
	<span class="input-group-label">$</span>
	<input class="input-group-field" type="number">
	<div class="input-group-button">
		<input type="submit" class="button" value="Submit">
	</div>
</div>
\end{lstlisting}

\noindent Hasil dari kode diatas terlihat pada gambar ~\ref{fig:inlineFoundation}:
\begin{figure} [H]
	\centering  
	\includegraphics[scale=0.7]{inline_zurb.png}  
	\caption{Inline Label dan Button pada Foundation}
	\label{fig:inlineFoundation}
\end{figure}

\subsubsection{Tabs}
Tab semakin banyak digunakan dalam desain web karena programmer dapat menyajikan konten secara seragam. Ini memungkinkan untuk menyimpan banyak dokumen dalam satu \textit{window}. \textit{Programmer} dapat menggunakan tab sebagai widget navigasi untuk beralih antar konten sehingga menghasilkan tata letak yang sistematis dan bersih.\\

\noindent Berikut penggunaan kelas yang dicantumkan pada kode ~\ref{lst:tabsFoundation}:
\begin{lstlisting}[style=customhtml, language=HTML,  basicstyle=\ttfamily, frame=single, columns=fullflexible, keepspaces=true, breaklines=true, showstringspaces=false, label={lst:tabsFoundation}, caption=Tabs pada foundation 6.] 
<ul class="tabs" data-tabs id="tab_component">
	<li class="tabs-title"><a href="#pub1">Section 1</a></li>
	<li class="tabs-title is-active"><a href="#pub2">Section 2</a></li>
	<li class="tabs-title"><a href="#pub3">Section 3</a></li>
	<li class="tabs-title"><a href="#pub4">Section 4</a></li>
</ul>
<div class="tabs-content" data-tabs-content="tab_component">
	<div class="tabs-panel" id="pub1">
		<p>Far far away, behind the word mountains, far from the countries Vokalia and Consonantia, there live the blind texts.</p>
	</div>
	<div class="tabs-panel is-active" id="pub2">
		<p> Separated they live in Bookmarksgrove right at the coast of the	Semantics, a large language ocean. </p>
	</div>
	<div class="tabs-panel" id="pub3">
		<p>A small river named Duden flows by their place and supplies it with the necessary regelialia.</p>
	</div>
	<div class="tabs-panel" id="pub4">
		<p>It is a paradisematic country, in which roasted parts of sentences fly into your mouth. </p>
	</div>
</div>
\end{lstlisting}

\noindent Hasil dari kode diatas terlihat pada gambar ~\ref{fig:tabsFoundation}:
\begin{figure} [H]
	\centering  
	\includegraphics[scale=0.9]{tabs_component_zurb.png}  
	\caption{Tabs pada Foundation 6}
	\label{fig:tabsFoundation}
\end{figure}

\subsubsection{Reveal}
Modal hanyalah wadah kosong, sehingga \textit{developer} dapat menaruh segala jenis konten di dalamnya, seperti teks ke formulir hingga video ke seluruh \textit{grid}.
Untuk membuat modal, tambahkan kelas \texttt{.reveal}, atribut \texttt{data-reveal}, dan ID yang unik ke dalam \textit{container}.
\\
\noindent Berikut implementasi modal yang tertera pada kode ~\ref{lst:revealFoundation}:

\begin{lstlisting}[style=customhtml, language=HTML,  basicstyle=\ttfamily, frame=single, columns=fullflexible, keepspaces=true, breaklines=true, showstringspaces=false, label={lst:revealFoundation}, caption=Reveal pada foundation 6.]  
<div class="reveal" id="exampleModal1" data-reveal>
	<h1>Awesome. I Have It.</h1>
	<p class="lead">Your couch. It is mine.</p>
	<p>I'm a cool paragraph that lives inside of an even cooler modal. Wins!</p>
	<button class="close-button" data-close aria-label="Close modal" type="button">
	<span aria-hidden="true">&times;</span>
	</button>
</div>
\end{lstlisting} 

\noindent Hasil dari kode diatas terlihat pada gambar ~\ref{fig:revealFoundation}:
\begin{figure} [H]
	\centering  
	\includegraphics[scale=0.9]{modal_zurb.png}  
	\caption{Reveal pada Foundation 6}
	\label{fig:revealFoundation}
\end{figure}

\section{Bootstrap 4}
Bootstrap adalah \textit{open source toolkit} yang dikembangkan dengan HTML, CSS, dan JS. Programmer dapat membangun seluruh aplikasi dengan sistem grid responsif, komponen prebuilt yang luas, dan plugin yang dibangun di jQuery. \cite{bootstrap:19}

\subsection{Struktur File}
Listing ~\ref{lst:strukturBootstrap} merupakan struktur paling dasar dari Bootstrap:
\begin{lstlisting}[basicstyle=\ttfamily, frame=single,
columns=fullflexible, keepspaces=true, breaklines=true, label={lst:strukturBootstrap}, caption=Struktur file pada bootstrap 4]
.bootstrap/
|-- css/
|   |-- bootstrap-grid.css 
|   |-- bootstrap-grid.css.map 
|   |-- bootstrap-grid.min.css
|   |-- bootstrap-grid.min.css.map
|   |-- bootstrap-reboot.css
|   |-- bootstrap-reboot.css.map
|   |-- bootstrap-reboot.min.css
|   |-- bootstrap-reboot.min.css.map
|   |-- bootstrap.css
|   |-- bootstrap.css.map
|   |-- bootstrap.min.css
|   |-- bootstrap.min.css.map
|-- js/
|   |-- bootstrap.bundle.js
|   |-- bootstrap.bundle.js.map
|   |-- bootstrap.bundle.min.js
|   |-- bootstrap.bundle.min.js.map
|   |-- bootstrap.js
|   |-- bootstrap.js.map
|   |-- bootstrap.min.js
|   |-- bootstrap.min.js.map
.
\end{lstlisting}

File yang dikompilasi terdiri dari: 
\begin{itemize}
	\item CSS dan JS yang dikompilasi (\texttt{bootstrap.*}).
	\item CSS dan JS yang dikompilasi dan diperkecil (\texttt{bootstrap.min.*}). \item Peta sumber (\texttt{bootstrap.*.map}) tersedia untuk digunakan dengan alat pengembang browser tertentu. 
	\item File JS yang dibundel (\texttt{bootstrap.bundle.js} dan \texttt{bootstrap.bundle.min.js} yang diperkecil) termasuk Popper, tetapi bukan jQuery.
\end{itemize}

\subsection{Sistem Grid}
Sistem grid Bootstrap menggunakan \textit{container}, \textit{rows}, dan \textit{columns} untuk tata letak dan penyelarasan konten. Selain itu sistem ini dibangun dengan \textit{flexbox} dan seluruhnya \textit{responsive}. \\

\noindent Dalam kode ~\ref{lst:gridBootstrap} akan dibuat tiga kolom yang memiliki lebar yang sama baik dalam \textit{device} \textit{small, medium, large} dan \textit{extra large} menggunakan kelas grid yang sudah ditentukan sebelumnya oleh Bootstrap. Penggunaan \verb|.container| akan membuat kolom berada ditengah halaman.
\begin{lstlisting}[style=customhtml, language=HTML,  basicstyle=\ttfamily, frame=single, columns=fullflexible, keepspaces=true, breaklines=true, showstringspaces=false, label={lst:gridBootstrap}, caption=Sistem grid pada bootstrap 4] 
<div class="container">
	<div class="row">
		<div class="col-sm">
			One of three columns
		</div>
		<div class="col-sm">
			One of three columns
		</div>
		<div class="col-sm">
			One of three columns
		</div>
	</div>
</div>
\end{lstlisting}
\noindent Hasil dari kode diatas terlihat pada gambar ~\ref{fig:gridBootstrap}:
\begin{figure} [H]
	\centering  
	\includegraphics[scale=0.7]{gridbasic_bootstrap.png}  
	\caption{Grid pada Bootstrap} 
	\label{fig:gridBootstrap}
\end{figure}

\subsubsection{Cara Kerja Grid Bootstrap}
\noindent Secara detil, bootstrap bekerja dengan cara:
\begin{itemize}
	\item \textit{Container} disediakan agar konten berada ditengah halaman dan mengisi konten tersebut secara horizontal. Penggunaan \verb|.container| untuk menentukan lebar pixel secara responsif atau \verb|.container-fluid| untuk membuat lebar: 100\%  di semua ukuran \textit{viewport} dan perangkat.
	\item Sebuah baris akan membungkus kolom - kolom. Setiap kolom akan memiliki \textit{padding} secara horizontal yang disebut \verb|gutter| untuk mengatur jarak antar kolom.
	\item Penggunaan flexbox akan membuat lebar pada kolom tidak perlu dispesifikasikan. Misalnya empat variabel dari \texttt{.col-sm} akan secara otomatis membuat lebar kolom sebesar 25\%.
	\item Kelas kolom menunjukkan jumlah kolom yang ingin digunakan, dengan maksimal 12 kolom per baris. Apabila \textit{developer} menginginkan tiga kolom yang memiliki lebar yang sama maka dapat menggunakan \texttt{.col-4}.
	\item Lebar kolom diatur dalam persentase, sehingga kolom akan memiliki lebar yang berubah-ubah dan ukuran bergantung dengan elemen \textit{parent} nya.
\end{itemize}
\subsubsection{Pilihan Grid}
Bootstrap menggunakan px untuk grid breakpoint dan lebar container. Ini dikarenakan lebar \textit{viewport} ditentukan denga satuan pixels.
Berikut ini tabel ~\ref{fig:choiceGridBootstrap} yang menjelaskan penggunaan kelas grid dalam berbagai perangkat:
\begin{figure} [H]
	\centering  
	\includegraphics[scale=0.7]{gridoption_bootstrap.png}  
	\caption{Pilihan kelas grid pada Bootstrap} 
	\label{fig:choiceGridBootstrap}
\end{figure}

\subsection{Konten}
\subsubsection{Tabel}
Bagian tabel akan menjelaskan penggunaan kelas tabel dengan menggunakan tiga kelas: kelas tabel secara default (\texttt{.table}), kelas tabel dengan garis batas (\texttt{table-bordered}) dan kelas tabel dengan warna baris berbeda (\texttt{table-striped}).
\subsubsection{Tabel Default}
Dengan penggunaan kelas \verb|.table| pada seluruh tag \texttt{<table>} maka \textit{style} pada bootstrap akan diterapkan, sehingga setiap tabel yang \textit{nested} akan diatur sesuai dengan \textit{parent} nya.\\
Gambar ~\ref{fig:tableBootstrap} merupakan hasil penggunaan tabel secara \textit{default}:
\begin{figure} [H]
	\centering  
	\includegraphics[scale=0.7]{tablebasic_bootstrap.png}  
	\caption{Tabel default pada Bootstrap} 
	\label{fig:tableBootstrap}
\end{figure}

\noindent Kode dari gambar diatas tertera pada kode ~\ref{lst:tableBootstrap}:
\begin{lstlisting}[style=customhtml, language=HTML,  basicstyle=\ttfamily, frame=single, columns=fullflexible, keepspaces=true, breaklines=true, showstringspaces=false, label={lst:tableBootstrap}, caption=Tabel pada bootstrap 4] 
<table class="table">
	<thead>
		<tr>
		<th scope="col">#</th>
		<th scope="col">First</th>
		<th scope="col">Last</th>
		<th scope="col">Handle</th>
		</tr>
	</thead>
	<tbody>
		<tr>
			<th scope="row">1</th>
			<td>Mark</td>
			<td>Otto</td>
			<td>@mdo</td>
		</tr>
		<tr>
			<th scope="row">2</th>
			<td>Jacob</td>
			<td>Thornton</td>
			<td>@fat</td>
		</tr>
		<tr>
			<th scope="row">3</th>
			<td>Larry</td>
			<td>the Bird</td>
			<td>@twitter</td>
		</tr>
	</tbody>
</table>
\end{lstlisting}

\subsubsection{Tabel dengan Garis Batas}
Penggunaan kelas \texttt{.table-bordered} akan membuat tabel memiliki garis batas untuk semua sisi didalam tabel dan \textit{cells}.\\
Gambar ~\ref{fig:tableBorderedBootstrap} merupakan hasil penggunaan tabel:
\begin{figure} [H]
	\centering  
	\includegraphics[scale=0.7]{tablebordered_bootstrap.png}  
	\caption{Tabel dengan Garis Batas pada Bootstrap} 
	\label{fig:tableBorderedBootstrap}
\end{figure}

\noindent Kode dari gambar diatas tertera pada kode ~\ref{lst:tableborderedBootstrap}:
\begin{lstlisting}[style=customhtml, language=HTML,  basicstyle=\ttfamily, frame=single, columns=fullflexible, keepspaces=true, breaklines=true, showstringspaces=false, label={lst:tableborderedBootstrap}, caption=Kode tabel dengan garis batas pada bootstrap 4] 
<table class="table table-bordered">
	<thead>
		<tr>
			<th scope="col">#</th>
			<th scope="col">First</th>
			<th scope="col">Last</th>
			<th scope="col">Handle</th>
		</tr>
	</thead>
	<tbody>
		<tr>
			<th scope="row">1</th>
			<td>Mark</td>
			<td>Otto</td>
			<td>@mdo</td>
		</tr>
		<tr>
			<th scope="row">2</th>
			<td>Jacob</td>
			<td>Thornton</td>
			<td>@fat</td>
		</tr>
		<tr>
			<th scope="row">3</th>
			<td colspan="2">Larry the Bird</td>
			<td>@twitter</td>
		</tr>
	</tbody>
</table>
\end{lstlisting}

\subsubsection{Tabel dengan Warna Baris Berbeda}
Penggunaan kelas \texttt{.table-striped} akan membuat tabel memiliki warna baris berbeda batas antara baris genap dan ganjil didalam tag \texttt{<tbody>}.
Gambar ~\ref{fig:tableStripedBootstrap} merupakan hasil penggunaan tabel:
\begin{figure} [H]
	\centering  
	\includegraphics[scale=0.7]{tablestriped_bootstrap.png}  
	\caption{Tabel dengan Warna Baris Berbeda pada Bootstrap} 
	\label{fig:tableStripedBootstrap}
\end{figure}

\begin{lstlisting}[style=customhtml, language=HTML,  basicstyle=\ttfamily, frame=single, columns=fullflexible, keepspaces=true, breaklines=true, showstringspaces=false, label={lst:tablestripedBootstrap}, caption=Kode tabel dengan warna baris berbeda pada bootstrap 4.] 
<table class="table table-striped">
	<thead>
		<tr>
			<th scope="col">#</th>
			<th scope="col">First</th>
			<th scope="col">Last</th>
			<th scope="col">Handle</th>
		</tr>
	</thead>
	<tbody>
		<tr>
			<th scope="row">1</th>
			<td>Mark</td>
			<td>Otto</td>
			<td>@mdo</td>
		</tr>
		<tr>
			<th scope="row">2</th>
			<td>Jacob</td>
			<td>Thornton</td>
			<td>@fat</td>
		</tr>
		<tr>
			<th scope="row">3</th>
			<td colspan="2">Larry the Bird</td>
			<td>@twitter</td>
		</tr>
	</tbody>
</table>
\end{lstlisting}



\subsection{Komponen}
\subsubsection{Formulir}
\textit{Form} pada Bootstrap menyediakan beragam tipe input sesuai dengan kebutuhan \textit{user}. Contohnya penggunaan kelas \texttt{email} untuk \textit{input} email atau \texttt{number} untuk input berupa angka.
\subsubsection{Form Controls}
Programmer dapat membuat form menggunakan kelas \texttt{.form-control}. Kelas ini terdiri dari beberapa tag seperti tag \texttt{<input>, <select>} dan \texttt{<textarea>}.\\
Kode ~\ref{lst:formControlsBootstrap} menjelaskan implementasi dari \textit{form controls}:
\begin{lstlisting}[style=customhtml, language=HTML,  basicstyle=\ttfamily, frame=single, columns=fullflexible, keepspaces=true, breaklines=true, showstringspaces=false, label={lst:formControlsBootstrap}, caption=Form controls pada bootstrap 4.]  
<form>
	<div class="form-group">
		<label for="exampleFormControlInput1">Email address</label>
		<input type="email" class="form-control" id="exampleFormControlInput1" placeholder="name@example.com">
	</div>
	<div class="form-group">
		<label for="exampleFormControlSelect1">Example select</label>
		<select class="form-control" id="exampleFormControlSelect1">
		<option>1</option>
		<option>2</option>
		<option>3</option>
		<option>4</option>
		<option>5</option>
		</select>
	</div>
	<div class="form-group">
		<label for="exampleFormControlSelect2">Example multiple select</label>
		<select multiple class="form-control" id="exampleFormControlSelect2">
		<option>1</option>
		<option>2</option>
		<option>3</option>
		<option>4</option>
		<option>5</option>
		</select>
	</div>
	<div class="form-group">
		<label for="exampleFormControlTextarea1">Example textarea</label>
		<textarea class="form-control" id="exampleFormControlTextarea1" rows="3"></textarea>
	</div>
</form>
\end{lstlisting}
Hasil dari kode diatas terlihat pada gambar ~\ref{fig:formBootstrap}
\begin{figure} [H]
	\centering  
	\includegraphics[scale=0.7]{formsbasic_bootstrap.png}  
	\caption{\textit{Forms Basic} pada Bootstrap} 
	\label{fig:formBootstrap}
\end{figure} 
\subsubsection{Column Sizing}
Bootstrap memungkinkan programmer untuk menempatkan sejumlah \texttt{.col} di dalam baris \texttt{.row} atau \texttt{.form} dengan lebar tertentu. Misalnya ada tiga buah kolom, kolom pertama memiliki lebar 7 dengan menggunakan kelas \texttt{.col-7} maka dua kolom sisanya akan memiliki lebar yang  memenuhi baris tersebut.
 
Kode ~\ref{lst:formControlsBootstrap} menjelaskan implementasi dari \textit{column sizing}:
\begin{lstlisting}[style=customhtml, language=HTML,  basicstyle=\ttfamily, frame=single, columns=fullflexible, keepspaces=true, breaklines=true, showstringspaces=false, label={lst:columnsizingBootstrap}, caption=Column sizing pada bootstrap 4.] 
<form>
	<div class="form-row">
		<div class="col-7">
			<input type="text" class="form-control" placeholder="City">
		</div>
		<div class="col">
			<input type="text" class="form-control" placeholder="State">
		</div>
		<div class="col">
			<input type="text" class="form-control" placeholder="Zip">
		</div>
	</div>
</form>
\end{lstlisting}
\noindent Hasil dari kode diatas terlihat pada gambar ~\ref{fig:columnSizingBootstrap}:
\begin{figure} [H]
	\centering  
	\includegraphics[scale=0.7]{columnsizing_bootstrap.png}  
	\caption{\textit{Column Sizing} pada Bootstrap} 
	\label{fig:columnSizingBootstrap}
\end{figure}
\subsubsection{Disabled Forms}
Penambahan atribut boolean \texttt{disabled} pada sebuah input membuat \textit{user} tidak bisa mengisi data pada \textit{field} tersebut. Untuk non-aktifkan seluruh \textit{field} pada sebuah kolom dapat menambahkan atribut \texttt{disabled} pada tag \texttt{<fieldset>}.

\noindent Kode ~\ref{lst:disabledFormsBootstrap} menjelaskan implementasi dari \textit{disabled forms}:
\begin{lstlisting}[style=customhtml, language=HTML,  basicstyle=\ttfamily, frame=single, columns=fullflexible, keepspaces=true, breaklines=true, showstringspaces=false, label={lst:disabledFormsBootstrap}, caption=Disabled forms pada bootstrap 4.]  
<form>
	<fieldset disabled>
	<div class="form-group">
		<label for="disabledTextInput">Disabled input</label>
		<input type="text" id="disabledTextInput" class="form-control"
		placeholder="Disabled input">
	</div>
	<div class="form-group">
		<label for="disabledSelect">Disabled select menu</label>
		<select id="disabledSelect" class="form-control">
		<option>Disabled select</option>
		</select>
	</div>
	<div class="form-check">
		<input class="form-check-input" type="checkbox" 
		id="disabledFieldsetCheck" disabled>
		<label class="form-check-label" for="disabledFieldsetCheck">
		Can't check this
		</label>
	</div>
	<button type="submit" class="btn btn-primary">Submit</button>
	</fieldset>
</form>
\end{lstlisting}

\noindent Hasil dari kode diatas terlihat pada gambar ~\ref{fig:columnSizingBootstrap}:
\begin{figure} [H]
	\centering  
	\includegraphics[scale=0.7]{disabledforms_bootstrap.png}  
	\caption{\textit{Disabled Basic} pada Bootstrap} 
	\label{fig:disabledFormBootstrap}
\end{figure}

\subsubsection{Button}
Bootstrap memasukan beberapa button dengan \textit{style} yang sudah didefinisikan sebelumnya, membuat setiap button akan memiliki makna nya sendiri.\\
\noindent Kode ~\ref{lst:buttonBootstrap} menjelaskan implementasi dari \textit{button}:

\begin{lstlisting}[style=customhtml, language=HTML,  basicstyle=\ttfamily, frame=single, columns=fullflexible, keepspaces=true, breaklines=true, showstringspaces=false, label={lst:buttonBootstrap}, caption=Button pada bootstrap 4.] 
<button type="button" class="btn btn-primary">Primary</button>
<button type="button" class="btn btn-secondary">Secondary</button>
<button type="button" class="btn btn-success">Success</button>
<button type="button" class="btn btn-danger">Danger</button>
<button type="button" class="btn btn-warning">Warning</button>
<button type="button" class="btn btn-info">Info</button>
<button type="button" class="btn btn-light">Light</button>
<button type="button" class="btn btn-dark">Dark</button>

<button type="button" class="btn btn-link">Link</button>
\end{lstlisting}

\noindent Hasil dari kode diatas terlihat pada gambar ~\ref{fig:buttonBootstrap}:
\begin{figure} [H]
	\centering  
	\includegraphics[scale=0.7]{buttons_bootstrap.png}  
	\caption{\textit{Button} pada Bootstrap} 
	\label{fig:buttonBootstrap}
\end{figure}

\subsubsection{Button with Dropdowns}
\noindent Kode ~\ref{lst:buttonDropdownBootstrap} menjelaskan implementasi dari \textit{button with dropdowns}:
\begin{lstlisting}[style=customhtml, language=HTML,  basicstyle=\ttfamily, frame=single, columns=fullflexible, keepspaces=true, breaklines=true, showstringspaces=false, label={lst:buttonDropdownBootstrap}, caption=Button dropdown pada bootstrap 4.] 
<div class="input-group mb-3">
	<div class="input-group-prepend">
		<button class="btn btn-outline-secondary dropdown-toggle" type="button" data-toggle="dropdown"
		aria-haspopup="true" aria-expanded="false">Dropdown</button>
		<div class="dropdown-menu">
			<a class="dropdown-item" href="#">Action</a>
			<a class="dropdown-item" href="#">Another action</a>
			<a class="dropdown-item" href="#">Something else here</a>
			<div role="separator" class="dropdown-divider"></div>
			<a class="dropdown-item" href="#">Separated link</a>
		</div>
	</div>
	<input type="text" class="form-control" aria-label="Text input with dropdown button">
</div>

<div class="input-group">
	<input type="text" class="form-control" aria-label="Text input with dropdown button">
	<div class="input-group-append">
	<button class="btn btn-outline-secondary dropdown-toggle" type="button" data-toggle="dropdown"
	aria-haspopup="true" aria-expanded="false">Dropdown</button>
		<div class="dropdown-menu">
			<a class="dropdown-item" href="#">Action</a>
			<a class="dropdown-item" href="#">Another action</a>
			<a class="dropdown-item" href="#">Something else here</a>
			<div role="separator" class="dropdown-divider"></div>
			<a class="dropdown-item" href="#">Separated link</a>
		</div>
	</div>
</div>
\end{lstlisting}

\noindent Hasil dari kode diatas terlihat pada gambar ~\ref{fig:dropdownBootstrap}:
\begin{figure} [H]
	\centering  
	\includegraphics[scale=1.0]{buttonsdropdown_bootstrap.PNG}  
	\caption{Tombol \textit{dropdown} pada Bootstrap} 
	\label{fig:dropdownBootstrap}
\end{figure}

\subsubsection{Badge}
Kelas \texttt{badge} dan \texttt{.badge-*} di dalam sebuah \texttt{<a>} akan memberikan badge yang dapat diberi atribut \textit{hover} dan \textit{focus}. Kode ~\ref{lst:badgeBootstrap} menjabarkan implementasi dari komponen \textit{badge}:
\begin{lstlisting}[style=customhtml, language=HTML,  basicstyle=\ttfamily, frame=single, columns=fullflexible, keepspaces=true, breaklines=true, showstringspaces=false, label={lst:badgeBootstrap}, caption=Badge pada bootstrap 4.] 
<a href="#" class="badge badge-primary">Primary</a>
<a href="#" class="badge badge-secondary">Secondary</a>
<a href="#" class="badge badge-success">Success</a>
<a href="#" class="badge badge-danger">Danger</a>
<a href="#" class="badge badge-warning">Warning</a>
<a href="#" class="badge badge-info">Info</a>
<a href="#" class="badge badge-light">Light</a>
<a href="#" class="badge badge-dark">Dark</a>
\end{lstlisting}
\noindent Hasil dari kode diatas terlihat pada gambar ~\ref{fig:badgeBootstrap}:
\begin{figure} [H]
	\centering  
	\includegraphics[scale=1.0]{badge_bootstrap.PNG}  
	\caption{Badge pada Bootstrap} 
	\label{fig:badgeBootstrap}
\end{figure}

\subsubsection{Card}
\texttt{Card} adalah kontainer konten yang fleksibel dan bisa diatur lebarnya. Sebuah card memiliki sebuah \textit{headers} dan \textit{footers}. \\
Berikut ini implementasi yang terlihat pada kode ~\ref{lst:cardBootstrap}:
\begin{lstlisting}[style=customhtml, language=HTML,  basicstyle=\ttfamily, frame=single, columns=fullflexible, keepspaces=true, breaklines=true, showstringspaces=false, label={lst:cardBootstrap}, caption=Card pada bootstrap 4.] 
<div class="card">
	<div class="card-header">
		Featured
	</div>
	<div class="card-body">
		<h5 class="card-title">Special title treatment</h5>
		<p class="card-text">With supporting text below as 
		a natural lead-in to additional content.</p>
		<a href="#" class="btn btn-primary"card>Go somewhere</a>
	</div>
</div>
\end{lstlisting}
\noindent Hasil dari kode diatas terlihat pada gambar ~\ref{fig:cardBootstrap}:
\begin{figure} [H]
	\centering  
	\includegraphics[scale=1.0]{card_bootstrap.PNG}  
	\caption{Card pada Bootstrap} 
	\label{fig:cardBootstrap}
\end{figure}

\subsubsection{Navigation Bar}
Navbar pada Bootstrap terdiri dari beberapa sub-komponen yang bisa digunakan sesuai dengan kebutuhan:
\begin{itemize}
	\item \texttt{.navbar-brand} : Komponen untuk menampilkan nama perusahaan, nama produk atau nama proyek.
	\item \texttt{.navbar-nav} : Komponen untuk membuat navigasi memiliki lebar yang memenuhi layar.
	\item \texttt{.navbar-toggler} : Komponen yang digunakan bersamaan dengan plugin untuk membuat efek jatuh dan perilaku navigasi lainnya.
	\item \texttt{.form-inline} : Komponen untuk pengaturan formulir dan aksi.
	\item \texttt{.collapse.navbar-collapse} : Komponen untuk mengelompokkan dan menyembunyikan \textit{navigation bar} dengan sebuah breakpoint induknya.
	%breakpoint : titik dimana terjadi perubahan (xs(br:0px), sm.. )
\end{itemize}

Berikut ini implementasi yang terlihat pada kode ~\ref{lst:navBarBootstrap}, dimana semua sub-komponen yang termasuk dalam navigation bar. Navbar mengimplementasikan tema \texttt{light-themed} yang secara otomatis menyembunyikan menu pada breakpoint \texttt{lg}

\begin{lstlisting}[style=customhtml, language=HTML,  basicstyle=\ttfamily, frame=single, columns=fullflexible, keepspaces=true, breaklines=true, showstringspaces=false, label={lst:navBarBootstrap}, caption=Navigation bar pada bootstrap 4.]  
<nav class="navbar navbar-expand-lg navbar-light bg-light">
	<a class="navbar-brand" href="#">Navbar</a>
	<button class="navbar-toggler" type="button" data-toggle="collapse" data-target="#navbarSupportedContent" 
	aria-controls="navbarSupportedContent" aria-expanded="false" aria-label="Toggle navigation">
		<span class="navbar-toggler-icon"></span>
	</button>
	
	<div class="collapse navbar-collapse" id="navbarSupportedContent">
		<ul class="navbar-nav mr-auto">
			<li class="nav-item active">
				<a class="nav-link" href="#">Home <span class="sr-only">(current)</span></a>
			</li>
			<li class="nav-item">
				<a class="nav-link" href="#">Link</a>
			</li>
			<li class="nav-item dropdown">
				<a class="nav-link dropdown-toggle" href="#" id="navbarDropdown" role="button" data-toggle="dropdown" aria-haspopup="true" aria-expanded="false">
					Dropdown
				</a>
				<div class="dropdown-menu" aria-labelledby="navbarDropdown">
					<a class="dropdown-item" href="#">Action</a>
					<a class="dropdown-item" href="#">Another action</a>
					<div class="dropdown-divider"></div>
					<a class="dropdown-item" href="#">Something else here</a>
				</div>
			</li>
			<li class="nav-item">
				<a class="nav-link disabled" href="#">Disabled</a>
			</li>
		</ul>
		<form class="form-inline my-2 my-lg-0">
			<input class="form-control mr-sm-2" type="search" placeholder="Search" aria-label="Search">
			<button class="btn btn-outline-success my-2 my-sm-0" type="submit">Search</button>
		</form>
	</div>
</nav>
\end{lstlisting}

\noindent Hasil dari kode diatas terlihat pada gambar ~\ref{fig:navBarBootstrap}:
\begin{figure} [H]
	\centering  
	\includegraphics[scale=1.0]{navbar_bootstrap.PNG}  
	\caption{Navigation Bar pada Bootstrap} 
	\label{fig:navBarBootstrap}
\end{figure}

\subsubsection{Modal}
Bagaimana Modal bekerja :
\begin{itemize}
	\item Modal dibangun dengan HTML, CSS dan Javascript. 
	\item Menekan modal "backdrop" otomatis menutup komponen modal.
	\item Bootstrap hanya mendukung satu modal dalam sebuah window pada satu waktu. Penggunaan modal yang bercabang dalam Bootstrap dipercaya memberikan \textit{user experience} yang buruk.
	\item Modal menggunakann \texttt{position: fixed} yang diletakkan pada posisi teratas dalam kode agar terhindar dari \textit{bug} yang disebabkan elemen lain yang memiliki posisi \textit{fixed}. 
\end{itemize}
Komponen modal terdiri dari modal headerm modal body dan modal footer (opsional). Berikut ini implementasi yang terlihat pada kode ~\ref{lst:modalBootstrap}:

\begin{lstlisting}[style=customhtml, language=HTML,  basicstyle=\ttfamily, frame=single, columns=fullflexible, keepspaces=true, breaklines=true, showstringspaces=false, label={lst:modalBootstrap}, caption=Modal bar pada bootstrap 4.] 
<!-- Button trigger modal -->
<button type="button" class="btn btn-primary" data-toggle="modal" data-target="#myModal">
	Launch demo modal
</button>

<!-- Modal -->
<div class="modal fade" id="exampleModal" tabindex="-1" role="dialog" 
aria-labelledby="exampleModalLabel" aria-hidden="true">
	<div class="modal-dialog" role="document">
		<div class="modal-content">
			<div class="modal-header">
			<h5 class="modal-title" id="exampleModalLabel">Modal title</h5>
			<button type="button" class="close" data-dismiss="modal" aria-label="Close">
				<span aria-hidden="true">&times;</span>
			</button>
			</div>
			<div class="modal-body">
			...
			</div>
			<div class="modal-footer">
				<button type="button" class="btn btn-secondary" data-dismiss="modal">Close</button>
				<button type="button" class="btn btn-primary">Save changes</button>
			</div>
		</div>
	</div>
</div>
\end{lstlisting}

\noindent Hasil dari kode diatas terlihat pada gambar ~\ref{fig:modalBootstrap}:
\begin{figure} [H]
	\centering  
	\includegraphics[scale=0.5]{livemodal_bootstrap.png}  
	\caption{Modal pada Bootstrap} 
	\label{fig:modalBootstrap}
\end{figure}

\subsubsection{Ikon}
Bootstrap tidak memiliki \textit{library} ikon secara \textit{default}, sehingga ikon yang digunakan diambil dari \texttt{Font Awesome}. Penggunaan ikon dengan menggunakan tag \texttt{<i>} yang disertai dengan kelas \texttt{fa} (font-awesome). Berikut ini implementasi yang terlihat pada kode ~\ref{lst:ikonBootstrap}:

\begin{lstlisting}[style=customhtml, language=HTML,  basicstyle=\ttfamily, frame=single, columns=fullflexible, keepspaces=true, breaklines=true, showstringspaces=false, label={lst:ikonBootstrap}, caption=Ikon pada bootstrap 4.]
<i class="fa fa-coffee"></i>
\end{lstlisting}

\noindent Hasil dari kode diatas terlihat pada gambar ~\ref{fig:fontAwesomeBootstrap}:
\begin{figure} [H]
	\centering  
	\includegraphics[scale=1.0]{fa_coffee.PNG}  
	\caption{Ikon \textit{Coffee} pada Font Awesome} 
	\label{fig:fontAwesomeBootstrap}
\end{figure}

\subsubsection{Alert}
Alert menyediakan pesan umpan balik untuk user untuk berbagai tipe pesan peringatan yang tersedia. Untuk gaya yang sesuai \textit{developer} dapat menggunakan delapan kelas yang tersedia. Berikut ini implementasi yang terlihat pada kode ~\ref{lst:alertBootstrap}:

\begin{lstlisting}[style=customhtml, language=HTML,  basicstyle=\ttfamily, frame=single, columns=fullflexible, keepspaces=true, breaklines=true, showstringspaces=false, label={lst:alertBootstrap}, caption=Alert pada bootstrap 4.] 
<div class="alert alert-primary" role="alert">
	This is a primary alert—check it out!
</div>
<div class="alert alert-secondary" role="alert">
	This is a secondary alert—check it out!
</div>
<div class="alert alert-success" role="alert">
	This is a success alert—check it out!
</div>
<div class="alert alert-danger" role="alert">
	This is a danger alert—check it out!
</div>
<div class="alert alert-warning" role="alert">
	This is a warning alert—check it out!
</div>
<div class="alert alert-info" role="alert">
	This is a info alert—check it out!
</div>
<div class="alert alert-light" role="alert">
	This is a light alert—check it out!
</div>
<div class="alert alert-dark" role="alert">
	This is a dark alert—check it out!
</div>
\end{lstlisting}
\noindent Hasil dari kode diatas terlihat pada gambar ~\ref{fig:alertBootstrap}:
\begin{figure} [H]
	\centering  
	\includegraphics[scale=1.0]{alert_bootstrap.PNG}  
	\caption{Alert pada Bootstrap} 
	\label{fig:alertBootstrap}
\end{figure}

\section{DateTimePicker}
\texttt{DateTimePicker} dengan menggunakan jQuery untuk memilih tanggal dan waktu pada bagian forms. \cite{datetimepicker}
\subsubsection{Inline DateTimePicker}
Penggunaan plugin ini, memungkinkan \textit{users} untuk memilih tanggal dan waktu secara bersamaan.

Penggunaan nya dalam kode HTML tertera pada kode ~\ref{lst:htmlPlugin}:
\begin{lstlisting}[style=customhtml, language=HTML,  basicstyle=\ttfamily, frame=single, columns=fullflexible, keepspaces=true, breaklines=true, showstringspaces=false, label={lst:htmlPlugin}, caption=Kode HTML pada plugin.] 
<input id="datetimepicker" type="text" >
\end{lstlisting}

\noindent Hasil dari kode diatas terlihat pada gambar ~\ref{fig:datetimepickerBootstrap}:
\begin{figure} [H]
	\centering  
	\includegraphics[scale=0.8]{datetimepicker_bootstrap4.PNG}  
	\caption{Datetimepicker pada Bootstrap} 
	\label{fig:datetimepickerBootstrap}
\end{figure}

Kemudian penggunaan dalam kode Javascript tertera pada kode ~\ref{lst:jQueryPlugin}:

\begin{lstlisting}[style=JavaScript, language=JavaScript,  basicstyle=\ttfamily, frame=single, columns=fullflexible, keepspaces=true, breaklines=true, showstringspaces=false, label={lst:jQueryPlugin}, caption=Kode Inline DateTimePicker di jQuery.] 
jQuery('#datetimepicker').datetimepicker();
\end{lstlisting}

\section{Prinsip Desain Grafik pada Website}
Pembangunan antarmuka desain perlu memperhatikan prinsip desain grafik untuk menunjang kenyamanan pengguna ketika mengakses sebuah website. Desain Grafik adalah kombinasi antara gambar, angka, grafik foto dan ilustrasi yang membutuhkan analisis seseorang, sehingga mampu menginformasikan pesan secara efektif. \cite{graphicdesign}  \\
\noindent Berikut ini enam prinsip desain grafik yang dapat digunakan untuk merancang sebuah website:
\subsection{Metaphor (Metafora)}
Metafora berarti mampu menampilkan dan menggambarkan elemen-elemen(titik, garis, warna dan ruang) yang \textit{relevant} dan dapat dikenali. Pada gambar ~\ref{fig:metaphor} adalah contoh hasil antarmuka untuk aplikasi penjualan di supermarket dengan meniru bagaiamana konsumen berjalan mengelilingi toko dengan membawa kereta dorong. 
\begin{figure} [H]
	\centering  
	\includegraphics[scale=0.4]{website/metaphor.png}  
	\caption{Implementasi prinsip metafora pada tampilan aplikasi penjualan di supermarket.}
	\label{fig:metaphor}	 
\end{figure}

\subsection{Clarity (Kejelasan)}
Setiap elemen dalam antarmuka harus memiliki alasan yang masuk akal mengapa digunakan. Selain itu website dapat memanfaatkan \textit{white space} atau ruang kosong antar elemen sehingga mata tidak terlalu lelah melihat desain yang terlalu padat. 
\begin{figure} [H]
	\centering  
	\includegraphics[scale=0.4]{website/clarity.png}  
	\caption{Implementasi prinsip kejelasan pada sebuah website.}
	\label{fig:clarity}	 
\end{figure}
\noindent Gambar ~\ref{fig:clarity} menampilkan contoh website yang memanfaatkan \textit{white space} sehingga menampilkan website yang minimalis.

\subsection{Consistency (Ketetapan)}
Konsistensi dalam tampilan, pewarnaan, gambar, ikon, dan typography seperti \textit{form} yang digunakan dalam website atau keseragaman antara \textit{home page} dan \textit{content page} dalam website.
\begin{figure} [H]
	\centering  
	\includegraphics[scale=0.8]{website/consistency.png}  
	\caption{Implementasi prinsip \textit{consistency} pada sebuah website.}
	\label{fig:consistency}	 
\end{figure}
\noindent Gambar ~\ref{fig:consistency} menampilkan contoh website yang konsisten meletakkan judul, gambar dan isi konten di tempat yang sama baik pada \textit{home page} dan \textit{content page}.

\subsection{Alignment (Perataan)}
\textit{Alignment} adalah penempatan elemen-elemen visual sehingga dapat berbaris dalam suatu komposisi. Dalam desain digunakan perataan dengan memanfaatkan \textit{grid} untuk mengatur elemen dan mengelompokkan elemen dalam kelompok sehingga menciptakan keseimbangan dan adanya koneksi antar elemen. Gambar ~\ref{fig:alignment} menampilkan pemanfaatan grid dalam sebuah windows.
\begin{figure} [H]
	\centering  
	\includegraphics[scale=0.7]{website/alignment.png}  
	\caption{Implementasi prinsip \textit{alignment} pada sebuah website.}
	\label{fig:alignment}	 
\end{figure}
\noindent Penggunaan grid pada gambar tersebut dapat meratakan konteks yang berkaitan seperti kolom "File Name" dan "Save as type" yang memiliki ukuran yang sama. 

\subsection{Proximity}
Item-item yang berkaitan ditampilkan secara berkelompok sedangkan yang tidak berkaitan ditampilkan dengan diberi jarak yang lebih jauh. Gambar ~\ref{fig:proximity} merupakan contoh penerapan prinsip \textit{proximity} dalam sebuah formulir.
\begin{figure} [H]
	\centering  
	\includegraphics[scale=0.7]{website/proximity.png}  
	\caption{Implementasi prinsip \textit{proximity} pada sebuah website.}
	\label{fig:proximity}	 
\end{figure}
\noindent Formulir yang berada ditengah meunjukkan penggunaan prinsip \textit{proximity} yang baik dengan mengelompokkan elemen yang berkaitan dalam sebuah \textit{border}.

\subsection{Contrast (Kontras)}
Prinsip desain ini memungkinkan untuk menempatkan dua unsur desain paling bertentangan antara satu dengan lainnya. Contohnya seperti gelap dan terang atau hitam dan putih. Penggunaan jenis ini akan memudahkan seseorang untuk melihat sebuah desain untuk langsung tertuju arah pandangannya ke bagian paling penting dari desain tersebut. 
\begin{figure} [H]
	\centering  
	\includegraphics[scale=0.4]{website/contrast.png}  
	\caption{Implementasi prinsip \textit{contrast} pada sebuah website.}
	\label{fig:contrast}	 
\end{figure}
\noindent Gambar ~\ref{fig:contrast} menunjukkan pemanfaatan kontras pada form utama website dengan memberi warna kuning pada \textit{background} formulirnya, sehingga mata akan tertuju langsung ke elemen tersebut.
