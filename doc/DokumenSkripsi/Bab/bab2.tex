%versi 2 (8-10-2016)
\chapter{Landasan Teori}
\label{chap:teori}

\section{BlueTape}
\label{sec:bluetape}
Bluetape merupakan aplikasi berbasis web, berguna sebagai aplikasi yang menunjang proses administrasi dalam lingkungan FTIS UNPAR. Web ini dapat diakses pada \url{http://www.bluetape.azurewebsites.net}.

\subsection{Login}
Halaman utama aplikasi BlueTape akan mengarahkan \textit{user} untuk \textit{login} dengan menggunakan Google, user akan login dengan melihat beberapa kondisi ini:
\begin{itemize}
\item Apabila \textit{user} belum pernah login menggunakan akun UNPAR(xxx@student.unpar.ac.id atau yyy@unpar.ac.id) maka  \textit{user} akan diminta untuk memasukan email UNPAR dan password
\item Apabila user sudah pernah login menggunakan akun UNPAR, maka \textit{user} akan diminta untuk memilih akun beserta password.
\item User akan terhubung otomatis dengan akun @gmail.com. Apabila BlueTape menolak autentikasi user maka: User akan diminta untuk buka halaman Gmail lalu klik avatar di kanan atas dan memilih akun UNPAR yang tepat pada \textit{button} "Add Account"
\end{itemize}
User akan melihat beberapa menu sesuai dengan \textit{role} user, sebagau mahasiswa, staf TU, dll.

\subsection{Dosen}
\subsubsection{Perubahan Kuliah}
Modul perubahan kuliah berguna untuk mengirimkan permintaan perubahan mata kuliah yang dikirim oleh dosen kepada staf Tata Usaha. Kolom - kolom yang terdapat dalam modul ini:
\begin{itemize}
\item Kode MK (Mata Kuliah)
\item Nama Mata Kuliah
\item Kelas
\item Jenis perubahan (diganti / tambahan / ditiadakan)
\item Dari (hari/jam dan ruang) dan ke(hari/jam dan tempat)
\item Keterangan
\end{itemize}
Apabila ada kolom yang belum dapat diisi(contoh : dosen belum tahu tempat kelas pengganti) maka kolom kelas dapat dikosongkan.
Dosen juga dapat membuat lebih dari 1 kelas pengganti, dengan mengklik tombol "Tambah Pertemuan Ekstra".
Setelah dosen klik "Kirim Permohonan", maka sistem akan mengirim permohonan ke halaman BlueTape bagian Tata Usaha utnuk diperiksa, disetujui, dan dicetak sebagai pengumuman. Jika staf Tata Usaha telah selesai mengkonfirmasi(atau menolak), maka dosen akan mendapatkan e-mail notifikasi.

\subsection{Dosen Informatika}
\subsubsection{Entri Jadwal Dosen}
Dosen informatika dapat menggunakan menu ini untuk mengisikan jadwal mingguan. Hasil dari pengisian jadwal dapat diekspor ke XLS, atau dapat dilihat oleh mahasiswa informatika melalui portal BlueTape.
\subsubsection{Tambah Jadwal}
Pada bagian entri jadwal, dosen informatika dapat mengisikan hari, jam mulai, durasi, label, dan sejenisnya. Berikut ini jenis yang dapat dipilih:
\begin{itemize}
\item Konsultasi : Waktu yang dosen siapkan untuk konsultasi mahasiswa. Pada tabel akan diberi \textit{background} berwaena kuning.
\item Terjadwal: Kegiatan mingguan dosen informatika yang telah terjadwal. Contoh : rapat jurusan
\item Kelas : Kelas kuliah maupun praktikum.
\end{itemize}
Lalu dosen dapat klik tombol "Tambah" untuk menambahkan

\subsubsection{Ubah/Hapus Jadwal}
Dosen dapat mengubah atau menghapus jadwal yang tertera pada tabel. Lalu Pop up window akan terbuka dengan pilihan-pilihan yang sesuai dengan permintaan dosen.

\subsubsection{Hapus Semua}
Tombol "Delete All" dapat digunakan untuk menghapus secara cepat seluruh jadwal yang telah dosen buat sebelumnya. Penggunaan tombol ini biasa nya digunakan pada awal semester, dimana jadwal yang dosen miliki berubah seluruhnya.

\subsubsection{Ekspor ke XLS}
Tombol "Ekspor ke XLS" berfungsi untuk membuat file XLS untuk jadwal dosen.

\subsection{Mahasiswa}
\subsubsection{Cetak Transkrip}
Mahasiswa dapat menggunakan menu ini untuk mengirimkan permohonan cetak transkrip
Mahasiswa mengirimkan permohonan pencetakan transkrip dengan mengisi kolom-kolom pada formulir "Permohonan Baru".
Mahasiswa hanya dapat mengirimkan permohonan:
\begin{itemize}
\item Maksimal 1x dalam satu semester (kecuali permohonan ditolak)
\item Jika ada permohonan yang belum dijawab.
\end{itemize}

\subsubsection{Mahasiswa Informatika}
\subsubsection{Lihat Jadwal Dosen}
Mahasiswa dapat melihat jadwal mingguan seluruh dosen dengan memilih nama dosen pada seleksi tab, dan tabel jadwal dosen akan ditampilkan pada bagian bawah halaman. Selain itu tabel juga berisi informasi tanggal terakhir dosen meng-\textit{update} jadwal sehingga mahasiswa dapat melihat apakah jadwal tersebut merupakan jadwal semester ini atau semester lalu. 
Lalu terdapat tombol "Ekspor ke XLS" pada halaman lihat jadwal dosen, sehingga mahasiswa dapat menyimpan atau mencetak jadwal tersebut. 

\subsection{Staf Tata Usaha}
\subsubsection{Manajemen Perubahan Kuliah}
Staf Tata Usaha dapat melakukan manajemen permintaan perubahan kuliah. Sebuah tabel akan menampilkan daftar permohonan dengan menampilkan tanggal kapan permohonan dibuat.
Setiap daftar permohonan akan memiliki beberapa tombol :
\begin{itemize}
\item \includegraphics[height=0.7\baselineskip]{tombol_eye.png} berfungsi untuk melihat detail permohonan sehingga dapat menentukan apakah permohonan disetujui atau tidak.
\item \includegraphics[height=0.7\baselineskip]{tombol_print.png} berfungsi untuk membuka pop-up print-out pengumuman.
\item \includegraphics[height=0.7\baselineskip]{tombol_good.png} berfungsi sebagai konfirmasi bahwa pengumuman telah dicetak dan disebarkan.
\item \includegraphics[height=0.7\baselineskip]{tombol_bad.png} berfungsi untuk menyatakan bahwa permohonan ditolak. Staf Tata Usaha akan mengisi alasan mengapa permohonan ditolak sehingga tidak membingungkan pemohon.
\item \includegraphics[height=0.7\baselineskip]{tombol_trash.png} berfungsi untuk menghapus permohonan \textbf{secara permanen}. Staf Tata Usaha dihimbau agar tidak menggunakan tombol ini kecuali dalam keadaan terpaksa.
\end{itemize}

\subsubsection{Manajemen Cetak Transkrip}
Staf Tata Usaha dapat melihat daftar perminttan transkrip dalam bentuk tabel. Keterangan mengenai transkrip dapat dilihat menggunakan tombol \includegraphics[height=0.6\baselineskip]{tombol_eye.png} (detail). Selain itu terdapat dua pilihan jawaban dalam setiap daftar permintaan yaitu \includegraphics[height=0.7\baselineskip]{tombol_bad.png} (tolak) dan 
\includegraphics[height=0.7\baselineskip]{tombol_print.png} (cetak). Masing-masing tombol memerlukan keterangan tambahan mengeai alasan mengapa transkrip dapat dicetak maupun ditolak.

Modul ini berguna untuk manajemen permohonan cetka transkrip. Terdapat sebuah tabel yang menapilkan daftar pemohonan dengan tanggal yang terurut. Staf Tata Usaha dapat mencari daftar permintaan berdasarkan NPM pemohon.

Beberapa tombol yang tersedia untuk setiap permohonan :

\begin{itemize}
\item \includegraphics[height=0.7\baselineskip]{tombol_eye.png} berfungsi untuk melihat detail permohonan sehingga dapat menentukan apakah permohonan disetujui atau tidak.
\item \includegraphics[height=0.7\baselineskip]{tombol_print.png} berfungsi untuk membuka pop-up print-out pengumuman. Dalam pop-up akan disediakan sebuah link menuju halaman percetakan transkrip pada SIAkad.
\item \includegraphics[height=0.7\baselineskip]{tombol_bad.png} berfungsi untuk menyatakan bahwa permohonan ditolak. Staf Tata Usaha akan mengisi alasan mengapa permohonan ditolak sehingga tidak membingungkan pemohon.
\item \includegraphics[height=0.7\baselineskip]{tombol_trash.png} berfungsi untuk menghapus permohonan \textbf{secara permanen}. Staf Tata Usaha dihimbau agar tidak menggunakan tombol ini kecuali dalam keadaan terpaksa.
\end{itemize}  

\section{CodeIgniter}
\label{sec:code_igniter}

\subsection{Application Flow Chart}
Gambar berikut mengilustrasikan bagaimana alur data pada sistem :

\begin{figure} [H]
	\centering  
	\includegraphics[scale=1.0]{appflowchart.png}  
	\caption{Flow Chart Aplikasi CodeIgniter}
	\label{fig:flow-chart-CodeIgniter} 
\end{figure}

\begin{enumerate}
\item index.php bertindak sebagai \textit{front controller}, menginisiasi \textit{base resources} yang dibutuhkan untuk menjalankan CodeIgniter.
\item Router akan memeriksa permintaan HTTP untuk menetapkan hal apa yang harus dilakukan dengan permintaan tersebut.
\item Apabila terdapat \textit{cache}, maka \textit{cache} tersebut akan dikirimkan langsung ke browser, dengan melewati sistem eksekusi normal.
\item Keamanan. Sebelum \textit{controller} aplikasi dimuat, \textit{HTTP request} dan \textit{user} mana pun yang mengirimkan data diseleksi dahulu untuk keamanan.
\item Controller memuat \textit{model, core libraries, helpers}, dan \textit{resources} yang dibutuhkan untuk proses \textit{request} yang spesifik.
\item \textit{View} yang telah selesai dirender kemudian dikirim ke \textit{web browser} untuk dilihat. Jika \textit{caching} diaktifkan, tampilan dicache terlebih dahulu sehingga pada permintaan selanjutnya dapat dilayani.\cite{codeigniter:17}
\end{enumerate}

\subsection{CodeIgniter URLs}
\label{subs:urls}
Codeigniter menggunakan pendekatan \textbf{berbasis-segmen}:\cite{codeigniter:17}
\begin{lstlisting}[frame=single] 
example.com/class/function/ID
\end{lstlisting}

\begin{enumerate}
\item Segmen pertama menyatakan kelas \textit{controller} yang harus dipanggil.
\item Segmen kedua menyatakan fungsi kelas, atau metode, yang harus dipanggil.
\item Segmen ketiga dan setiap segmen setelahnya menyatakan ID dan variabel apa pun yang akan diteruskan ke controller.
\end{enumerate}

\subsection{Model}
\label{subs:model}
\textit{Model} merepresentasikan struktur data. Biasanya kelas \textit {model} akan berisi fungsi yang membantu untuk \textit{retrieve, insert}, dan \textit{update} informasi di database.

Dalam Codeigniter \textit{models} merupakan opsi yang tersedia untuk mereka yang ingin lebih menggunakan sebuah pendekatan tradisional MVC.\cite{codeigniter:17}

\subsubsection{Anatomi Model}
\label{sssec:model_1}

Kelas model akan disimpan di direktori \textbf{application/models/directory}. Kelas ini dapat bersarang didalam \textit{sub-directories} jika Anda menginginkan tipe organisasi seperti ini. \cite{codeigniter:17}

Prototipe dasar dari sebuah model kelas :
\begin{lstlisting}[frame=single]  
<?php
class Model_name extends CI_Model {

}
\end{lstlisting}

Nama file juga harus sama dengan nama kelas. Sehingga apabila kita ada kelas \verb|User_model| maka file Anda akan seperti ini.

\begin{lstlisting}[frame=single]  
application/models/User_model.php
\end{lstlisting}

\subsubsection{Loading a Model}
\label{sssec:model_2}

Model Anda biasanya akan dimuat dan dipanggil didalam metode \textit{controller} Anda. Untuk memuat sebuah model anda akan menggunakan metode berikut:\cite{codeigniter:17}

\begin{lstlisting}[frame=single] 
$this->load->model('model_name');
\end{lstlisting}

\subsubsection{Koneksi ke Database}
\label{sssec:model_3}
Apabila model sudah dimuat, model tersebut tidak terhubung secara langsung ke database. Dengan cara secara manual mengatur konektfitas database melalui parameter ketiga:\cite{codeigniter:17}

\begin{lstlisting}[frame=single]
$config['hostname'] = 'localhost';
$config['username'] = 'myusername';
$config['password'] = 'mypassword';
$config['database'] = 'mydatabase';
$config['dbdriver'] = 'mysqli';
$config['dbprefix'] = '';
$config['pconnect'] = FALSE;
$config['db_debug'] = TRUE;

$this->load->model('model_name', '', $config);
\end{lstlisting}{listing only}

\subsection{View}
\label{subs:view}
\textit{View} adalah informasi yang sedang dilihat oleh \textit{user}. Sebuah \textit{View} normalnya menjadi sebuah halaman web, namun dalam CodeIgniter, sebuah \textit{view} dapat menjadi sebuah \textit{page fragment} seperti \textit{header} atau \textit{footer}. Dapat juga menjadi halaman RSS, atau tipe apapun dari "page".

\textit{Views} tidak pernah dipanggil secara langsung, harus dimuat dalam sebuah \textit{controller}. Ingat bahwa dalam \textit{MVC framework}, \textit{controller} bertanggung jawab untuk mengambil \textit{view} tertentu.\cite{codeigniter:17} 

\subsubsection{Membuat sebuah View}
\label{sssec:view_1}
Dengan menggunakan \textit{text editor}, buat sebuah file yang memanggil \verb|blogview.php|, dan isi dengan kode berikut:\cite{codeigniter:17}
\begin{lstlisting}[frame=single, language=html]% 
<html>
<head>
        <title>My Blog</title>
</head>
<body>
        <h1>Welcome to my Blog!</h1>
</body>
</html>
\end{lstlisting}

Kemudian simpan file tersebut di application/views/ directory.

\subsubsection{Loading sebuah View}
\label{sssec:view_1}

View dapat dimuat dengan membuat file \textit{view} dengan syntax berikut:

\begin{lstlisting}[frame=single] 
$this->load->view('name');
\end{lstlisting}

Dimana \textit{name} adalah nama dari file \textit{view}.

Lalu, buka file \textit{controller} yang dibuat sebelumnya bernama Blog.php, dan pindahkan \textit{echo statement} dengan \textit{view loading method.}

\begin{lstlisting}[frame=single] 
<?php
class Blog extends CI_Controller {

        public function index()
        {
                $this->load->view('blogview');
        }
}
\end{lstlisting}

\subsubsection{Memuat Beberapa View}
\label{sssec:view_2}
Codeigniter akan menangani beberapa panggilan dari dalam controller dengan syntax \verb|$this->load->view()|. Apabila ada lebih dari satu panggilan yang terjadi, maka \textit{views} akan dilampirkan secara bersamaan. Berikut ini kode yang digunakan apabila pengembang web ingin mempunyai sebuah \textit{header view}, sebuah \textit{menu view}, sebuah \textit{content view}, dan sebuah \textit{footer view}. \cite{codeigniter:17}
\begin{lstlisting}[frame=single] 
<?php

class Page extends CI_Controller {

        public function index()
        {
                $data['page_title'] = 'Your title';
                $this->load->view('header');
                $this->load->view('menu');
                $this->load->view('content', $data);
                $this->load->view('footer');
        }

}
\end{lstlisting}

\subsubsection{Menyimpan Views didalam \textit{Sub-directories}}
\label{sssec:view_3}
View files dapat disimpan didalam \textit{sub-directories} dengan menyertakan nama direktori yang memuat \textit{view}.
\begin{lstlisting}[frame=single] 
$this->load->view('directory_name/file_name');
\end{lstlisting}

\subsubsection{Menambahkan data dinamis ke View}
\label{sssec:view_4}
Data yang dikirim dari controller menuju view dalam bentuk \textbf{array} atau objek akan dilampirkan dalam parameter kedua dalam metode loading view. \cite{codeigniter:17}
Berikut ini pengguanaan dengan array:
\begin{lstlisting}[frame=single] 
$data = array(
        'title' => 'My Title',
        'heading' => 'My Heading',
        'message' => 'My Message'
);

$this->load->view('blogview', $data);
\end{lstlisting}

Kemudian, penggunaan dengan objek:
\begin{lstlisting}[frame=single] 
$data = new Someclass();
$this->load->view('blogview', $data);
\end{lstlisting}

Sehingga apabila dimasukan ke controller, kode yang ditambahkan adalah:
\begin{lstlisting}[frame=single] 
<?php
class Blog extends CI_Controller {

        public function index()
        {
                $data['title'] = "My Real Title";
                $data['heading'] = "My Real Heading";

                $this->load->view('blogview', $data);
        }
}
\end{lstlisting}

Untuk mengaksesnya dalam file HTML maka dapat digunakan syntax php
\begin{lstlisting}[frame=single] 
<html>
<head>
        <title><?php echo $title;?></title>
</head>
<body>
        <h1><?php echo $heading;?></h1>
</body>
</html>
\end{lstlisting}

\subsection{Controller}
\label{subs:controller}
\textit{Controller} bertindak sebagai sebuah penengah antara Model, View dan \textit{resources} lain yang dibutuhkan untuk proses \textit{HTTP requests} dan menghasilkan sebuah halaman web.

Sebuah \textit{controller} secara sederhana merupakan sebuah file yang dinamakan sehingga dapat dikaitkan dengan URl.\cite{codeigniter:17}
Misalnya untuk URl ini:
\begin{lstlisting}[frame=single] 
<?php
example.com/index.php/blog/
\end{lstlisting}

Dalam contoh diatas, \textit{Codeigniter} berusaha menemukan \textit{controller} bernama Blog.php dan memuatnya. Ketika sebuah nama \textit{controller} sesuai dengan \textit{first segment} dari sebuah URl, maka URl akan memuatnya.\cite{codeigniter:17}

Kode berikut merupakan contoh dari \textit{controller} sederhana.
\begin{lstlisting}[frame=single] 
<?php
class Blog extends CI_Controller {

        public function index()
        {
                echo 'Hello World'
        }
}
\end{lstlisting} 

\subsubsection{Method}
\label{sssec:controller_1}
Dalam sebuah kelas \textit{controller} akan terdapat beberapa method, untuk memanggil fungsi didalamnya maka dapat mengisi segmen kedua dari sebuah url. \cite{codeigniter:17}
\begin{lstlisting}[frame=single] 
<?php
class Blog extends CI_Controller {

        public function index()
        {
                echo 'Hello World!';
        }

        public function comments()
        {
                echo 'Look at this!';
        }
}
\end{lstlisting}

Pemanggilan method index dapat secara otomatis dilakukan apabila segmen kedua kosong.Cara lain untuk menjalankan method comments() dapat dilakukan dengan:
\begin{lstlisting}[frame=single] 
example.com/index.php/blog/index/
\end{lstlisting}

Kemudian untuk memuat method comment dapat dituliskan sebagai berikut:
\begin{lstlisting}[frame=single] 
example.com/index.php/blog/comments/
\end{lstlisting}

\section{Zurb Foundation 6}
\label{sec:zurb_foundation6}

\subsection{Struktur File}
\label{subs:strukturfile_zurb}
\begin{figure} [H]
	\centering  
	\includegraphics[scale=1.0]{filestructure_zurb.png}  
	\caption{Struktur File Zurb Foundation}
	\label{fig:filestructure_zurb} 
\end{figure}

Framework Foundation terdiri dari 3 folder utama:
Folder \textbf{css} terdiri dari semua \textit{CSS Style} yang digunakan dalam Foundation 6. Didalam folder terdapat versi yag diperkecil \verb|foundation.min.css| atau versi yang tidak dikompresi \verb|foundation.css|. Seluruh modifikasi \textit{stylesheets} ditempatkan pada folder ini agar lebih terstruktur.
Folder \textbf{img} tempat meletakkan semua gambar untuk projek web.
Folder \textbf{js} terdiri dari semua file Javascript yang sudah ditentukan sebelumnya.\cite{zurbfoundation:17}

\subsection{Sistem Grid pada Foundation}
\label{subs:grid_zurb}
Penggunaan grid pada \texttt{Foundation} dapat dilakukan dengan menambahkan sebuah elemen dengan sebuah kelas \texttt{.row} sehingga akan membuat sebuah blok horizontal yang berisi kolom vertikal. Kemudian tambahkan kelas \texttt{.column} pada baris tersebut, serta tentukan masing-masing kolom dengan kelas \texttt{.small-#, .medium-#} dan \texttt{.large-#}. 
\textbf{\textit{Foundation}} adalah \textit{mobile-first}. Kode yang dihasilkan dibuat untuk layar kecil terlebih dahulu, dan layar besar akan mewarisi \textit{style} dari kode tersebut. \cite{zurbfoundation:17}
\begin{lstlisting}[frame=single] 
<div class="row">
  <div class="columns small-2 large-4"><!-- ... --></div>
  <div class="columns small-4 large-4"><!-- ... --></div>
  <div class="columns small-6 large-4"><!-- ... --></div>
</div>
<div class="row">
  <div class="columns large-3"><!-- ... --></div>
  <div class="columns large-6"><!-- ... --></div>
  <div class="columns large-3"><!-- ... --></div>
</div>
<div class="row">
  <div class="columns small-6 large-2"><!-- ... --></div>
  <div class="columns small-6 large-8"><!-- ... --></div>
  <div class="columns small-12 large-2"><!-- ... --></div>
</div>
<div class="row">
  <div class="columns small-3"><!-- ... --></div>
  <div class="columns small-9"><!-- ... --></div>
</div>
<div class="row">
  <div class="columns large-4"><!-- ... --></div>
  <div class="columns large-8"><!-- ... --></div>
</div>
<div class="row">
  <div class="columns small-6 large-5"><!-- ... --></div>
  <div class="columns small-6 large-7"><!-- ... --></div>
</div>
<div class="row">
  <div class="columns large-6"><!-- ... --></div>
  <div class="columns large-6"><!-- ... --></div>
</div>
\end{lstlisting}

\begin{figure} [H]
	\centering  
	\includegraphics[scale=0.7]{gridbasic_zurb.png}  
	\caption{Grid pada Zurb Foundation}
	\label{fig:gridbasic_zurb} 
\end{figure}

\subsection{\textit{Navigation} dan \textit{Media Attributes}}
\label{subs:view}
Komponen menu yang fleksibel pada Foundation membuat pembangunan navigasi secara umum lebih mudah karena semua pola memiliki markup yang sama.

\subsubsection{Basic Menu}
\label{sssec:navigation_1}

Semua versi menu terdiri dari sebuah \texttt{<ul>} yang diisi oleh beberapa elemen \texttt{<li>}. Secara default, menu akan berorientasi horizontal.\cite{zurbfoundation:17}

Berikut ini contoh penggunaan kode navigasi pada menu:

\begin{lstlisting}[frame=single] 
<ul class="menu">
  <li><a href="#">One</a></li>
  <li><a href="#">Two</a></li>
  <li><a href="#">Three</a></li>
  <li><a href="#">Four</a></li>
</ul>
\end{lstlisting}

\begin{figure} [H]  
    \centering  
	\includegraphics[scale=0.7,\textwidth]{basicmenu_zurb.png}  
	\caption{\textit{Basic Navigation Menu} pada Foundation}
	\label{fig:gridbasic_zurb} 
\end{figure}

\subsubsection{Item Alignment}
\label{sssec:navigation_2}
Secara default, setiap item dalam menu sejajar ke arah kiri. Menu dapat diubah sejajar ke arah kanan dengan menggunakan kelas \texttt{.align-right} atau kearah tengah dengan menambahkan kelas \texttt{.align-center} ke kelas \texttt{.menu} \cite{zurbfoundation:17}
\begin{lstlisting}[frame=single] 
<ul class="menu align-right">
  <li><a href="#">One</a></li>
  <li><a href="#">Two</a></li>
  <li><a href="#">Three</a></li>
  <li><a href="#">Four</a></li>
</ul>
\end{lstlisting}

\begin{figure} [H]
	\centering  
	\includegraphics[scale=0.7]{basicmenuRight_zurb.png}  
	\caption{Menu \textit{align to right in Foundation}}
	\label{fig:gridbasic_zurb} 
\end{figure}

\begin{lstlisting}[frame=single] 
<ul class="menu align-center">
  <li><a href="#">One</a></li>
  <li><a href="#">Two</a></li>
  <li><a href="#">Three</a></li>
  <li><a href="#">Four</a></li>
</ul>
\end{lstlisting}

\begin{figure} [H]
	\centering  
	\includegraphics[scale=0.7]{basicmenuCenter_zurb.png}  
	\caption{Menu \textit{align to center in Foundation}}
	\label{fig:gridbasic_zurb} 
\end{figure}

\subsubsection{Active State}
\label{sssec:navigation_3}
Kelas \texttt{.is-active} dapat ditambahkan ke dalam tag \texttt{<li>} untuk membuat sebuah \texttt{active state}. \texttt{Active state} bisa diatur dengan  menandai halaman aktif secara dinamis dengan Javascript atau menerapkannya pada \textit{server-side}.\cite{zurbfoundation:17}

\begin{lstlisting}[frame=single] 
<ul class="menu">
  <li class="is-active"><a>Home</a></li>
  <li><a>About</a></li>
  <li><a>Nachos</a></li>
</ul>
\end{lstlisting}

\begin{figure} [H]
	\centering  
	\includegraphics[scale=0.7]{activestatemenu_zurb.png}  
	\caption{Menu \textit{active state menu in Foundation}}
	\label{fig:activestate_zurb} 
\end{figure}


\subsubsection{Text}
\label{sssec:navigation_4}
Karena \textit{padding} item menu digunakan pada tag \texttt{<a>}, maka saat menerapkan item yang berisi teks saja, teks tersebut akan tidak selaras. Untuk menyiasatinya, maka dapat menggunakan kelas \texttt{.menu-text} ke \textit{<li>} dengan menyertakan teks tanpa link.\cite{zurbfoundation:17}

\begin{lstlisting}[frame=single]
 <ul class="menu">
  <li class="menu-text">Site Title</li>
  <li><a href="#">One</a></li>
  <li><a href="#">Two</a></li>
  <li><a href="#">Three</a></li>
</ul>
\end{lstlisting}

\begin{figure} [H]
	\centering  
	\includegraphics[scale=0.7]{menutext_zurb.png}  
	\caption{Menu \textit{active state menu in Foundation}}
	\label{fig:activestate_zurb} 
\end{figure}

\subsection{Komponen CSS}
\label{subs:css_zurb}
Alasan penggunaan \textit{CSS framework} pada Foundation adalah komponen bawaan antarmuka pengguna. Dengan sistem grid dan komponennya, cukup mudah bagi pengembang untuk mengembangkan situs web yang rumit.~\cite{zurb:15:introfoundation}Beberapa komponen tersebut adalah \textit{\textbf{button, tables}} dan \textit{\textbf{forms}}.\cite{zurbfoundation:17}

\subsubsection{Button}
\label{sssec:css_1}
\textit{Basic button} dapat dibuat dengan markup minimal. Karena tombol dapat digunakan untuk banyak tujuan, penting untuk menggunakan tag yang tepat.
\begin{itemize}
  \item Gunakan tag \texttt{<a> }jika tombolnya adalah tautan ke halaman lain, atau tautan ke jangkar di dalam halaman. Umumnya jangkar tidak memerlukan JavaScript untuk berfungsi.
  \item Gunakan tag \texttt{<button>} jika tombol melakukan tindakan yang mengubah sesuatu pada halaman saat ini. Elemen \texttt{<button>} hampir selalu membutuhkan JavaScript agar berfungsi. \cite{zurbfoundation:17}
\end{itemize}\cite{zurbfoundation:17}

\begin{lstlisting}[frame=single] 
<!-- Anchors (links) -->
<a href="about.html" class="button">Learn More</a>
<a href="#features" class="button">View All Features</a>

<!-- Buttons (actions) -->
<button type="button" class="success button">Save</button>
<button type="button" class="alert button">Delete</button>
\end{lstlisting}

\begin{figure} [H]
	\centering  
	\includegraphics[scale=0.7]{basicbutton_zurb.png}  
	\caption{Basic Button pada Foundation}
	\label{fig:gridbasic_zurb} 
\end{figure}

Warna dapat ditambahkan untuk memberikan \textit{buttons} arti yang bermakna.
\begin{lstlisting}[frame=single] 
<a class="button primary" href="#">Primary</a>
<a class="button secondary" href="#">Secondary</a>
<a class="button success" href="#">Success</a>
<a class="button alert" href="#">Alert</a>
<a class="button warning" href="#">Warning</a>
\end{lstlisting}

\begin{figure} [H]
	\centering  
	\includegraphics[scale=0.7]{coloringbutton_zurb.png}  
	\caption{Coloring Button pada Foundation}
	\label{fig:gridbasic_zurb} 
\end{figure}

\subsubsection{Tables}
\label{sssec:css_2}
\begin{lstlisting}[frame=single] 
<table>
  <thead>
    <tr>
      <th width="200">Table Header</th>
      <th>Table Header</th>
      <th width="150">Table Header</th>
      <th width="150">Table Header</th>
    </tr>
  </thead>
  <tbody>
    <tr>
      <td>Content Goes Here</td>
      <td>This is longer content Donec id elit non mi porta gravida at eget metus.</td>
      <td>Content Goes Here</td>
      <td>Content Goes Here</td>
    </tr>
    <tr>
      <td>Content Goes Here</td>
      <td>This is longer Content Goes Here Donec id elit non mi porta gravida at eget metus.</td>
      <td>Content Goes Here</td>
      <td>Content Goes Here</td>
    </tr>
    <tr>
      <td>Content Goes Here</td>
      <td>This is longer Content Goes Here Donec id elit non mi porta gravida at eget metus.</td>
      <td>Content Goes Here</td>
      <td>Content Goes Here</td>
    </tr>
  </tbody>
</table>
\end{lstlisting}

\begin{figure} [H]
	\centering  
	\includegraphics[scale=0.7]{basictable_zurb.png}  
	\caption{Basic Table pada Foundation}
	\label{fig:gridbasic_zurb} 
\end{figure}

\paragraph{Hover State}
Penggunaan Hover State dengan menambahkan kelas \texttt{.hover} untuk sedikit menggelapkan baris tabel.
\begin{lstlisting}[frame=single] 
<table class="hover">
</table>
\end{lstlisting}

\paragraph{Striped}
Secara default, tabel akan memiliki baris yang bergaris. Untuk menghapus garis-garis tersebut dapat menggunakan kelas \texttt{.unstriped} atau dengan mengubah \verb|$table-is-striped| ke \textit{false} untuk menghapus semua strip pada seluruh tabel. Gunakan pula kelas \texttt{.striped} untuk menambahkan strip.\cite{zurbfoundation:17}

\subsubsection{Forms}
\label{sssec:css_3}
Pembuatan sebuah \textit{form} di \textit{Foundation} didesain mudah namum fleksibel. \texttt{Forms} dibuat dengan kombinasi standar dari elemen \texttt{form}, serta \textit{grid rows} dan \textit{columns} atau \textit{cells}. \cite{zurbfoundation:17}

\paragraph{Text Inputs}
Tipe input berikut ini akan membuat sebuah \textit{text field} : \verb|text, date, datetime, datetime-local, email, month, number, password, search, tel, time, url,| dan \verb|week|.

\begin{lstlisting}[frame=single] 
<form>
  <div class="grid-container">
    <div class="grid-x grid-padding-x">
      <div class="medium-6 cell">
        <label>Input Label
          <input type="text" placeholder=".medium-6.cell">
        </label>
      </div>
      <div class="medium-6 cell">
        <label>Input Label
          <input type="text" placeholder=".medium-6.cell">
        </label>
      </div>
    </div>
  </div>
</form>
\end{lstlisting}

\begin{figure} [H]
	\centering  
	\includegraphics[scale=0.7]{input_zurb.png}  
	\caption{Text Input pada Foundation}
	\label{fig:gridbasic_zurb} 
\end{figure}

\paragraph{Text Inputs}
Penggunaan \texttt{select menus} untuk kombinasi beberapa pilihan ke dalam satu menu.
\begin{lstlisting}[frame=single] 
<label>Select Menu
  <select>
    <option value="husker">Husker</option>
    <option value="starbuck">Starbuck</option>
    <option value="hotdog">Hot Dog</option>
    <option value="apollo">Apollo</option>
  </select>
</label>
\end{lstlisting}\cite{zurbfoundation:17}

\subsection{Komponen JavaScript}
\label{subs:javascript_zurb}
Foundation dilengkapi dengan komponen JavaScript untuk menambah fungsionalitas yang rumit. Komponen JavaScript dapat dimasukkan ke dalam proyek \textit{developer} sehingga membuat pengembangan front-end lebih cepat dan lebih mudah.

\subsubsection{Tabs}
\label{subs:tabs_javascript_zurb}
Tab semakin banyak digunakan dalam desain web karena \textit{developer} dapat menyajikan konten secara seragam. Ini memungkinkan \textit{developer} untuk menyimpan banyak dokumen dalam satu \textit{window}. \textit{developer} dapat menggunakan tab sebagai widget navigasi untuk beralih antar konten sehingga menghasilkan tata letak yang sistematis dan bersih. Komponen Tab dari Foundation membantu \textit{developer} melakukan hal itu hanya dengan menambahkan beberapa baris kode. \cite{zurb:15:introfoundation}

\begin{lstlisting}[frame=single] 
<ul class="tabs" data-tabs id="tab_component">
 <li class="tabs-title"><a href="#pub1">Section 1</a></li>
 <li class="tabs-title is-active"><a href="#pub2">Section 2</a></li>
 <li class="tabs-title"><a href="#pub3">Section 3</a></li>
 <li class="tabs-title"><a href="#pub4">Section 4</a></li>
</ul>
<div class="tabs-content" data-tabs-content="tab_component">
 <div class="tabs-panel" id="pub1">
 <p>Far far away, behind the word mountains, far from the countries
Vokalia and Consonantia, there live the blind texts.</p>
 </div>
 <div class="tabs-panel is-active" id="pub2">
 <p> Separated they live in Bookmarksgrove right at the coast of the
Semantics, a large language ocean. </p>
 </div>
 <div class="tabs-panel" id="pub3">
 <p>A small river named Duden flows by their place and supplies it with
the necessary regelialia.</p>
 </div>
 <div class="tabs-panel" id="pub4">
 <p>It is a paradisematic country, in which roasted parts of sentences
fly into your mouth. </p>
 </div>
</div>
\end{lstlisting}

\begin{figure} [H]
	\centering  
	\includegraphics[scale=0.7]{tabs_component_zurb.png}  
	\caption{Grid pada Zurb Foundation}
	\label{fig:gridbasic_zurb} 
\end{figure}

\subsubsection{Dropdown Menu}
\label{subs:dropdown_javascript_zurb}
Berfungsi untuk mengubah menu dasar menjadi menu dropdown yang dapat di-expand dengan plugin Menu Dropdown.
Menu dropdown dibangun berdasarkan sintaks komponen \textbf{Menu}. Tambahkan kelas \texttt{.dropdown} dan atribut \texttt{data-dropdown-menu} ke wadah menu untuk mengatur dropdown. \cite{zurbfoundation:17}
\begin{lstlisting}[frame=single] 
<ul class="dropdown menu" data-dropdown-menu>
  <li><a href="#">Item 1</a></li>
  <li><a href="#">Item 2</a></li>
  <li><a href="#">Item 3</a></li>
  <li><a href="#">Item 4</a></li>
</ul>
\end{lstlisting}

\subsubsection{Reveal}
\label{subs:modals_javascript_zurb}
Modal hanyalah wadah kosong, sehingga \textit{developer} dapat menaruh segala jenis konten di dalamnya, seperti teks ke formulir hingga video ke seluruh \textit{grid}.
Untuk membuat modal, tambahkan kelas \texttt{.reveal}, atribut \texttt{data-reveal}, dan ID yang unik ke dalam \textit{container}.

\begin{lstlisting}[frame=single] 
<div class="reveal" id="exampleModal1" data-reveal>
  <h1>Awesome. I Have It.</h1>
  <p class="lead">Your couch. It is mine.</p>
  <p>I'm a cool paragraph that lives inside of an even cooler modal. Wins!</p>
  <button class="close-button" data-close aria-label="Close modal" type="button">
    <span aria-hidden="true">&times;</span>
  </button>
</div>
\end{lstlisting} 

\section{Bootstrap 4}
\subsection{Sistem Grid Bootstrap}
Sistem grid Bootstrap menggunakan \textit{container}, \textit{rows}, dan \textit{columns} untuk tata letak dan penyelarasan konten. Selain itu sistem ini dibangun dengan \textit{flexbox} dan seluruhnya \textit{responsive}.
\begin{figure} [H]
	\centering  
	\includegraphics[scale=0.7]{gridbasic_bootstrap.png}  
	\caption{Grid pada Bootstrap} 
\end{figure}

\begin{lstlisting}[frame=single] 
<div class="container">
  <div class="row">
    <div class="col-sm">
      One of three columns
    </div>
    <div class="col-sm">
      One of three columns
    </div>
    <div class="col-sm">
      One of three columns
    </div>
  </div>
</div>
\end{lstlisting}
Dalam contoh diatas akan dibuat tiga kolom yang memiliki lebar yang sama baik dalam \textit{device} \textit{small, medium, large} dan \textit{extra large} menggunakan kelas grid yang sudah ditentukan sebelumnya oleh Bootstrap. Penggunaan \verb|.container| akan membuat kolom berada ditengah halaman.

Secara detil, bootstrap bekerja dengan cara:
\begin{itemize}
\item \textit{Container} disediakan agar konten berada ditengah halaman dan mengisi konten tersebut secara horizontal. Penggunaan \verb|.container| untuk menentukan lebar pixel secara responsif atau \verb|.container-fluid| untuk membuat lebar: 100\%  di semua ukuran \textit{viewport} dan perangkat.
\item Sebuah baris akan membungkus kolom - kolom. Setiap kolom akan memiliki \textit{padding} secara horizontal yang disebut \verb|gutter| untuk mengatur jarak antar kolom.
\item Penggunaan flexbox akan membuat lebar pada kolom tidak perlu dispesifikasikan. Misalnya empat variabel dari \verb|.com-sm| akan secara otomatis membuat lebar kolom sebesar 25\%.
\item Kelas kolom menunjukkan jumlah kolom yang ingin digunakan, dengan maksimal 12 kolom per baris. Apabila \textit{developer} menginginkan tiga kolom yang memiliki lebar yang sama maka dapat menggunakan \texttt{.col-4}.
\item Lebar kolom diatur dalam persentase, sehingga kolom akan memiliki lebar yang berubah-ubah dan ukuran bergantung dengan elemen \textit{parent} nya.
\end{itemize}
\subsubsection{Pilihan Grid}
Bootstrap menggunakan px untuk grid breakpoint dan lebar container. Ini dikarenakan lebar \textit{viewport} ditentukan denga satuan pixels.
Berikut ini tabel yang menjelaskan penggunaan kelas grid dalam berbagai perangkat :
\begin{figure} [H]
	\centering  
	\includegraphics[scale=0.7]{gridoption_bootstrap.png}  
	\caption{Pilihan kelas grid pada Bootstrap} 
\end{figure}

\subsection{Konten}
\subsubsection{Tabel}
Dengan penggunaan kelas \verb|.table| pada seluruh tag \texttt{<table>} maka \textit{style} pada bootstrap akan diterapkan, sehingga setiap tabel yang \textit{nested} akan diatur sesuai dengan \textit{parent} nya.
\begin{figure} [H]
	\centering  
	\includegraphics[scale=0.7]{tablebasic_bootstrap.png}  
	\caption{Tabel default pada Bootstrap} 
\end{figure}

\begin{lstlisting}[frame=single] 
<table class="table">
  <thead>
    <tr>
      <th scope="col">#</th>
      <th scope="col">First</th>
      <th scope="col">Last</th>
      <th scope="col">Handle</th>
    </tr>
  </thead>
  <tbody>
    <tr>
      <th scope="row">1</th>
      <td>Mark</td>
      <td>Otto</td>
      <td>@mdo</td>
    </tr>
    <tr>
      <th scope="row">2</th>
      <td>Jacob</td>
      <td>Thornton</td>
      <td>@fat</td>
    </tr>
    <tr>
      <th scope="row">3</th>
      <td>Larry</td>
      <td>the Bird</td>
      <td>@twitter</td>
    </tr>
  </tbody>
</table>
\end{lstlisting}

\subsubsection{Opsi Tabel dengan \textit{head} yang dimodifikasi}
Penggunaan kelas \texttt{.thead-light} atau \texttt{.thead-dark} dapat digunakan untuk membuat \texttt{<thead>} menjadi abu muda atau abu tua.
\begin{figure} [H]
	\centering  
	\includegraphics[scale=0.7]{tablehead_bootstrap.png}  
	\caption{Tabel dengan thead yang dimodifikasi pada Bootstrap} 
\end{figure}

\begin{lstlisting}[frame=single] 
<table class="table">
  <thead class="thead-dark">
    <tr>
      <th scope="col">#</th>
      <th scope="col">First</th>
      <th scope="col">Last</th>
      <th scope="col">Handle</th>
    </tr>
  </thead>
  <tbody>
    <tr>
      <th scope="row">1</th>
      <td>Mark</td>
      <td>Otto</td>
      <td>@mdo</td>
    </tr>
    <tr>
      <th scope="row">2</th>
      <td>Jacob</td>
      <td>Thornton</td>
      <td>@fat</td>
    </tr>
    <tr>
      <th scope="row">3</th>
      <td>Larry</td>
      <td>the Bird</td>
      <td>@twitter</td>
    </tr>
  </tbody>
</table>

<table class="table">
  <thead class="thead-light">
    <tr>
      <th scope="col">#</th>
      <th scope="col">First</th>
      <th scope="col">Last</th>
      <th scope="col">Handle</th>
    </tr>
  </thead>
  <tbody>
    <tr>
      <th scope="row">1</th>
      <td>Mark</td>
      <td>Otto</td>
      <td>@mdo</td>
    </tr>
    <tr>
      <th scope="row">2</th>
      <td>Jacob</td>
      <td>Thornton</td>
      <td>@fat</td>
    </tr>
    <tr>
      <th scope="row">3</th>
      <td>Larry</td>
      <td>the Bird</td>
      <td>@twitter</td>
    </tr>
  </tbody>
</table>
\end{lstlisting}

\subsubsection{Gambar}
Gambar dalam Bootstrap akan memiliki sifat \textit{responsive} dengan menerapkan kelas \texttt{.img-fluid} serta mengatur lebar gambar dengan properties \texttt{max-width: 100\%} dan \texttt{height: auto}. Sehingga gambar tidak pernah lebih besar dari \textit{parent} nya. 

\textit{Developer} dapat menyelaraskan (align) sebuah gambar ke kiri atau kanan dengan \textbf{helper float classes} atau \textbf{text alignment classes}. 
\begin{figure} [H]
	\centering  
	\includegraphics[scale=0.7]{imgalign_bootstrap.PNG}  
	\caption{\it{Menyelaraskan gambar ke kanan dan kiri pada bootstrap}} 
\end{figure}
\begin{lstlisting}[frame=single]
<img src="..." class="rounded float-left" alt="...">
<img src="..." class="rounded float-right" alt="...">
\end{lstlisting}



\subsection{Komponen}
\subsubsection{Formulir}
\textit{Form} pada Bootstrap menyediakan beragam tipe input sesuai dengan kebutuhan \textit{user}. Contohnya penggunaan kelas \texttt{email} untuk \textit{input} email atau \texttt{number} untuk input berupa angka.
\subsubsection{Form Controls}
\textit{Developer} dapat membuat form menggunakan kelas \texttt{.form-control}. Kelas ini terdiri dari beberapa tag seperti tag \texttt{<input>, <select>} dan \texttt{<textarea>}.
\begin{lstlisting}[frame=single, basicstyle=\tiny] 
<form>
  <div class="form-group">
    <label for="exampleFormControlInput1">Email address</label>
    <input type="email" class="form-control" id="exampleFormControlInput1" placeholder="name@example.com">
  </div>
  <div class="form-group">
    <label for="exampleFormControlSelect1">Example select</label>
    <select class="form-control" id="exampleFormControlSelect1">
      <option>1</option>
      <option>2</option>
      <option>3</option>
      <option>4</option>
      <option>5</option>
    </select>
  </div>
  <div class="form-group">
    <label for="exampleFormControlSelect2">Example multiple select</label>
    <select multiple class="form-control" id="exampleFormControlSelect2">
      <option>1</option>
      <option>2</option>
      <option>3</option>
      <option>4</option>
      <option>5</option>
    </select>
  </div>
  <div class="form-group">
    <label for="exampleFormControlTextarea1">Example textarea</label>
    <textarea class="form-control" id="exampleFormControlTextarea1" rows="3"></textarea>
  </div>
</form>
\end{lstlisting}

\begin{figure} [H]
	\centering  
	\includegraphics[scale=0.7]{formsbasic_bootstrap.png}  
	\caption{Forms Basic pada Bootstrap} 
\end{figure} 
\subsubsection{Column Sizing}
Bootstrap memungkinkan \textit{developer} untuk menempatkan sejumlah \texttt{.col} di dalam baris \texttt{.row} atau \texttt{.form} dengan lebar tertentu. Misalnya ada tiga buah kolom, kolom pertama memiliki lebar 7 dengan menggunakan kelas \texttt{.col-7} maka dua kolom sisanya akan memiliki lebar yang  memenuhi baris tersebut.
\begin{figure} [H]
	\centering  
	\includegraphics[scale=0.7]{columnsizing_bootstrap.png}  
	\caption{Forms Basic pada Bootstrap} 
\end{figure} 
\begin{lstlisting}[frame=single] 
<form>
  <div class="form-row">
    <div class="col-7">
      <input type="text" class="form-control" placeholder="City">
    </div>
    <div class="col">
      <input type="text" class="form-control" placeholder="State">
    </div>
    <div class="col">
      <input type="text" class="form-control" placeholder="Zip">
    </div>
  </div>
</form>
\end{lstlisting}
\subsubsection{Disabled Forms}
Penambahan atribut boolean \texttt{disabled} pada sebuah input membuat \textit{user} tidak bisa mengisi data pada \textit{field} tersebut. Untuk non-aktifkan seluruh \textit{field} pada sebuah kolom dapat menambahkan atribut \texttt{disabled} pada tag \texttt{<fieldset>}.
\begin{figure} [H]
	\centering  
	\includegraphics[scale=0.7]{disabledforms_bootstrap.png}  
	\caption{Disabled Basic pada Bootstrap} 
\end{figure}
\begin{lstlisting}[frame=single, basicstyle=\tiny] 
<form>
  <fieldset disabled>
    <div class="form-group">
      <label for="disabledTextInput">Disabled input</label>
      <input type="text" id="disabledTextInput" class="form-control" placeholder="Disabled input">
    </div>
    <div class="form-group">
      <label for="disabledSelect">Disabled select menu</label>
      <select id="disabledSelect" class="form-control">
        <option>Disabled select</option>
      </select>
    </div>
    <div class="form-check">
      <input class="form-check-input" type="checkbox" id="disabledFieldsetCheck" disabled>
      <label class="form-check-label" for="disabledFieldsetCheck">
        Can't check this
      </label>
    </div>
    <button type="submit" class="btn btn-primary">Submit</button>
  </fieldset>
</form>
\end{lstlisting}
\subsubsection{Tombol}
Bootstrap memasukan beberapa button dengan \textit{style} yang sudah didefinisikan sebelumnya, membuat setiap button akan memiliki makna nya sendiri.
\begin{figure} [H]
	\centering  
	\includegraphics[scale=0.7]{buttons_bootstrap.png}  
	\caption{Button pada Bootstrap} 
\end{figure}
\begin{lstlisting}[frame=single] 
<button type="button" class="btn btn-primary">Primary</button>
<button type="button" class="btn btn-secondary">Secondary</button>
<button type="button" class="btn btn-success">Success</button>
<button type="button" class="btn btn-danger">Danger</button>
<button type="button" class="btn btn-warning">Warning</button>
<button type="button" class="btn btn-info">Info</button>
<button type="button" class="btn btn-light">Light</button>
<button type="button" class="btn btn-dark">Dark</button>

<button type="button" class="btn btn-link">Link</button>
\end{lstlisting}
\subsubsection{Button with Dropdowns}
\begin{lstlisting}[frame=single, basicstyle=\tiny]
<div class="input-group mb-3">
  <div class="input-group-prepend">
    <button class="btn btn-outline-secondary dropdown-toggle" type="button" data-toggle="dropdown"
    aria-haspopup="true" aria-expanded="false">Dropdown</button>
    <div class="dropdown-menu">
      <a class="dropdown-item" href="#">Action</a>
      <a class="dropdown-item" href="#">Another action</a>
      <a class="dropdown-item" href="#">Something else here</a>
      <div role="separator" class="dropdown-divider"></div>
      <a class="dropdown-item" href="#">Separated link</a>
    </div>
  </div>
  <input type="text" class="form-control" aria-label="Text input with dropdown button">
</div>

<div class="input-group">
  <input type="text" class="form-control" aria-label="Text input with dropdown button">
  <div class="input-group-append">
    <button class="btn btn-outline-secondary dropdown-toggle" type="button" data-toggle="dropdown"
    aria-haspopup="true" aria-expanded="false">Dropdown</button>
    <div class="dropdown-menu">
      <a class="dropdown-item" href="#">Action</a>
      <a class="dropdown-item" href="#">Another action</a>
      <a class="dropdown-item" href="#">Something else here</a>
      <div role="separator" class="dropdown-divider"></div>
      <a class="dropdown-item" href="#">Separated link</a>
    </div>
  </div>
</div>
\end{lstlisting}

\begin{figure} [H]
	\centering  
	\includegraphics[scale=1.0]{buttonsdropdown_bootstrap.PNG}  
	\caption{Tombol \textit{dropdown} pada Bootstrap} 
\end{figure}

\subsubsection{Navigation Bar}
Navbar pada Bootstrap terdiri dari beberapa sub-komponen yang bisa digunakan sesuai dengan kebutuhan:
\begin{itemize}
    \item \texttt{.navbar-brand} : Komponen untuk menampilkan nama perusahaan, nama produk atau nama proyek.
    \item \texttt{.navbar-nav} : Komponen untuk membuat navigasi memiliki lebar yang memenuhi layar.
    \item \texttt{.navbar-toggler} : Komponen yang digunakan bersamaan dengan plugin untuk membuat efek jatuh dan perilaku navigasi lainnya.
    \item \texttt{.form-inline} : Komponen untuk pengaturan formulir dan aksi.
    \item \texttt{.collapse.navbar-collapse} : Komponen untuk mengelompokkan dan menyembunyikan \textit{navigation bar} dengan sebuah breakpoint induknya.
    %breakpoint : titik dimana terjadi perubahan (xs(br:0px), sm.. )
\end{itemize}
Berikut ini merupakan semua sub-komponen yang termasuk dalam navigation bar, navbar mengimplementasikan tema \texttt{light-themed} yang secara otomatis menyembunyilan menu pada breakpoint \texttt{lg}
\begin{figure} [H]
	\centering  
	\includegraphics[scale=1.0]{navbar_bootstrap.PNG}  
	\caption{Navigation Bar pada Bootstrap} 
\end{figure}

\begin{lstlisting}[frame=single, basicstyle=\tiny] 
<nav class="navbar navbar-expand-lg navbar-light bg-light">
  <a class="navbar-brand" href="#">Navbar</a>
  <button class="navbar-toggler" type="button" data-toggle="collapse" data-target="#navbarSupportedContent" 
  aria-controls="navbarSupportedContent" aria-expanded="false" aria-label="Toggle navigation">
    <span class="navbar-toggler-icon"></span>
  </button>

  <div class="collapse navbar-collapse" id="navbarSupportedContent">
    <ul class="navbar-nav mr-auto">
      <li class="nav-item active">
        <a class="nav-link" href="#">Home <span class="sr-only">(current)</span></a>
      </li>
      <li class="nav-item">
        <a class="nav-link" href="#">Link</a>
      </li>
      <li class="nav-item dropdown">
        <a class="nav-link dropdown-toggle" href="#" id="navbarDropdown" role="button" 
        data-toggle="dropdown" aria-haspopup="true" aria-expanded="false">
          Dropdown
        </a>
        <div class="dropdown-menu" aria-labelledby="navbarDropdown">
          <a class="dropdown-item" href="#">Action</a>
          <a class="dropdown-item" href="#">Another action</a>
          <div class="dropdown-divider"></div>
          <a class="dropdown-item" href="#">Something else here</a>
        </div>
      </li>
      <li class="nav-item">
        <a class="nav-link disabled" href="#">Disabled</a>
      </li>
    </ul>
    <form class="form-inline my-2 my-lg-0">
      <input class="form-control mr-sm-2" type="search" placeholder="Search" aria-label="Search">
      <button class="btn btn-outline-success my-2 my-sm-0" type="submit">Search</button>
    </form>
  </div>
</nav>
\end{lstlisting}

\subsubsection{Modal}
Bagaimana Modal bekerja :
\begin{itemize}
\item Modal dibangun dengan HTML, CSS dan Javascript. 
\item Menekan modal "backdrop" otomatis menutup komponen modal.
\item Bootstrap hanya mendukung satu modal dalam sebuah window pada satu waktu. Penggunaan modal yang bercabang dalam Bootstrap dipercaya memberikan \textit{user experience} yang buruk.
\item Modal menggunakann \texttt{position: fixed} yang diletakkan pada posisi teratas dalam kode agar terhindar dari \textit{bug} yang disebabkan elemen lain yang memiliki posisi \textit{fixed}. 
\end{itemize}
Komponen modal terdiri dari modal headerm modal body dan modal footer (opsional).
\begin{figure} [H]
	\centering  
	\includegraphics[scale=0.5]{livemodal_bootstrap.png}  
	\caption{Modal pada Bootstrap} 
\end{figure}
\begin{lstlisting}[frame=single, basicstyle=\tiny] 
<!-- Button trigger modal -->
<button type="button" class="btn btn-primary" data-toggle="modal" data-target="#myModal">
  Launch demo modal
</button>

<!-- Modal -->
<div class="modal fade" id="exampleModal" tabindex="-1" role="dialog" 
aria-labelledby="exampleModalLabel" aria-hidden="true">
  <div class="modal-dialog" role="document">
    <div class="modal-content">
      <div class="modal-header">
        <h5 class="modal-title" id="exampleModalLabel">Modal title</h5>
        <button type="button" class="close" data-dismiss="modal" aria-label="Close">
          <span aria-hidden="true">&times;</span>
        </button>
      </div>
      <div class="modal-body">
        ...
      </div>
      <div class="modal-footer">
        <button type="button" class="btn btn-secondary" data-dismiss="modal">Close</button>
        <button type="button" class="btn btn-primary">Save changes</button>
      </div>
    </div>
  </div>
</div>
\end{lstlisting}

\subsubsection{Ikon}
Bootstrap tidak memiliki \textit{library} ikon secara \textit{default}, sehingga ikon yang digunakan diambil dari \textbf{Font Awesome}. Penggunaan ikon dengan menggunakan tag \texttt{<i>} yang disertai dengan kelas \texttt{fa} (font-awesome). 

\begin{lstlisting}[frame=single]
<i class="fa fa-coffee"></i>
\end{lstlisting}

\begin{figure} [H]
	\centering  
	\includegraphics[scale=1.0]{fa_coffee.PNG}  
	\caption{Ikon \textit{Coffee} pada Font Awesome} 
\end{figure}
\subsubsection{Alert}
Alert menyediakan pesan umpan balik untuk user untuk berbagai tipe pesan peringatan yang tersedia. Untuk gaya yang sesuai \textit{developer} dapat menggunakan delapan kelas yang tersedia.
\begin{figure} [H]
	\centering  
	\includegraphics[scale=1.0]{alert_bootstrap.PNG}  
	\caption{Alert pada Bootstrap} 
\end{figure}
\begin{lstlisting}[frame=single] 
<div class="alert alert-primary" role="alert">
  This is a primary alert—check it out!
</div>
<div class="alert alert-secondary" role="alert">
  This is a secondary alert—check it out!
</div>
<div class="alert alert-success" role="alert">
  This is a success alert—check it out!
</div>
<div class="alert alert-danger" role="alert">
  This is a danger alert—check it out!
</div>
<div class="alert alert-warning" role="alert">
  This is a warning alert—check it out!
</div>
<div class="alert alert-info" role="alert">
  This is a info alert—check it out!
</div>
<div class="alert alert-light" role="alert">
  This is a light alert—check it out!
</div>
<div class="alert alert-dark" role="alert">
  This is a dark alert—check it out!
</div>
\end{lstlisting}


 

 
